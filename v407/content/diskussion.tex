\section{Diskussion}
\label{sec:Diskussion}

Als Hauptfehlerquellen sind einige Aspekte des Aufbaus und der Geräte heranzuziehen.
Es ist anzunehmen, dass der Polarisationsfilter nicht rein parallel- oder 
senkrecht polarisiertes Licht filtern kann, das soll im Folgenden noch weiter 
thematisiert werden. Weiterhin ist die Ausrichtung von Laser und Spiegel inklusive 
Intensitätsmesser anzumängeln. Zum einen war es mühsam, diese einigermaßen präzise 
aufzubauen, die Ausrichtung erfolgt nach Augenmaß. Darüber hinaus lässt sich 
vermuten, dass das Luxmeter fehlerbehaftet ist. So ergaben sich 
Intensitätsveränderungen bei Kontakt mit der Apparatur, was nicht der Fall sein 
durfte.

\subsection{Brechungsindizes}
Aus den Berechnungen ergaben sich die Brechungsindizes
\begin{align*}
    \overline{n_\text{parallel}}  &= \qty{2.50(0.32)}{} \\
    \overline{n_\text{senkrecht}} &= \qty{1.48(0.05)}{}
\end{align*}
\noindent Die signifikante Abweichung zwischen den
beiden Werten ($\increment n = 1.02$) deutet auf eine starke Doppelbrechung des
Materials hin. Die vergleichsweise große Unsicherheit bei $\overline{n_\text{parallel}}$
von $\pm 0.32$ lässt sich auf mehrere Faktoren zurückführen, wie die bereits 
angesprochenen Polarisationsfilterprobleme oder die manuelle Ausrichtung von 
Instrumenten, was zu einer Vermischung der Polarisationsebenen führt.
Dies äußert sich besonders stark bei der Messung der parallelen Komponente.
Bei dem Vergleich mit der Literatur wird der Verdacht einer sehr fehlerbelasteten 
Messung bestätigt. Ausgehend von einem Wert von $n_{Lit} = 3.42$ \cite{brisi} 
für Silicium würde das einer Abweichung von $26.9 \%$ bzw. $56.73 \%$ entsprechen. 
Für die Bestimmung des Brechungsindex mithilfe des Brewster Winkels von 
$\theta_\text{Brewster}=75\unit{\degree}$ ergab sich ein Wert von 
\begin{equation*}
    n_{Brewster} = 0.96,
\end{equation*}
der deutlich unter dem zu erwartenden Literaturwert von $3,42$ für Silicium liegt.
Dieser würde weit mehr abweichen mit einem Prozentwert 
von $71.93$. Eine derartig hohe Abweichung ist ungewöhnlich, lässt sich jedoch
durch mehrere systematische Fehlerquellen erklären. Bereits der gemessene
Brewster-Winkel wich signifikant vom theoretisch vorhergesagten Wert ab (mehr 
dazu in \autoref{sec:disb}). Diese systematischen Unsicherheiten potenzieren sich
bei der Berechnung des Brechungsindex und führen so zu der beobachteten starken
Abweichung vom Referenzwert.

\subsection{Brewsterwinkel}
\label{sec:disb}
Da der Polarisationsfilter nicht so funktioniert, wie vorgesehen, unterliegt die
Bestimmung des Brewsterwinkels dem systematischen Fehler. Trotzdem ist zu erkennen, 
dass nach \autoref{tab:10} die Intensität im Bereich zwischen x und y 
sehr abnimmt. Die genauere Bestimmung des experimentell bestimmten Wertes 
belief sich auf
\begin{equation*}
    \theta_\text{Brewster} = \qty{68.17}{\degree}.
\end{equation*} 
\noindent In dem Diagramm \autoref{fig:10} ist jedoch zu sehen dass der Brewster Winkel ziemlich genau 
bei $\alpha = \qty{75}{\degree}$ liegen sollte. Dies ist vermutlich den Messwerten und den damit einhergehenden systematischen 
Fehlern zuzuschreiben. 
Der Literatur nach liegt der Brewster-Winkel bei Silicium bei $74°$ \cite{brewster}.
Dementsprechend ist die Bestimmung via Diagramm wesentlich präziser mit einer 
Abweichung von $1.35 \%$ im Gegensatz zur Berechnung, welche sich um $10 \%$
von der Literatur unterscheidet. Die Brewster-Methode setzt eine perfekt glatte
Oberfläche voraus, was im Versuch nur sehr schwer realisierbar ist.
Oberflächenkontaminationen oder Oxidschichten auf Silicium
reduzieren die reflektierte Intensität und verfälschen $\theta_{Brewster}$.

\subsection{Diagramm zum Vergleich}
In \autoref{fig:10} sind die Messewerte veranschaulicht. Es fällt auf, dass 
die experimentellen Werte keine große Ähnlichkeit zu der Theorie zeigen. Die 
gemessenen Werte bei der senkrechten Polarisation beginnen zwar wie gemutmaßt, 
allerdings ist der weitere Verlauf durch eine zu starke Steigung gekennzeichnet. 
Für die parallele Polarisation liegen die Werte zunächst zu weit unterhalb der 
Theoriekurve. Ebenso abweichend ist der restliche Verlauf; zwar gibt es ein 
erkennbares Minimum, jedoch fällt dieses später an, als vorgesehen. Dazu ist 
ein dezenter Anstieg vor Abfall zum Minimum erkennbar, was unregelmäßig ist. 
Für diese Unstimmigkeiten kann es zahlreiche Ursachen geben, als Hauptproblem 
ist vermutlich die Justierung der Instrumente zu identifizieren. Wie schon 
erwähnt, erfolgt die Justierung via Augenmaß und ist dementsprechend bedingt 
präzise. Die systematischen Abweichungen legen nahe, dass neben den 
offensichtlichen Installationsproblemen weitere systematische Fehlerquellen 
existieren. Besonders der unerwartete Anstieg vor dem Minimum bei paralleler
Polarisation deutet auf zusätzliche physikalische Effekte hin, die im 
theoretischen Modell nicht berücksichtigt wurden. Für zukünftige Experimente
wäre eine präzisere mechanische Führung der Komponenten und eine automatische
Winkelmessung empfehlenswert.