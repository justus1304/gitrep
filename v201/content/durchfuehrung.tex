\section{Durchführung}
\label{sec:Durchführung}

\subsection{Aufbau}
Eine senkrecht hängende Feder ist oberhalb mit einem Kraftmesser verbunden. Dieser gibt jene Kraft 
auf dem Bildschirm aus, welche erzeugt wird, wenn von unten an der Feder gezogen wird.


Das untere Ende der Feder ist mit einem Seil verbunden, welches entlang eines 
Maßbandes (Messgröße in mm) bewegt werden kann, um die Feder so aus ihrer 
Ruhelage auszulenken. Nun ist der  Verschiebungsweg $\Delta x$ abzulesen.


Die Feder wird in je 2cm-Schritten zehn Mal von 0cm bis 20cm aus der Ruhelage 
ausgelenkt. Die Kraft wird jedes mal abgetragen und ein Mittelwert berechnet.