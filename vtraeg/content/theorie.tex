\section{Theorie}
\label{sec:Theorie}

Das Trägheitsmoment entspricht der Kraft, welche gegen die Änderung der
Winkelgeschwindigkeit $\omega$ wirkt. Ausgehend von einer punktförmigen 
Masse m mit einem Radius r (der Abstand zur Drehachse) lässt sich das 
Trägheitsmoment berechnen durch

\begin{equation}
    I = m r^2 .
\end{equation}

Dreht sich ein Körper um eine (beliebige) Achse, so bewegen sich auch alle
Elemente dieses Körpers mit identischer Winkelgeschwindigkeit um Diese. Das
gesamte Trägheitsmoment eines Körpers berechnet sich somit aus der Summe

\begin{equation}
    I = \sum\limits_{i} m_i r_i^2
\end{equation}

der einzelnen Trägheitsmomente der verschiedenen Masseelementen. Für
infinitesimale Massen gilt

\begin{equation}
    I = \int r^2 dm .
\end{equation}

Das Trägheitsmoment hängt dabei von Lage und Objekt ab. Sei die
Achse, um die sich das Objekt drehe, jeweils in positiver z-Richtung liegend.
Dann wären die Trägheitsmomente folgender Objekte folgendermaßen definiert:\\
Langer, infinitesimal dünner Stab mit Länge l
\begin{equation}
    I = \frac{ml^2}{12}
\end{equation}
Kugel
\begin{equation}
    I = \frac{2}{5} m r^2
\end{equation}
Zylinder (stehend)
\begin{equation}
    I = \frac{1}{2} m r^2
\end{equation}
Zylinder (liegend) mit Höhe h
\begin{equation}
    I = m \left(\frac{r^2}{4} + \frac{h^2}{12} \right) 
\end{equation}

\cite{sample}
