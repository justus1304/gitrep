\section{Auswertung}
\label{sec:Auswertung}

\subsection{Verdampfungswärme $L$ im bereich bis 1 bar}
Hier soll die Verdampfungswärme $\symbf{L}$ für Wasser im Bereich von 
$48 - 1000$ \unit{\milli\bar} experiementell ermittelt werden. Bei der Verdampfung von Wasser
wurden folgende Werte für die Dampftemperatur $\symbf{T}/\unit{\celsius}$ und
den zugehörigen Druck $\symbf{p}/\unit{\milli\bar}$ gemessen.  

  \begin{longtblr}[caption = {Messwerte Temperatur/Druck für $p < 1$ bar}]{
      colspec = {S S },
      row{1} = {guard, mode=math},
      vline{2} = {2}{-}{text=\clap{$\pm$}},
      vline{4} = {2}{-}{text=\clap{$\pm$}},
      %vline{4} = {2}{-}{text=\clap{$\pm$}},
    }
    \toprule
    \SetCell[c=2]{c} T/\unit{\celsius}& &\SetCell[c=2]{c} p/\unit{\milli\bar}\\
    \midrule
    19 & 1 & 48  & 1     \\   
    20 & 1 & 51  & 1     \\   
    21 & 1 & 53  & 1     \\   
    22 & 1 & 56  & 1     \\   
    24 & 1 & 59  & 1     \\   
    24 & 1 & 60  & 1     \\   
    25 & 1 & 63  & 1     \\   
    26 & 1 & 64  & 1     \\   
    27 & 1 & 66  & 1     \\   
    28 & 1 & 67  & 1     \\   
    29 & 1 & 67  & 1     \\   
    26 & 1 & 71  & 1     \\   
    27 & 1 & 75  & 1     \\   
    28 & 1 & 80  & 1     \\   
    29 & 1 & 83  & 1     \\   
    30 & 1 & 85  & 1     \\   
    31 & 1 & 87  & 1     \\   
    32 & 1 & 90  & 1     \\   
    33 & 1 & 96  & 1     \\   
    34 & 1 & 99  & 1     \\   
    35 & 1 & 102 & 1     \\   
    36 & 1 & 105 & 1     \\   
    37 & 1 & 109 & 1     \\   
    38 & 1 & 113 & 1     \\   
    39 & 1 & 117 & 1     \\   
    40 & 1 & 121 & 1     \\   
    41 & 1 & 125 & 1     \\   
    42 & 1 & 129 & 1     \\   
    43 & 1 & 134 & 1     \\   
    44 & 1 & 137 & 1     \\   
    45 & 1 & 142 & 1     \\   
    46 & 1 & 147 & 1     \\   
    47 & 1 & 151 & 1     \\   
    48 & 1 & 156 & 1     \\   
    49 & 1 & 162 & 1     \\   
    50 & 1 & 167 & 1  \\
    51 & 1 & 172 & 1\\
    52 & 1 & 177 & 1\\
    53 & 1 & 183 & 1\\
    54 & 1 & 189 & 1\\
    55 & 1 & 195 & 1\\
    56 & 1 & 200 & 1\\
    57 & 1 & 206 & 1\\
    58 & 1 & 212 & 1\\
    59 & 1 & 220 & 1\\
    60 & 1 & 226 & 1\\
    61 & 1 & 234 & 1\\
    62 & 1 & 242 & 1\\
    63 & 1 & 250 & 1\\
    64 & 1 & 259 & 1\\
    65 & 1 & 268 & 1\\
    66 & 1 & 277 & 1\\
    67 & 1 & 287 & 1\\
    68 & 1 & 298 & 1\\
    69 & 1 & 309 & 1\\
    70 & 1 & 322 & 1\\
    71 & 1 & 336 & 1\\
    72 & 1 & 349 & 1\\
    73 & 1 & 363 & 1\\
    74 & 1 & 378 & 1\\
    75 & 1 & 394 & 1\\
    76 & 1 & 410 & 1\\
    77 & 1 & 427 & 1\\
    78 & 1 & 443 & 1\\
    79 & 1 & 460 & 1\\
    80 & 1 & 476 & 1\\
    81 & 1 & 497 & 1\\
    82 & 1 & 517 & 1\\
    83 & 1 & 536 & 1\\
    84 & 1 & 557 & 1\\
    85 & 1 & 575 & 1\\
    86 & 1 & 594 & 1\\
    87 & 1 & 633 & 1\\
    88 & 1 & 646 & 1\\
    89 & 1 & 658 & 1\\
    90 & 1 & 690 & 1\\
    91 & 1 & 713 & 1\\
    92 & 1 & 741 & 1\\
    93 & 1 & 765 & 1\\
    94 & 1 & 794 & 1\\
    95 & 1 & 821 & 1\\
    96 & 1 & 852 & 1\\
    97 & 1 & 881 & 1\\
    98 & 1 & 911 & 1\\
    99 & 1 & 948 & 1\\
    100& 1 & 975 & 1\\   
   \bottomrule
  \end{longtblr}

\noindent Um aus diesen Messwerten einen Ausdruck für die Verdampfungswärme
 $\symbf{L}$ zu bekommen, nutzen wir die in der Theorie erwähnte 
 Gleichung
 \begin{equation}
  \ln \left(\frac{p}{p_0}\right) = -\frac{L}{R} \cdot \frac{1}{T}.
 \end{equation}
Da angenommen wird, dass $\symbf{L}$ im Druckbereich von bis zu ein Bar 
als konstant angesehen werden kann, kann Verdmpfungswärme über
eine lineare Ausgleichsrechnung mit $\ln\left(\frac{p}{p_0}\right)$ in 
Abhängigkeit von $\frac{1}{T}$ berechnet werden.
Für den Parameter $a$ der Ausgleichsgeraden mit der Geradengleichung
\begin{equation}
  y = m \cdot x + b
\end{equation}
gilt dann 
\begin{equation*}
  \label{eqn:a}
  a = -\frac{L}{R}.
\end{equation*}

\begin{figure}[H]
  \centering
  \includegraphics{linreg.pdf}
  \label{fig:1}
  \caption{Lineare Ausgleichsrechnung}
\end{figure}

\noindent Mit den Parametern der Ausgleichsgeraden $a = (\num{-3874 +- 33})\unit{\kelvin}
$ und $b = (\num{21.7 +- 0.1})$ ergibt sich mit \autoref{eqn:a}
umgestellt zu
\begin{equation}
  L = -Ra
\end{equation}
mit der idealen Gaskonstante $R =  8,314[\unit{\joule\per\mol\per\kelvin}] $ 
folgender Wert für die Verdampfungswärme:
\begin{equation}
  \label{eqn:vdw}
  L = \num{3.22 +- 0.02 e4} \unit{\joule\per\mol}.
\end{equation}
Nun wird die äußere Verdampfungswärme $\symbf{L_a}$ mithilfe der idealen 
Gasgleichung (\autoref{eqn:igg}) abgeschätzt. $L_a$ ist die Arbeit, welche 
in Form von Wärme an einem Mol Wasser verrichtet werden muss, um es
in den gasförmigen Aggregatzustand zu überführen. 
\begin{equation}
  L_a = RT
\end{equation}
Für eine Temperatur von $373 \unit{\kelvin}$ ergibt sich eine äußere 
Verdampfungswärme von 
\begin{equation*}
  L_a = \qty{3101.12}{\joule\per\mol}.
\end{equation*}
Damit lässt sich durch Einsetzen in \autoref{eqn:l} jene Arbeit $\symbf{L_i}$
berechnen, welche notwendig ist um die molekularen Anziehungskräfte 
bei der Verdampfung zu überwinden.
\begin{equation}
  \label{eqn:l}
  L_i = L - L_a
\end{equation}
Einsetzen der Werte für die Verdampfungswärme aus \autoref{eqn:vdw} und der 
eben berechneten äußeren Verdampfungswärme ergibt den Wert $L_i = \qty{2.912(0.028)e04}{\joule\per\mol}$.
Um das Ergebnis in Elektronenvolt $(\unit{\eV})$ pro Molekül angeben zu können, wird es durch die 
Avogadro Konstante $N_A = 6.022 \cdot 10^23 \quad \unit{\per\mol}$,
die Anzahl der Moleküle pro einem Mol, geteilt, was den Wert den Wert 
\qty{0.301(0.002)}{\eV} ergibt (mit $\qty{1}{\eV} = \qty{1.602e-19}{\joule}$)

Zusätzlich die ideale Gasgleichung:
\begin{equation}
  \label{eqn:igg}
  pV = RT
\end{equation}



\subsection{Verdampfungswärme im Bereich 1-15 Bar}
Nun wird die Verdampfungswärme im Bereich von 1 - 15 Bar bestimmt. 
Da in diesem Bereich $V_D$ in \autoref{eqn:2} nicht mehr durch die ideale 
Gasgleichung beschrieben werden kann, wird die Verdampfungswärme in diesem 
Druckbereich temperaturabhängig. $V_D$ kann nun über folgende Formel bestimmt 
werden.
\begin{equation}
  \left(p + \frac{a}{V^2}\right)V = RT \Leftrightarrow V_D = \frac{RT}{2p} \pm \sqrt{\left(\frac{RT}{2p}\right)^2-\frac{a}{p}}
\end{equation}
Dabei ist $a$ eine Konstante der Größe $0.9 \unit{\joule\cubic\meter\per\square\mole}$.

  %Tabelle Messwerte
\begin{table}[h]
  \centering 
  \caption{}
  \label{tab:messwerte2}
  \begin{tblr}{
  colspec = {S S },
  row{1} = {guard, mode=math},
    %vline{4} = {2}{-}{text=\clap{$\pm$}},
  }
  \toprule
  T/\unit{\celsius} & p/\unit{\bar}\\
  \midrule    
  118 & 1 \\
  131 & 2\\
  148 & 3\\
  155 & 4\\
  161 & 5\\
  166 & 6\\
  172 & 7\\
  176 & 8\\
  181 & 9\\
  185 & 10\\
  188 & 11\\
  194 & 12\\
  194 & 13\\
  203 & 14\\                                                                                              
 \bottomrule
  \end{tblr}
\end{table}

\noindent Die Messwerte aus Tabelle \autoref{tab:messwerte2} werden nun in einem T/p 
Diagramm durch ein Ausgleichspolynom vom Grad 3 genähert. um daraus den 
Differentialkoeffizienten $dp/dT$ bestimmen zu können, wird das 
Ausgleichpolynom nach der Temperatur abgeleitet.

\begin{figure}[H]
  \centering
  \includegraphics[width=0.7\textwidth]{ausgleichspolynom.pdf}
  \label{fig:2}
  \caption{Ausgleichspolynom 3. Grades und Messwerte für $p > 1\unit{\bar}$}
\end{figure}

\noindent Es ergibt sich das Ausgleichspolynom der Form 
\begin{equation*}
  p(T) = a \cdot x^3 + b \cdot x^2 + c \cdot x + d 
\end{equation*}
\noindent und die resultierende Ableitung
\begin{equation*}
  \frac{dp(T)}{dT} = 3a \cdot x^2 + 2b \cdot x + c 
\end{equation*}
mit den Werten der Parameter des Ausgleichspolynoms:
\begin{align*}
  a &= \num{-1.0 +- 0.7}\unit{\pascal\per\cubic\kelvin}\\
  b &= \num{1531 +- 959}\unit{\pascal\per\squared\kelvin}\\
  c &= \num{-714342 +- 415959}\unit{\pascal\per\kelvin}\\
  d &= \num{108512685 +- 59968812}\unit{\pascal}\\
\end{align*}

\noindent Damit ist der Differentialkoeffizienten $dp/dT$ über die
Ausgleichsrechnung dargestellt. Nun ist es möglich, durch Einsetzen in
\autoref{eqn:2}, umgestellt nach der Verdampfungswärme $L$,
diese in Abhängigkeit der Temperatur bestimmen.
\begin{equation}
L(T) = T \left(\frac{RT}{2p}\pm\sqrt{\left(\frac{RT}{2p}\right)^2-\frac{a}{p}}\right) \frac{dp}{dT}
\end{equation}
\noindent Diese Werte sind in \autoref{fig:e} und \autoref{fig:3} 
für die Fälle der Addition, sowie der Subtraktion der 
Wurzel aufgetragen.

% \begin{figure}[h]
%   \centering
%   \begin{minipage}{0.45\textwidth}
%       \centering
%       \includegraphics[width=\textwidth]{minus.pdf}
%       \caption{Verdampfungswärme bei Subrtaktion der Wurzel}
%       \label{fig:e}
%   \end{minipage}
%   \hfill
%   \begin{minipage}{0.45\textwidth}
%       \centering
%       \includegraphics[width=\textwidth]{plus.jpg}
%       \caption{Verdampfungswärme bei Addition der Wurzel}
%       \label{fig:3}
%   \end{minipage}
% \end{figure}






negative wurzel
\begin{figure}[H]
  \centering
  \includegraphics[width=0.5\textwidth]{minus.pdf}
  \caption{Verdampfungswärme bei Subrtaktion der Wurzel}
  \label{fig:e}
\end{figure}

%positive wurzel
\begin{figure}[H]
  \centering
  \includegraphics[width=0.5\textwidth]{plus.pdf}
  \caption{Verdampfungswärme bei Addition der Wurzel}
  \label{fig:3}
\end{figure}


