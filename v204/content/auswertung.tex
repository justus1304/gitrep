\section{Auswertung}
\label{sec:Auswertung}

\subsection{statische Methode}
In diesem Kapitel werden die Unterschiede und Gemeinsamkeiten der vier 
Temperaturverläufe $T1$, $T4$, $T5$, $T8$ zu den unterschiedlichen Materialien 
ausgewertet. Weiterhin wird der Stab mit der besten Wärmeleitung bestimmt und 
der Wärmestrom zu jeder Messzeit. Letztlich werden noch die Temperaturdifferenzen 
$\Delta T_{ST}=T7-T8$ und $T_{ST}=T2-T1$ als Funktionen der Messzeit t dargestellt.

\subsubsection{Temperaturverläufe und Wärmeleitung}
\begin{figure}
    \centering
    \includegraphics{tempverlaufT1T4.pdf}

\end{figure}

\subsection{Dynamische Methode}
Hier wird die Wärmeleitfähigkeit für den Breiten Messingstab, den Aluminiumstab und den Elelstahlstab
mithilfe der in der \autoref{sec:Auswertung} erwähnten Dynamischen methode ermittelt.

\subsubsection{Wärmeleitfähigkeit Messing}
Zur berechnung der Wärmeleitfähigkeit von messung werden Die messwerte des dicken Messsingstabes
mit den Temperaturmesselementen $T_1$ und $T_2$ benutzt. Jene Wärmeleitfähigkeit lässt sich mit
\autoref{} bestimmen. Darin sind $\Delta x$ der Abstand der Temperaturmesselemente, $c$ die Materialabhängige
Wärmekapazität, $\rho $ die Dichte des Stabes, $\Delta t$ die Phasendifferenz
zwischen der Temperaturwelle am Ersten und zweiten Temperaturelement und $A_{nah(fern)}$ 
die Amplitude der Jewailigen Temperaturwellle. Die Ampituden werden durch die 
funktion "find_peaks" aus der bibiliothek SciPy, angewendet aus die Messdaten, bereitgestellt. 

\begin{table}[H]
    \centering
    \caption{Ermittelte Amplituden und Phasendifferenz}
    \label{tab:maßePuppe}
    %\sisetup{table-format=1.1, per-mode=reciprocal}
    \begin{tblr}{
        colspec = {S S S S},
        row{1} = {guard, mode=math},
        row{14} = {guard},
        %vline{4} = {2}{-}{text=\clap{$\pm$}},
      }
      \toprule
        A_{nah} & A_{fern} & \log{\frac{A_{nah}}{A_{fern}}}
      \midrule
      \bottomrule
    \end{tblr}
  \end{table}

  \begin{figure}
    \centering
    \includegraphics{messingPot}
  \end{figure}
  