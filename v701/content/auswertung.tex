\section{Auswertung}
\label{sec:Auswertung}

\subsection{Energieverlust}
Um den Energieverlust zu berechnen wird die effektive Weglänge gegen die Energie aufgetragen und der Energieverlust
durch eine lineare Ausgleichsrechnung bestimmt.
Um die effektive Weglänge zu berechnen werden die Werte für den Druck aus \autoref{tab:11} und \autoref{tab:12} für die jewailigen Abstände von $x=4$ und $x=7\unit{\centi\meter}$ in \autoref{eqn:eff}
eingesetzt. Die Energie wird in einer Dreisatzrechnung bestimmt, wobei von einer maximalen Energie $E = \qty{4}{\mega\electronvolt}$
bei einem Druck $p = \qty{0}{\bar}$ ausgegangen wird. Es wird von einer lineare Energieskala ausgegangen. Die Ausgleichsrechnungen sind
\autoref{fig:11} und \autoref{fig:12} zu entnehmen. 

\begin{table}[H]
  \centering
  \caption{Druck, Anzahl der pulse und Channel für einen Abstand von $x=4$.}
  \label{tab:11}
  \begin{tblr}{
          colspec = {S S S S S S},
          row{1} = {guard, mode = math},
      }
      \toprule
      p/\unit{\bar}&\text{Anzahl Impulse}&\text{Channel}\\
      \midrule
      0   & 59397  & 1159 \\
      50  & 58355  & 1082\\
      100 & 57520  & 1039\\
      150 & 56528  & 1023\\
      200 & 56276  & 967\\
      250 & 56068  & 896\\
      300 & 55243  & 870\\
      350 & 55010  & 879\\
      400 & 54460  & 768\\
      450 & 53865  & 768\\
      500 & 52403  & 687\\
      550 & 51494  & 666\\
      600 & 50589  & 612\\
      650 & 48451  & 534\\
      700 & 44760  & 474\\
      750 & 36158  & 391\\
      800 & 19362  & 366\\
      850 & 3498   & 363\\
      900 & 501    & 367\\
      \bottomrule
  \end{tblr}
\end{table}
\begin{table}[H]
  \centering
  \caption{Druck, Anzahl der pulse und Channel für einen Abstand von $x=7$.}
  \label{tab:12}
  \begin{tblr}{
          colspec = {S S S S S S},
          row{1} = {guard, mode = math},
      }
      \toprule
      p/\unit{\bar}&\text{Anzahl Impulse}&\text{Channel}\\
      \midrule
      0   & 11903 & 1140\\
      50  & 11615 & 1000\\
      100 & 11442 & 943\\
      150 & 10859 & 805\\
      200 & 11039 & 763\\
      250 & 10634 & 640\\
      300 & 9397  & 514\\
      350 & 5680  & 386\\
      400 & 33    & 363\\
      \bottomrule
  \end{tblr}
\end{table}

\begin{figure}[H]
  \centering
  \caption{Bestimmung des Energieverlusts bei einem Abstand von $x = \qty{4}{\centi\meter}$}
  \label{fig:11}
  \includegraphics{build/abstand4.pdf}
\end{figure}
Der steigungsparameter der Ausgleichsrechnung gibt den Energieverlust an, welcher sich damit bei einem Abstand von 
$x = \qty{4}{\centi\meter}$ zu 
\begin{equation}
  m_\text{4cm} = \qty{-81.26(2.24)}{\mega\electronvolt\per\meter}
\end{equation}

\begin{figure}[H]
  \centering
  \caption{Bestimmung des Energieverlusts bei einem Abstand von $x = \qty{7}{\centi\meter}$}
  \label{fig:12}
  \includegraphics{build/abstand4.pdf}
\end{figure}
Bei einem Abstand von $x = \qty{7}{\centi\meter}$ ergibt sich durh den Steigungsparameter 
ein Energieverlust von 
\begin{equation}
  m_\text{7cm} = \qty{-99.14(3.83)}{\mega\electronvolt\per\meter}
\end{equation}

\subsection{Mittlere Reichweite}
Um die Mittlere Reichweite zu bestimmen wird die Anzahl der Pulse gegen die Effektive Weglänge 
aufgetragen. durch den Stärksten Abfall wird eine Ausgleisgerade gezogen und dann der Schnittpunkt dieser geraden mit einer geraden Auf 
Höhe der halben maximalen Intensität gesucht. Die Ausgleichsrechnungen, sowie die eignezeichnete Gerade und 
der Schnittpunkt befinden sich für die Jewailigen abstände in \autoref{fig:13} und \autoref{fig:14} abgebildet.

\begin{figure}[H]
  \centering
  \caption{Messwerte der Pulse pro $120\unit{\second}$ und bestimmung der mittleren Reichweite für den Abstand $x = \qty{4}{\centi\meter}$}
  \label{fig:13}
  \includegraphics{build/anzahl4.pdf}
\end{figure}

\begin{figure}[H]
  \centering
  \caption{Messwerte der Pulse pro $120\unit{\second}$ und bestimmung der mittleren Reichweite für den Abstand $x = \qty{7}{\centi\meter}$}
  \label{fig:14}
  \includegraphics{build/anzahl7.pdf}
\end{figure}

Die schnittpunkte der Ausgleichgeraden mit der konstanten geraden auf höhe der halben Maximalen pulsrate 
geben die mittlere Reichweite an. Die Reichweiten ergeben sich damit zu 
\begin{align}
  R_\text{4cm} = \qty{53(9)}{\centi\meter}\\
  R_\text{7cm} = \qty{24(4)}{\centi\meter}
\end{align}