\section{Diskussion}
\label{sec:Diskussion}
Die Parameter im ersten Teil des Versuches sind Für den Versuch uninteressant. Wichtiger ist der gut erkennbare 
sinusförmige Verlauf der Spannungsamplitude, der das eingangssignal wiedergibt. Die Parameter 
wären für einen vergleich interessant, sofern man die Messungen bei eingeschaltetem Rauschen 
noch einmal durchführen würde, was auch der Hintergedanke dieses Experiemntes wäre, jedoch wird 
das nicht Durchgeführt, da die Geräte zu fehleranfällig sind, um ein rauschen in einem richtigen Maß
hinzuzufügen. 

\subsection{Verifizierung durch Leuchtdiode}
Eigentlich sollte hier eine $\frac{1}{x^2}$ Abhängigkeit rauskommen, da diese nicht einmal 
näherungsweise im Fehlerbereich des Parameters b liegt, lässt dies auf grobe Ungenauigkeiten
im Experiment hindeuten.