\section{Diskussion}
\label{sec:Diskussion}

\subsection{Mittlere freie Weglänge}
Durch \autoref{sec:mfw} kann darauf geschlossen werden, dass die Franck-Hertz
Kurven bei niedrigen Temperaturen bis ca. $150 \unit{\celsius}$ nicht entstehen. 
Jenes stimmt mit der Theorie überein; Bei geringer Temperatur ist der Dampfdruck 
des Gases ebenso gering, was in einer geringen Anzahl an Stößen zwischen 
Elektronen und Atomen resultiert.
Das zeigt sich in der hohen mittleren freien Weglänge des Elektrons im
Niedrigtempratur-Bereich.

\subsection{Franck-Hertz Kurven}
Als Literaturwert für die Anregungsenergie der Quecksilberatome sing 
$4.9 \unit{e\volt}$ gegeben \cite{anregung}, was der mittlere Differenz der
Maxima entsprechen würde.
Der experimentell bestimmte Werte liegt bei $5.5 \unit{\volt}$, was einer Abweichung
von $11 \%$ entspricht. Das kann mehrere Ursachen haben: zum einen kann die
Ungenauigkeit durch Ablesen der Steigung vom Millimeterpapier als 
Fehlerquelle herangezogen werden. Andererseits unterliegt das Experiment dem
statistischen Fehler, da auch Abweichungen des XY-Schreibers miteinbezogen 
werden müssen. Als Hauptfehlerquelle kann jedoch die Temperaturregulation des 
Gases genannt werden, diese konnte nicht konstant eingestellt werden, sondern 
musste am Heizgenerator durch einen Drehregler eingestellt werden.