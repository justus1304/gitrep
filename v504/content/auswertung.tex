\section{Auswertung}
\label{sec:Auswertung}

\subsection{Kennliene der Hochvakuumdiode und Sättigungsströme}
Die verschiedenen Werte für die Heizströme $I_H$ und Saugspannungen $U_S$ 
befinden sich in den Tabellen \autoref{tab:1} und \autoref{tab:2}. Tabelle 
1 stellt die Werte für die erste Messung bei einem Heizstrom von $I_H = 1,8 \unit{\ampere}$
dar. Die restlichen Messungen mit mehr Messdaten befinden sich in Tabelle 2.
\begin{table}[H]
    \centering
    \caption{Messdaten für die Kennlinien.}
    \label{tab:1}
    \begin{tblr}{
        colspec = {Q[l,si={table-format=3.0}] Q[l,si={table-format=1.3}]},
        row{1} = {font=\bfseries},
    }
    \toprule
    \( U_S \) / \si{\volt} 
        & \( I_S \) / \si{\milli\ampere} \\
    \midrule
    0   & 0.000 \\
    5   & 0.001 \\
    10  & 0.003 \\
    15  & 0.004 \\
    20  & 0.004 \\
    25  & 0.005 \\
    30  & 0.005 \\
    35  & 0.005 \\
    40  & 0.005 \\
    45  & 0.005 \\
    50  & 0.005 \\
    55  & 0.005 \\
    60  & 0.006 \\
    65  & 0.006 \\
    70  & 0.006 \\
    80  & 0.006 \\
    90  & 0.006 \\
    100 & 0.007 \\
    \bottomrule
    \end{tblr}
\end{table}

\begin{table}[H]
    \centering
    \caption{Messdaten für die Kennlinien.}
    \label{tab:2}
    \begin{tblr}{
        colspec = {c *{4}{c}},  % 1 Spalte für U_S + 4 Spalten für I_S
        row{1} = {font=\bfseries},
        hline{1,2,Z} = {1pt},  % Linien oben, unter Überschrift und unten
    }
    \toprule
        & \( I_H = \SI{2.0}{\ampere} \) 
        & \( I_H = \SI{2.1}{\ampere} \) 
        & \( I_H = \SI{2.3}{\ampere} \) 
        & \( I_H = \SI{2.5}{\ampere} \) \\
    \midrule
    \( U_S \) / \si{\volt} 
        & \( I_S \) / \si{\milli\ampere} 
        & \( I_S \) / \si{\milli\ampere} 
        & \( I_S \) / \si{\milli\ampere} 
        & \( I_S \) / \si{\milli\ampere} \\
    \midrule
    0   & 0.000 & 0.000 & 0.000 & 0.000 \\
    5   & 0.010 & 0.012 & 0.020 & 0.023 \\
    10  & 0.023 & 0.030 & 0.051 & 0.062 \\
    15  & 0.035 & 0.051 & 0.090 & 0.109 \\
    20  & 0.042 & 0.069 & 0.133 & 0.160 \\
    25  & 0.046 & 0.079 & 0.179 & 0.217 \\
    30  & 0.049 & 0.086 & 0.216 & 0.276 \\
    35  & 0.050 & 0.091 & 0.255 & 0.329 \\
    40  & 0.052 & 0.094 & 0.292 & 0.321 \\
    45  & 0.053 & 0.097 & 0.292 & 0.358 \\
    50  & 0.054 & 0.099 & 0.309 & 0.397 \\
    55  & 0.055 & 0.100 & 0.341 & 0.444 \\
    60  & 0.056 & 0.102 & 0.365 & 0.484 \\
    70  & 0.056 & 0.103 & 0.406 & 0.594 \\
    80  & 0.056 & 0.100 & 0.438 & 0.686 \\
    90  & 0.059 & 0.105 & 0.466 & 0.791 \\
    100 & 0.060 & 0.108 & 0.488 & 0.890 \\
    110 & 0.061 & 0.111 & 0.513 & 1.010 \\
    120 & 0.062 & 0.113 & 0.540 & 1.121 \\
    130 & 0.063 & 0.114 & 0.549 & 1.228 \\
    140 & 0.064 & 0.116 & 0.554 & 1.336 \\
    150 & 0.064 & 0.117 & 0.559 & 1.436 \\
    160 & 0.065 & 0.118 & 0.563 & 1.516 \\
    170 & 0.066 & 0.119 & 0.566 & 1.592 \\
    180 & 0.066 & 0.119 & 0.569 & 1.666 \\
    190 & 0.066 & 0.119 & 0.573 & 1.746 \\
    200 & 0.066 & 0.120 & 0.576 & 1.822 \\
    210 & 0.066 & 0.120 & 0.579 & 1.903 \\
    220 & 0.066 & 0.120 & 0.583 & 1.969 \\
    230 & 0.066 & 0.120 & 0.586 & 2.040 \\
    240 & 0.066 & 0.120 & 0.590 & 2.090 \\
    250 & 0.066 & 0.120 & 0.593 & 2.140 \\
    \bottomrule
    \end{tblr}
\end{table}
\noindent In \autoref{fig:REFDUHS} ist die Saugspannung gegen den detektierten Strom 
Strom aufgetragen. Dabei ist das sigmoidale Wachstum einigermaßen erkennbar, 
vor Allem jedoch bei den höheren Heizspannungen fällt es einfacher, den Verlauf 
zu identifizieren.
\noindent Mithilfe dieser Daten lassen sich die unterschiedlichen Sättigungsströme 
ablesen, diese lauten wiefolgt:
\begin{figure}[H]
    \centering
    \includegraphics[width=0.95\textwidth]{mw_plot.pdf}
    \caption{Kennlinien bei unterschiedlichen Heizströmen.}
    \label{fig:REFDUHS}  % <- Label NACH der Caption!
\end{figure}
\begin{table}[H]
    \centering
    \caption{Ermittelte Sättigungsströme.}
    \label{tab:t1}
    \begin{tblr}{
        colspec = {S S},
        row{1} = {guard, mode=math},
      }
    \toprule
    I / \si{\volt} & I_H / \si{\milli\ampere}\\
    \midrule
    1.8 & 0.007 \\
    2.0 & 0.066 \\
    2.1 & 0.120 \\
    2.3 & 0.593 \\
    2.5 & 2.140 \\
    \bottomrule 
    \end{tblr}
\end{table}

\subsection{Langmuir-Schottky'sche Raumladungsgesetz}
In dieser Sektion soll der Gültigkeitsbereich des Langmuir-Schottkyschen 
Raumladungsgesetzes bestimmt werden. Dafür essentiell ist die Strom-Spannungs-
Beziehung mit der Form
\begin{equation}
    j = \frac{4}{9} \varepsilon_0 \sqrt{2 \frac{e_0}{m_0}} \frac{U^b}{a^2}
\end{equation}
Die nicht-lineare Regeression dazu wird in Python mithilfe eines Curve fits 
erstellt und verfügt über die Form: $a \cdot U^b$. Um mit der Theorie
übereinzustimmen, sollte $b$ dem Wert $\frac{3}{2}$ möglichst nahe kommen.
Nach Einsetzen der Parameter und Erstellung des Fits ergeben sich die Parameter 
zu:
\begin{align*}
    a &= 0.001 \pm 0.007 \\
    b &= 1.4 \pm 1.8 \\
\end{align*}
Die Kurve ist in \autoref{fig:2} dargestellt. Aufgrund technischer Schwierigkeiten
konnten keine verwertbaren eignen Messwerte für diesen Teil verwendet werden. 
Daher wurden die Messdaten des Tutors zur Verfügung gestellt, um eine vernünftige 
Auswertung zu ermöglichen.
\begin{figure}[H]
    \centering
    \includegraphics[width=0.8\textwidth]{langmuir.pdf}
    \caption{Langmuir-Schottky.}
    \label{fig:2}  % <- Label NACH der Caption!
\end{figure}
\noindent Der Wendepunkt der Messwerte ist bei etwa $100 \unit{\volt}$ abzulesen.
Demzufolge findet die Raumladungsgleichung im Interval [0,100] Anwendung.

\subsection{Anlaufstromgebiet}
In \autoref{tab:3} ist einerseits die Spannung des Gleichstromspannungsgeräts
und andererseits die detektierte Spannung am Amperemeter dargelegt. Die dazugehörige 
Heizstromstärke und Heizspannung betragen $I_H = 5 \unit{\volt}$ und $U_H = 2.5 \unit{\ampere}$.
\begin{table}[H]
    \centering
    \caption{Anlaufstromgebiet-Messwerte.}
    \label{tab:3}
    \begin{tblr}{
        colspec = {S S},
        row{1} = {guard, mode=math},
      }
    \toprule
    U / \unit{\volt} & I / \unit{\ampere}\\
    \midrule
    84   & 0.95 \\
    105  & 0.90 \\
    135  & 0.85 \\
    183  & 0.80 \\
    243  & 0.75 \\
    250  & 0.70 \\
    340  & 0.65 \\
    450  & 0.60 \\
    600  & 0.55 \\
    770  & 0.50 \\
    980  & 0.45 \\
    1080 & 0.40 \\
    1380 & 0.35 \\
    1830 & 0.30 \\
    2370 & 0.25 \\
    2970 & 0.20 \\
    3100 & 0.15 \\
    4000 & 0.10 \\
    5000 & 0.05 \\
    6200 & 0.00 \\
    \bottomrule 
    \end{tblr}
\end{table}
Mithilfe dieser Werte kann eine Ausgleichsrechnung der Form
\begin{equation}
    I = a \cdot e^{\left(-\frac{e_0 U}{k_B b}\right)}
\end{equation}
erstellt werden. Mit den eingesetzten Werten ergibt sich folgender Fit:
\begin{figure}[H]
    \centering
    \includegraphics[width=0.8\textwidth]{f.pdf}
    \caption{Anlaustromgebiet-Regression.}
    \label{fig:3}  % <- Label NACH der Caption!
\end{figure}
\noindent Die Fitparameter ergeben sich zu
\begin{align*}
    a &= (6.183 \pm 0.8) \cdot 10^{-6} \unit{\ampere} \\
    b &= (2782.9 \pm 0.5) \unit{\kelvin} \\
\end{align*}
Die gesuchte Temperatur entspricht dem Parameter $b = (2782.9 \pm 0.5) \unit{\kelvin}$.

\subsection{Kathodentemperatur}
Wie in \autoref{sec:Theorie} hergleitet, gilt für die Kathodentemperatur der 
Zusammenhang:
\begin{equation*}
    T= \left(\frac{I_H U_H - N_{WL}}{f \eta \sigma}\right)^{1/4}
\end{equation*}
$I_H$ und $U_H$ sind die Heizspannungen (deren Werte können \autoref{tab:4} 
entnommen werden), $N_{WL}$ ist die Wärmeleistung mit 
einem Wert von $N_{WL} = 0.95 \unit{\watt}$. $f$ ist die Kathodenoberfläche, 
welche durch $3.5 \cdot 10^{-5} \unit{\meter}^2$ gegeben ist. Der Wert $\eta$ 
ist der Emissionsgrad der Oberfläche und beträgt $\eta = 0.28$. Das $\sigma$ 
steht für die Stefan-Boltzmannsche Strahlungskonstante, diese beläuft sich auf:
$\sigma = 5.7 \cdot 10^{-12} \frac{\unit{\watt}}{\unit{\centi\meter}^2 \unit{\kelvin}^4}$.
Mit allen eingesetzten Werten ergeben sich die folgenden Temperaturen:
\begin{table}[H]
    \centering
    \caption{Ermittelte Sättigungsströme.}
    \label{tab:t4}
    \begin{tblr}{
        colspec = {S S S},
        row{1} = {guard, mode=math},
      }
    \toprule
    I_H / \unit{\ampere} & U_H / \unit{\volt} & T / \unit{\kelvin}\\
    \midrule
    1.8 & 3 & 1680.021 \\
    2.0 & 4 & 1884.829 \\
    2.1 & 4 & 1911.014 \\
    2.3 & 5 & 2084.673 \\
    2.5 & 5 & 2132.408 \\
    \bottomrule 
    \end{tblr}
\end{table}

\subsection{Berechnung der Austrittsarbeit}
Die Austrittsarbeit des Kathodenmaterials wird mithilfe von \autoref{eqn:12}
errechnet. Da diese $W_A = e_0 \Phi$ ist, ergibt sich diese mit Abkürzung 
der spezifischen Stromdichte als $j_S = \frac{I_S}{f}$ als:
\begin{equation*}
    W_A = -T k_B \ln\left(\frac{j_S h^3}{4 \pi e_0 m_0 k_B^2 T^2}\right)
\end{equation*}
Die Austrittsarbeit aus dem Wolfram zu allen Kennlinien befindet sich in
Tabelle \autoref{tab:x} %REF BEARBEITEN
\begin{table}[H]
    \centering
    \caption{Ermittelte Sättigungsströme.}
    \label{tab:x}
    \begin{tblr}{
        colspec = {S S},
        row{1} = {guard, mode=math},
      }
    \toprule
    \text{Kennlinie} & W_A / \unit{\electronvolt}\\
    \midrule
    1 & 0.007 \\
    2 & 0.066 \\
    3 & 0.120 \\
    4 & 0.593 \\
    5 & 2.140 \\
    \bottomrule 
    \end{tblr}
\end{table}