\section{Auswertung}
\label{sec:Auswertung}

\subsection{Abmessung des Acrylblocks} 
Es werden die Abmessungen eines Acrylblocks mit der Schieblehre und durch
einen Amplituden-Scan bestimmt. Die ermittelten Abmessungen mit der Schieblehre
sind \autoref{tab:10} zu entnehmen. Bei der Methode durch den A-Scan wird der 
Acrylblock von oben und unten an der Stelle von jedem Loch gescannt und 
Die Laufzeit des Ultraschalls gemessen. Aus dieser Laufzeit lässt sich durch
\autoref{eqn:2} die Strecke $s$ bis zur Fehlstelle bestimmen. Die Fehlstellen
sind hier die Löcher im Acrylblock. Die Laufzeiten von beiden Seiten sowie die
Umrechnung in die Abstände sind \autoref{tab:11} zu entnehmen. 
\begin{table}[H]
  \centering
  \caption{Abmessungen Schiebleere}
  \label{tab:10}
  \begin{tblr}{
          colspec = {S S S},
          row{1} = {guard, mode = math},
      }
      \toprule
      Loch & d \, \unit{\mm}& y-Position \, \unit{\mm}\\
      \midrule
      1  &   1.55 +- 0.05&   59.05 +- 0.05\\
      2  &   1.55 +- 0.05&   60.8  +- 0.05\\
      3  &   6.05 +- 0.05&   13.15 +- 0.05 \\
      4  &   5    +- 0.05&   21.8  +- 0.05\\
      5  &   4.95 +- 0.05&   30.3  +- 0.05\\
      6  &   3.05 +- 0.05&   38.7  +- 0.05\\
      7  &   3.1  +- 0.05&   46.7  +- 0.05\\
      8  &   3.05 +- 0.05&   54.75 +- 0.05 \\
      9  &   3.1  +- 0.05&   62.7  +- 0.05\\
      10  &  3.1  +- 0.05&   70.7  +- 0.05\\
      11  &  10.0 +- 0.05 &   15.3 +- 0.05 \\
      \bottomrule 
  \end{tblr}
\end{table}
\noindent Um aus der Laufzeit auf den Abstand schließen zu können, muss die
Schallgeschwindigkeit im Medium bekannt sein. Diese wird durch eine lineare 
Regression bestimmt: Dabei wird der mit einer Schieblehre gemessene y-Abstand
der Trennlinie jedes Lochs gegen die zugehörige Laufzeit des Schalls (Hin- und
Rückweg zur Sonde) aufgetragen. Aus der Steigung der Ausgleichsgeraden ergibt
sich die Schallgeschwindigkeit als zurückgelegte Strecke pro Zeiteinheit. Die
Ausgleichsrechnung ist in \autoref{fig:10} dargestellt.
\begin{figure}[H]
  \centering 
  \caption{Lineare Regression zur Bestimmung der Schallgeschwindigkeit im Acrylblock.}
  \label{fig:10}
  \includegraphics{build/plot1.pdf}
\end{figure}
\noindent Die Parameter der Ausgleichsrechnung lauten 
\begin{align}
  m &= \qty{1374(14)}{\meter\per\second}\\
  b &= \qty{-0.0012(0.0004)}{\meter}.
\end{align}
Daraus ergibt sich eine Schallgeschwindigkeit von $v_\text{s,Acryl} = m
\cdot 2 = \qty{2748(28)}{\meter\per\second}$, welche in die Berechnung der
Abmessungen bereits einfließt.

\begin{table}[H]
  \centering
  \caption{Laufzeiten im Acryl und daraus ermittelte Abmessungen.}
  \label{tab:11}
  \begin{tblr}{
          colspec = {S S S S S S},
          row{1} = {guard, mode = math},
      }
      \toprule
      Loch & t_\text{Oben} \, / \unit{\micro\second} 
           & s_\text{Oben} \, / \unit{\mm} 
           & t_\text{Unten} \, / \unit{\micro\second}
           & s_\text{Unten} \, / \unit{\mm}
           & d \, \unit{\mm}\\
      \midrule
      1   & 13.8  & 18.96+-0.19 & 44.0 & 60.5+-0.6   & 1.924+-0.020 \\
      2   & 15.3  & 21.02+-0.21 & 45.1 & 62.0+-0.6   & -1.649+-0.017\\
      3   & 45.1  & 62.0+-0.6   & 10.4 & 14.29+-0.15 & 5.08+-0.05   \\
      4   & 39.7  & 54.5+-0.6   & 16.5 & 22.67+-0.23 & 4.12+-0.04   \\
      5   & 34.0  & 46.7+-0.5   & 23.6 & 32.43+-0.33 & 2.198+-0.022 \\
      6   & 28.6  & 39.3+-0.4   & 29.0 & 39.8+-0.4   & 2.198+-0.022 \\
      7   & 22.6  & 31.05+-0.32 & 34.9 & 48.0+-0.5   & 2.336+-0.024 \\
      8   & 16.8  & 23.08+-0.24 & 40.0 & 55.0+-0.6   & 3.298+-0.034 \\
      9   & 10.9  & 14.98+-0.15 & 46.8 & 64.3+-0.7   & 2.061+-0.021 \\
      10  & 5.1   & 7.01+-0.07  & 0    &  0.0+-0     & \\
      11  & 0     & 0.0+-0      & 11.9 & 16.35+-0.17 &      \\
      \bottomrule
  \end{tblr}
\end{table}

\subsection{Das Auge}
Als Nächstes werden die Abmessungen in einem Auge bestimmt. Dazu wird die 
Schallgeschwindigkeit im Medium mit der für den Weg zur jewailigen Trennstelle und zurück benötigte 
Zeit multipliziert. So erhällt man die Wegstrecke von hin und Rückweg des Ultraschalls zur Trennstelle und zurück. 
Um den Abstand zur jewailigen trennstelle zu erhalten wird diese Weglänge noch einmal halbiert.Als Schallgeschwindigkeit wird wieder $v_s$ angenommen.
\begin{table}[H]
  \centering
  \caption{Abmessungen Schieblehre}
  \label{tab:10}
  \begin{tblr}{
          colspec = {S S S},
          row{1} = {guard, mode = math},
      }
      \toprule
      \text{Part} & t \, / \unit{\micro\second} & \text{Abstand} \, x \, / \unit{\mm}\\
      \midrule
      \text{Iris}          & 11.0 & 15.11+-0.15 \\
      \text{Linseneingang} & 17.0 & 23.36+-0.24 \\
      \text{Linsenausgang} & 24.9 & 34.21+-0.35 \\
      \text{Retina}        & 70.0 & 96.2+-1.0   \\
      \bottomrule 
  \end{tblr}
\end{table}

\subsection{Die Tumore}
Die beiden Tumore im Brustmodell sind in \autoref{fig:11} $\left(T_1\right)$ 
und \autoref{fig:12} $\left(T_2\right)$ abgebidet. Anhand der Bilder wird die
Lage der Tumore bestimmt.
\begin{figure}[H]
  \centering
  \caption{Tumor $T_1$}
  \label{fig:11}
  \includegraphics[width=0.8\textwidth]{Bilder/BrustA1.png}
\end{figure}
\noindent Der Tumor in \autoref{fig:11} befindet sich \qty{8.7(1.7)}{\mm} tief
in der Brust und verfügt über einen Durchmesser von \qty{37.5(1.7)}{\mm}.
\begin{figure}[H]
  \centering
  \caption{Tumor $T_2$}
  \label{fig:12}
  \includegraphics[width=0.8\textwidth]{Bilder/BrustB1.png}
\end{figure}
\noindent Tumor $T_2$ in \autoref{fig:11} befindet sich \qty{8.7(1.7)}{\mm}
tief in der Brust und beläuft sich auf einen Durchmesser von \qty{22.3(1.7)}{\mm}.

%\subsection{Herzmodell}
%\begin{figure}
%  \centering
%  \caption{Tumor $T_1$}
%  \label{fig:13}
%  \includegraphics[width=0.5\textwidth]{Bilder/Herz.png}
%\end{figure}
\subsection{Das Herzmodell}

\begin{figure}[H]
  \centering
  \caption{Herzschläge}
  \label{fig:14}
  \includegraphics[width=0.8\textwidth]{Bilder/Herz2.png}
\end{figure}

\begin{table}[H]
  \centering
  \caption{Abmessungen Schieblehre.}
  \label{tab:13}
  \begin{tblr}{
          colspec = {S S S},
          row{1} = {guard, mode = math},
      }
      \toprule
      \text{Part} & t \, / \unit{\micro\second}& \text{Abstand} \, x / \unit{\micro\second}\\
      \midrule
      1 & 1 & 93.5  \\
      2 & 2 & 93.5  \\
      3 & 2 & 93.5   \\
      4 & 1 & 93.5  \\
      5 & 1 & 93.5 \\
      6 & 1 & 93.5  \\
      7 & 1 & 93.5   \\
      8 & 1 & 93.5  \\
      \bottomrule 
  \end{tblr}
\end{table}
\noindent In \autoref{fig:14} sind die Herzschläge aus dem TM-Scan abgebildet. 
Daraus wird graphisch die Breite und Höhe der Schläge abgelesen, in \autoref{tab:13}
eingetragen und der Mittelwert gebildet. In 18 Sekunden schlug das Herz 
neun mal, was einer Herzfrequenz von $v_{Herz} = 0.5$ Schlägen pro Seknde entspricht.
Der endystolische Durchmesser kann durch folgende Formel bestimmt werden 
\begin{equation}
  ESD = \frac{c_\text{wasser} \cdot A}{2}
\end{equation}
A ist dabei die gemittelte Amplitude $\overline{A} = \qty{93.5(0.5)}{\micro\second}$. Mit
Werten ergibt sich $ ESD = \qty{0.0694(0.0004)}{\meter}$. Daraus ergibt sich
das endystolische Herzvolumen zu 
\begin{equation}
  ESV = \frac{4\cdot \pi}{3}\cdot\left(\frac{ESD}{2}\right)^3 = \qty{0.0001748(0.0000028)}{\meter^3}
\end{equation}
Das Herzschlagvolumen ist damit zu
\begin{equation}
  V_\text{herz} = ESV \cdot v_\text{herz} = \qty{8.74(0.14)e-05}{\meter^3\per\second}
\end{equation}
bestimmt.
