\section{Auswertung}
\label{sec:Auswertung}


\subsection{Winkelrichtgröße D}
Im Folgenden wird zunächst die Winkelrichtgröße $D$ und das 
Eigenträgheitsmoment der Drehachse bestimmt. Dazu wird die
rücktreibende Kraft $F$ der Spiralfeder für verschiedene 
Auslenkungen $\varphi$ gemessen. \\
\begin{table}[H]
  \centering
  \caption{Messwerte zur Bestimmung von $D$.}
  \label{tab:tabelle}
  %\sisetup{table-format=1.1, per-mode=reciprocal}
  \begin{tblr}{
      colspec = {S S },
      row{1} = {guard, mode=math},
      %vline{4} = {2}{-}{text=\clap{$\pm$}},
    }
    \toprule
    \varphi(°) &  F(N)\\
    \midrule
    22.5  & 0\\
    45.0  & 0.04\\
    67.5  & 0.068\\
    90.0  & 0.09\\
    112.5 & 0.126\\
    135.0 & 0.164\\
    157.5 & 0.172\\
    180.0 & 0.18\\
    202.5 & 0.2\\
    225.0 & 0.28\\
    \midrule
    $\overline{\varphi}$ & $\overline{F}$\\
    123.75 & 0.132\\
    \midrule
    \bottomrule
  \end{tblr}
\end{table}
Mit diesen Mittelwerten kann mithilfe der Formel
\eqref{eqn:gl9} und $r= 29,2cm$ die Winkelrichtgröße der Spiralfeder zu %ich weiß nicht, was ich falsch mache, er will einfach nicht die gleichung im pdf referenzieren.
0,0031 $\frac{N \cdot m}{rad}$ bestimmt werden. 

\subsection{Eigenträgheitsmoment $I_D$}
Als nächstes wird das Eigenträgheitsmoment $I_D$ der Drillachse bestimmt.
Um das zu tun, wird das Trägheitsmoment des gesamten System berechnet.
Dafür wird das Trägheitsmoment des Stabs um die Drehachse ermittelt, ebenso
wie das Trägheitsmoment der beiden Hohlzylinder, welche jeweils im Abstand
a zu der Drehachse stehen. Im Anschluss wird das Trägheitsmoment des Stabes
und der Hohlzylinder von dem Gesamtergebnis abgezogen, sodass sich als 
Ergebnis das gesuchte Trägheitsmoment der Drillachse ergibt.

das theoretische Trägheitsmoment 

 \begin{table}[H]
   \centering
   \caption{Messwerte T/a.}
   \label{tab:tabelle}
   %\sisetup{table-format=1.1, per-mode=reciprocal}
   \begin{tblr}{
       colspec = {S S },
       row{1} = {guard, mode=math},
       %vline{4} = {2}{-}{text=\clap{$\pm$}},
     }
     \toprule
     a(cm) & T(s)\\
     \midrule
     10    & 3.28\\
     11.5  & 3.71\\
     13    & 4.28\\
     14.5  & 4.75\\
     16    & 5.25\\
     17.5  & 5.90\\
     19    & 6.28\\
     20.5  & 6.78\\
     22    & 7.28\\
     23.5  & 7.97\\
     \midrule
     $\overline{T}$ & $\overline{a}$\\
     \midrule
     \bottomrule
   \end{tblr}
 \end{table}

\begin{figure}[H]
  \centering
  \includegraphics{plot.pdf}
  \caption{Plot.}
  \label{fig:plot}
\end{figure}

\subsection{Trägheitsmoment Kugel und Zylinder}
%Kugel
\begin{table}[H]
  \centering
  \caption{Abmessungen Holzkugel.}
  \label{tab:kugel}
  %\sisetup{table-format=1.1, per-mode=reciprocal}
  \begin{tblr}{
      colspec = {S S },
      row{1} = {guard, mode=math},
      %vline{4} = {2}{-}{text=\clap{$\pm$}},
    }
    \toprule
    r(m) & m(kg)\\
    \midrule
    0.0605\pm 0.0001  & 0.638\pm 0.001 \\
    \bottomrule
  \end{tblr}
\end{table}
%zylinder
\begin{table}[H]
  \centering
  \caption{Abmessungen Holzzylinder.}
  \label{tab:zylinder}
  %\sisetup{table-format=1.1, per-mode=reciprocal}
  \begin{tblr}{
      colspec = {S S S},
      row{1} = {guard, mode=math},
      %vline{4} = {2}{-}{text=\clap{$\pm$}},
    }
    \toprule
    r(m) & h(m) & m(kg)\\
    \midrule
    0.0493\pm 0.0001 & 0.0995\pm 0.0001 & 0.366\pm 0.001 \\
    \bottomrule
  \end{tblr}
\end{table}

Hier sollte das Trägheitsmoment von einer kugel und eines Zylinders berechnet werden. mit den maßen aus \autoref{tab:kugel}
und \autoref{eqn:gl5} berechnet sich $I_{Holzkugel} = (0.0009350\pm0.0000034)kg ⋅ m²$. und Das Trägheitsmoment des Zylinder
mit den abmessungen \autoref{tab:zylinder} eingesetzt in \autoref{eqn:gl6} zu $I_Holzzylinder = 
0.000444\pm0.0000022 kg ⋅ m²$.





