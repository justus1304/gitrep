\section{Diskussion}
\label{sec:Diskussion}

\subsection{Verifizierung der Funktionsweise}
Die Parameter im ersten Teil des Versuches sind für das tatsächliche
Experiment irrelevant. Wichtiger ist der gut erkennbare sinusförmige Verlauf
der Spannungsamplitude, der das Eingangssignal wiedergibt. Die Parameter wären
für einen Vergleich interessant, sofern die Messungen bei eingeschaltetem
Rauschen wiederholt würden. Jenes war der Ursprungsgedanke des Versuchs, 
allerdings wurde dieser Teil gestrichen, da die Geräte zu fehleranfällig sind, 
um ein Rauschen in einem vernünftigen Maße hinzuzufügen.

\subsection{Verifizierung durch Leuchtdiode}
Die Intensität des Lichtes sollte tatsächlich eine $\frac{1}{x^2}$ 
Abhängigkeit haben. Da diese nicht näherungsweise im Fehlerbereich des
Parameters b liegt, kann auf Ungenauigkeiten im Experiment geschlossen 
werden. Der berechnete Wert liegt bei
\begin{align*}
    b =& \qty{0.80(0.32)}{}.     
\end{align*}
\noindent Der maximale Abstand, bei dem die Intensität minimal (aber noch 
groß genug für einen Nachweis) ist, liegt bei 
\begin{align*}
    x_{max} =& \, 42 \, \unit{\centi\meter}.
\end{align*}