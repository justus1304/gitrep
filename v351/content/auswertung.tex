\section{Auswertung}
\label{sec:Auswertung}

\subsection{Fourier-Synthese}


\subsection{Fourier-Analyse}
Bei der Fourier-Analyse werden am Oberwellengeneratur die Rechtecksfunktion,
die Dreiecksfunktion und die Sägezahnfunktion generiert. Am Amperemeter werden 
die Amplituden bei den jewailigen Frequenzen der Oberwelle abgelesen, um so
auf die Abhängigkeit der Spannung der $n$-ten Oberwelle von $n$ schließen zu
können.
\subsubsection{Rechteckspannung}
\begin{table}[H]
    \centering
    \caption{Amplituden der Oberschwingungen Rechtecksfunktion.}
    \label{tab:j1}
    \begin{tblr}{
        colspec = {S S S},
        row{1} = {guard, mode=math},
      }
    \toprule
    f (\unit{\hertz}) &  A (\unit{\deci\bel})\\
    \midrule
    10 & 17.8\\
    30  & 9.01\\
    50  & 4.61\\
    70  & 1.81\\
    90  &-0.19\\
    110 &-1.79\\
    130 &-3.39\\
    150 &-4.59\\
    170 &-6.19\\
    190 &-8.19\\
    \bottomrule
    \end{tblr}
\end{table}
Um auf die Abhängigkeit von n schließen zu können, fitten wir die 
in Volt umgerechneten Spannungswerte die gegen die ordnung der jewailigen 
Oberwelle aufgetragen sind an einefunktion der form 
\begin{equation}
    \label{eqn:1}
    U(n) = \frac{a}{n^b}
\end{equation}


\begin{figure}
    \centering
    \caption{Curve Fit und Messwerte Rechteckspannung}
    \includegraphics{viereck.pdf}
\end{figure}

aus dem Curve fit mithilfe der Python Bibiliothek "Pyplot" erhalten wir 
für die Rechtecksspannung die Paarameter 
\begin{align*}
    a = & \num{7.7 +- 1.0}\\
    b = & \num{1.34 +- 0.26}\\
\end{align*}
Der Ausgleichsparameter b gibt und somit den Grad der abhängigkeit von n an.



\subsubsection{Dreieckspannung}
\begin{table}[H]
    \centering
    \caption{Amplituden der Oberschwingungen Dreiecksfunktion.}
    \label{tab:j1}
    \begin{tblr}{
        colspec = {S S S},
        row{1} = {guard, mode=math},
      }
    \toprule
    f (\unit{\hertz}) &  A (\unit{\deci\bel})\\
    \midrule
    10  & 14.2  \\
    30  & -4.59 \\
    50  & -13.4 \\
    70  & -19.4 \\
    90  & -23.4 \\
    110 & -27.0 \\
    130 & -29.4 \\
    \bottomrule
    \end{tblr}
\end{table}
Auch bei der Dreiecksspannung wird ein Curf fit durch eine Funktion der Form
aus \autoref{eqn:1} auf die Messdaten angewendet. 

\subsubsection{Dreiecksfunktion}
\begin{figure}[H]
    \centering
    \caption{Curve Fit und Messwerte Dreiecksspannung}
    \includegraphics{dreieck.pdf}
\end{figure}

Die Parameter a und b ergeben sich zu 
\begin{align*}
    a = & \num{5.1 +- 1.0}\\
    b = & \num{3.0 +- 2.1}\\
\end{align*}
Wobei b wieder den grad der n abhängigkeit angibt.


\subsubsection{Sägezahnspannung}
\begin{table}[H]
    \centering
    \caption{Amplituden der Oberschwingungen Sägezahnfunktion.}
    \label{tab:j1}
    \begin{tblr}{
        colspec = {S S S},
        row{1} = {guard, mode=math},
      }
    \toprule
    f (\unit{\hertz}) &  A (\unit{\deci\bel})\\
    \midrule
    10  & 11.8  \\
    20  &  7.01 \\
    30  &  3.01 \\
    40  &  1.01 \\
    50  & -1.39 \\
    60  & -2.99 \\
    70  & -4.00 \\
    80  & -5.79 \\
    90  & -6.19 \\
    100 & -8.99 \\
    110 & -7.79 \\
    120 & -9.79 \\
    130 & -9.39 \\
    \bottomrule
    \end{tblr}
\end{table}


\subsubsection{Sägezahnfunktion}[H]
\begin{figure}
    \centering
    \caption{Curve Fit und Messwerte Sägezaahnspannung}
    \includegraphics{saege.pdf}
\end{figure}
Die Ausgleichsparameter für die Sägezahnspannung lauten
\begin{align*}
    a = & \num{4.0 +- 0.9}\\
    b = & \num{0.94 +- 0.28}\\
\end{align*}


%Auskommentiert, da es irgendwie Probleme mit "\includegraphics{rechteck.pdf}"
%gab (bereits bei make nachdem ich direkt gepullt habe)



