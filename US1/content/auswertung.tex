\section{Auswertung}
\label{sec:Auswertung}

\subsection{Messwerte}
\begin{table}[H]
    \centering
    \caption{Messwerte der vier Peaks zur Programmeinstellung.}
    \label{tab:10}
    \begin{tblr}{
            colspec = {S S S S S S},
            row{1} = {guard, mode = math},
        }
        \toprule
        \text{Peak}& \text{Laufzeit}\unit{\micro\second}\\
        \midrule
        1  & 3.2\\
        2  & 30.0\\
        3  & 59.9\\
        4  & 89.4\\
        \bottomrule 
    \end{tblr}
\end{table}

\begin{table}[H]
    \centering
    \caption{Messwerte der Periodendauer.}
    \label{tab:11}
    \begin{tblr}{
            colspec = {S S S S S S},
            row{1} = {guard, mode = math},
        }
        \toprule
        \text{Periodendauer}/\unit{\micro\second}\\
        \midrule
        0.5\\
        0.6\\
        0.7\\
        0.5\\
        0.7\\
        0.7\\
        \bottomrule 
    \end{tblr}
\end{table}

\begin{table}[H]
    \centering
    \caption{Höhen der jewailigen zylinder sowie Laufzeiten zu den Grenzflächen.}
    \label{tab:12}
    \begin{tblr}{
            colspec = {S S S S S S},
            row{1} = {guard, mode = math},
        }
        \toprule
        \text{Zylinder}&\text{Laufzeit Acryl}/\unit{\micro\second} & \text{Laufzeit Alu}/\unit{\micro\second} & \text{Höhe Acryl}/\unit{\centi\meter}& \text{Höhe Alu}/\unit{\centi\meter}\\
        \midrule
        1 &  85.7  &  37.3  &  12.07  & 12.35\\
        2 &  73.2  &  33.9  &  10.22  & 11.51\\
        3 &  56.7  &  25    &  8.06   & 8.12\\
        4 &  43.2  &  19.1  &  6.17   & 6.22\\
        5 &  36.7  &  16.3  &  5.30   & 5.32\\
        6 &  27.5  &  12.2  &  4.03   & 4.01\\
        7 &  20.7  &  9.4   &  3.12   & 3.12\\
        8 &  65.9  &  16.5  &  9.33   & 9.33\\
        \bottomrule 
    \end{tblr}
\end{table}

\begin{table}[H]
    \centering
    \caption{Messwerte der Amplituden mit zwei verschiedenen Sonden.}
    \label{tab:13}
    \begin{tblr}{
            colspec = {S S S S S S},
            row{1} = {guard, mode = math},
        }
        \toprule
        \text{Zylinder}&A_{einf,2\unit{\mega\hertz}}\text{Acryl}/\unit{\micro\second} & A_{ref,2\unit{\mega\hertz}}\text{Acryl}/\unit{\micro\second} &A_{einf,1\unit{\mega\hertz}}\text{Acryl}/\unit{\micro\second} &A_{ref,1\unit{\mega\hertz}}\text{Acryl}/\unit{\micro\second}\\
        \midrule
        1 &  1.244 &  0.647 &  0.727  & 0.536\\
        2 &  1.244 &  0.62  &  0.727  & 0.609\\   
        3 &  1.244 &  0.385 &  0.727  & 0.617\\
        4 &  1.244 &  0.62  &  0.727  & 0.470\\
        5 &  1.244 &  0.604 &  0.727  & 0.470\\
        6 &  1.244 &  0.659 &  0.727  & 0.505\\
        7 &  1.244 &  0.724 &  0.727  & 0.393\\
        9 &  1.244 &  0.602 &  0.727  & 0.386\\
        \bottomrule 
    \end{tblr}
\end{table}


\begin{table}[H]
    \centering
    \caption{Laufzeiten zur Wasseroberfläche für verschiedene Füllmengen.}
    \label{tab:14}
    \begin{tblr}{
            colspec = {S S S S S S},
            row{1} = {guard, mode = math},
        }
        \toprule
        Wassermenge / \unit{mili\liter}&Laufzeit / \unit{\micro\second}\\
        \midrule
        1 &  1.244 &  0.647 &  0.727  & 0.536\\
        2 &  1.244 &  0.62  &  0.727  & 0.609\\   
        3 &  1.244 &  0.385 &  0.727  & 0.617\\
        4 &  1.244 &  0.62  &  0.727  & 0.470\\
        5 &  1.244 &  0.604 &  0.727  & 0.470\\
        6 &  1.244 &  0.659 &  0.727  & 0.505\\
        7 &  1.244 &  0.724 &  0.727  & 0.393\\
        9 &  1.244 &  0.602 &  0.727  & 0.386\\
        \bottomrule 
    \end{tblr}
\end{table}

\subsection{Programm und Geräteeinstellungen}
Um die funktionsweise des Programms zu verstehen wurden am anfang 
einmal vier x Werte von Peaks (\autoref{fig:10} sowie Werte in \autoref{tab:10}) bei der durchschallung eines Acrylblockes 
mit der $2 \unit{\mega\hertz}$ Sonde aufgenommen.Diese entstehen durch reflexion an der Grenzfläche 
bei eintritt in das Acryl und bei austritt und mit geschwächter Ampilitude auch weitere male 
an diesen grenzflächen. Um trotsdem die hinteren Peaks erkennbar zu machen wurde 
die Verstärkung, welche unten im Bild zu erkennen ist, eingeschaltet.
\begin{figure}[H]
    \centering
    \caption{Perks der Grenzflächenreflexion}
    \label{fig:10}
    \includegraphics[width=\textwidth]{bilder/ssprogramm.jpg}
\end{figure}
 Des weiteren wurden der Abbildung mehrere Periodendauern $T$ entnommen, indem 
 der Zeitabstand zwischen Anfang und Ende einer Ganzen Schwingugsperiode 
 gemessen wurde. Diese Werte aus \autoref{tab:11} wurden gemittelt und das ergibt eine 
 mittlere Periodendauer von 
 \begin{equation}
    T = \qty{0.61(0.04)}{\micro\second}
 \end{equation}

 \subsection{Schallgeschwindigkeit in Acryl und Aluminium}
 Die Schallgeschwindigkeiten in den verschiedenen Medien werden ermittelt indem 
 die Höhe der Zylinder gegen die halbe laufzeitdifferenz des Ultraschalls zwischen 
 Anfangs- und Endgrenzfläcke aufgetragen wird. Wenn man eine Ausgleichsgerade der Form 
 $y = m \cdot x + b$ an die Messwerte fitet, kann die Schallgeschwindigkeit 
 duch den Steigungsparameter $m$ bestimmt werden. Die Messwerte der Höhen, sowie die Laufzeiten 
 befinden sich für Aluminium und Messing in \autoref{tab:12}, sowie die graphischen Darstellungen für 
 Acryl in \autoref{fig:11} und für Aluminium in \autoref{tab:12}
 
 \begin{figure}[H]
    \centering
    \caption{Messwerte und regression zur ermittlung der Schallgeschwindigkeit in Acryl}
    \label{fig:11}
    \includegraphics{build/schallgeschwindigkeit_acryl.pdf}
\end{figure}
Die Parameter der Ausgleichsrechnung ergeben sich zu 
\begin{align*}
    a &= \qty{2744(14)}{\meter\per\second}\\
    b &= \qty{0.002(0.0004)}{meter}
\end{align*}
Somit ergibt sich eine Schallgeschwindigkeit in Acryl von 
\begin{equation}
    c_\text{Acryl} = \qty{2744(14)}{\meter\per\second}
\end{equation}

\begin{figure}[H]
    \centering
    \caption{Messwerte und regression zur ermittlung der Schallgeschwindigkeit in Aluminium}
    \label{fig:12}
    \includegraphics{build/schallgeschwindigkeit_alu.pdf}
\end{figure}
Die Parameter der Ausgleichsrechnung ergeben sich zu 
\begin{align*}
    a &= \qty{6213(1107)}{\meter\per\second}\\
    b &= \qty{0.0091(0.012)}{meter}
\end{align*}
Somit ergibt sich eine Schallgeschwindigkeit in Acryl von 
\begin{equation}
    c_\text{Aluminium} = \qty{6213(1107)}{\meter\per\second}
\end{equation}

\subsection{Dämpfungsfaktor}
Um den Dämpfungsfaktor der Schwingungsamplitude zu ermitteln, werden 
Amplituden von an Grenzflächen verschiedener Abstände reflektierten schwingungen 
des Ultraschalls betrachtet. Die Amplituden aus \autoref{tab:13} werden in \autoref{fig:13}(2$\unit{\mega\hertz}$ Sonde) 
sowie in \autoref{fig:14}(1$\unit{\mega\hertz}$ Sonde) gegen Die Abstände der Jewailigen grenzfläche (Höhen der Zylinder) aufgetragen 
und dann eine Regression der Form $y = m \cdot x + b$  durchgeführt. In \autoref{fig:131} sowie \autoref{141} sind die fits der Ursprünglichen Messwerte 
aufgetragen, jedoch fallen einige Messwerte so weit aus dem Muster, dass in \autoref{fig:13} die Messwerte 
1,2,3 und 7 ausgenommen wurden und in \autoref{fig:14} die Werte 1,2 und 9 ausgenommen wurden. Somit ergeben sich deutlich sinnvollere 
und genauere Dämpfungsfaktoren.
 
\begin{figure}[H]
    \centering
    \caption{Messwerte und regression der Amplituden Mit der 2 $\unit{\mega\hertz}$ Sonde ohne die Werte 1,2,3 und 7}
    \label{fig:13}
    \includegraphics{build/amplituden_alu2mhzbearbeitet.pdf}
\end{figure}
Die Parameter der Ausgleichsrechnung ergeben sich zu 
\begin{align*}
    a &= \qty{-11.6(2.9)}{1 \per\meter}\\
    b &= \qty{0.13(0.16)}{}
\end{align*}
Somit ergibt sich ein Dämpfungsfaktor von 
\begin{equation}
    d_{1\unit{\mega\hertz},angepasst} = \qty{-11.6(2.9)}{1 \per\meter}.
\end{equation}

\begin{figure}[H]
    \centering
    \caption{Messwerte und regression der Amplituden Mit der 1 $\unit{\mega\hertz}$ Sonde ohne die Werte 1,2 und 9}
    \label{fig:14}
    \includegraphics{build/amplituden_alu1mhzbearbeitet.pdf}
\end{figure}
Die Parameter der Ausgleichsrechnung ergeben sich zu 
\begin{align*}
    a &= \qty{-1.31(0.92)}{1 \per\meter}\\
    b &= \qty{-0.61(0.06)}{meter}
\end{align*}
Somit ergibt sich ein Dämpfungsfaktor von 
\begin{equation}
    d_{2 \unit{\mega\hertz},angepasst} = \qty{-1.31(0.92)}{1 \per\meter}.
\end{equation}

\subsubsection{Uhrsprüngliche Messwerte}
\begin{figure}[H]
    \centering
    \caption{Messwerte und regression der Amplituden Mit der 2 $\unit{\mega\hertz}$ Sonde.}
    \label{fig:131}
    \includegraphics{build/amplituden_alu1mhz.pdf}
\end{figure}
\begin{figure}[H]
    \centering
    \caption{Messwerte und regression der Amplituden Mit der 1 $\unit{\mega\hertz}$ Sonde }
    \label{fig:141}
    \includegraphics{build/amplituden_alu1mhz.pdf}
\end{figure}
In \autoref{fig:131} und \autoref{fig:141} ist zu sehen, dass wenn 
man die ursprünglichen Messdaten für die regression nutzt, eine Positive Steigung 
das ergebnis wäre. Dies macht Physikalisch keinen Sinn, da es bedeuten würde das Material verstärkt die Schwingungen. 

\subsection{Kallibrierkurve}
Es wird davon ausgegangen, dass der Füllstand eines Elenmeyer-kolbens 
durch eine Funktion dritten grades genähert werden kann. Es werden die Messwerte 
aus \autoref{tab:14} an eine Funktion der Form $y = a \cdot x^2 + b \cdot x + c $ gefittet 
um die Parameter der Füllstandsfunktion zu bestimmen. 
\begin{figure}[H]
    \centering
    \caption{Messwert und regression des Füllstandes gegen die Laufzeit.}
    \label{fig:15}
    \includegraphics{build/fuellstand.pdf}
\end{figure}

Mit den Parametern der regression ergibt sich die Füllstandsfunktion zu 
\begin{equation}
    u_\text{füllstand}(x) = 0 \cdot x^2 + 0.21 \cdot x + 8.69 
\end{equation}

