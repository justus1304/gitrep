\section{Auswertung}
\label{sec:Auswertung}

\subsection{Verifizierung der Bragg-Bedingung.}
\begin{figure}
    \centering
    \caption{Messwerte zur Bragg-Überprüfung}
    \label{fig:brag}
    \includegraphics{build/bragg.pdf}
\end{figure}
Anhand von \autoref{fig:brag} lässt sich das Maximum ziemlich genau bei 
$\theta = 28\unit{\degree}$ erkennen. Da das Medium, an welchen 
die Photonen reflektiert werden in einem Winkel von $14\unit{\degree}$ stand, 
ist das Maximum bei dem Winkel erwartbar. (Einfallswinkel = Ausfallswinkel).

\subsection{Das Emissionsspektrum einer Cu-Roentgenroehre}
In \autoref{fig:spektrum} sind die Messwerte geplotet. Eingezeichnet sind auch der Bremsbergm sowie die $K_\alpha $ und $K_\beta $-Linien.
Der Grenzwinkel bestimmt sich durch ablesen zu 
\begin{equation}
    \Theta_\text{G} = \qty{5.7(0.1)}{\degree}
\end{equation}
daraus ergibt sich nach \autoref{eqn:E} und \autoref{eqn:B}
eine minimale Wellenlänge, bzw eine maximale Energie ableiten,
 \begin{align}
    \lambda_\text{min} &= \qty{40(0.7)}{\pico\meter}\\
    E_\text{max} &= \qty{49.7(0.7)}{\kilo\electronvolt}
 \end{align}
\begin{figure}
    \centering
    \caption{Emmisionsspektrum einer Cu-Röhre.}
    \label{fig:spektrum}
    \includegraphics{build/wertejust.pdf}
\end{figure}
Um nun das Auflösevermögen der Apparatur zu bestimmen Werden die K-Linien noch einmal 
vergrößert in \autoref{fig:zoom} dargestellt. Des Weiteren wurden die Wargerechten linien 
auf der höhe der beiden Halbwertsbreiten der K-Linien eingezeichnet. Daraus ergeben sich jewails zwei 
Winkel aus den Schnittpunkten mit der $K_\alpha$ und $K_\beta$-Linie.
Für $K_\alpha$ ergibt sich $\Theta_{min,\alpha} = \qty{22(0.1)}{\degree}$ 
, $\Theta_{max,\alpha} = \qty{22.7(0.1)}{\degree}$ sowie für $K_\beta
$ $\Theta_{min,\beta} = \qty{19.8(0.1)}{\degree}$ und $\Theta_{max,\alpha} = \qty{20.4(0.1)}{\degree}$
Durch \autoref{eqn:E} werden daraus erneut die Energien ermittelt. Das Auflösungsvermögen lässt sich dann 
durch folgende Gleichung bestimmen.Die Ergebnisse sind \autoref{tab:auf} zu entnehmen.

\begin{equation}
    A = \frac{E}{\Delta E}
\end{equation}
\begin{table}[H]
  \centering
  \caption{Berechnete Energien und Energiedifferenzen.}
  \label{tab:auf}
  \begin{tblr}{
          colspec = {S S S S S S},
          row{1} = {guard, mode = math},
      }
      \toprule
      Linie&IE/\unit{\kilo\electronvolt}& \Delta E /\unit{\kilo\electronvolt}&A\\
      \midrule
      $\alpha$ & 8.10 +- 0.04 & 0.26+-0.07 & 31+-8\\
      $\beta$ & 8.95 +- 0.04 & 0.26+-0.07 & 34+-9\\
      \bottomrule
  \end{tblr}
\end{table}
\begin{figure}
    \centering
    \caption{Zoom auf $K_{\alpha} $und $ K_{\beta}$ Lienien des Emissionsspektrums}
    \label{fig:zoom}
    \includegraphics{build/zoom.pdf}
\end{figure}
Aus Den Gleichungen \autoref{eqn:a}, \autoref{eqn:b} sowie \autoref{eqn:c} Lassen sich mit den berechneten Werten für die 
Energien die Abschirmkonstanten berechnen. bei der berechnung von $\sigma_1$ Wird für die Energie $E_\text{K,abs} = \qty{8.987}{\kilo\electronvolt}
$ als Theoriewert angenommenm, da dieser nicht experimentell bestimmt wurde. Die berechneten Abschirmkonstanten befinden sich in \autoref{tab:ab}
\begin{table}[H]
  \centering
  \caption{Berechnete Abschirmkonstanten.}
  \label{tab:ab}
  \begin{tblr}{
          colspec = {S S S S S S},
          row{1} = {guard, mode = math},
      }
      \toprule
      \sigma & experimenteler Wert&Theoretischer Wert\\
      \midrule
       \sigma 1 & 0& 3.29\\
       \sigma 2& 12.85+-0.36 & 12.14\\
       \sigma 3& 24.05+-2.67 & 24.04\\
      \bottomrule
  \end{tblr}
\end{table}


\subsection{Das Absorptionsspektrum}
In den folgenden Abbildungen sind die Messwerte für das Absorptionsspektrum graphisch dargestellt.
Daraus wird der Braggwinkel $\Theta_\text{k,exp}$ und zur besseren Übersicht ist dazu noch der Mittelwert der Messwerte eingezeichnet.
Aus dem Bragwinkel werden wie schon im letzen Unterabschnitt die Energien bestimmt und letztendlich mit \autoref{sigma} die Abschirmkonstanten bestimmt. 
Die ergebnisse sind in \autoref{tab:abs} dargestelt.
\begin{table}[H]
  \centering
  \caption{Berechnete Braggwinkel,Absorptionsenergie und Abschirmkonstanten.}
  \label{tab:abs}
  \begin{tblr}{
          colspec = {S S S S S S},
          row{1} = {guard, mode = math},
      }
      \toprule
      Material & \Theta_\text{k,exp}\unit{\degree}&E_\text{k,exp}\unit{\degree}&E_\text{k,lit}\unit{\kilo\electronvolt}&\sigma_{K,exp}&\sigma_{K,lit}\\
      \midrule
      Strontium & 10.95 +- 0.1 & 16.19+-0.15&16.12&3.50+-0.09& 3.61\\
      Gallium &17.12+-0.1 & 10.46+-0.05 & 11.11& 3.26+-0.08 &3.41\\
      Brom &13.1+-0.1 & 13.58+-0.11 & 13.48& 3.41+-0.06&3.83\\
      Zink &18.45+-0.1 & 9.74+-0.05 & 9.65& 4.26 +- 0.06 &3.37\\
      Zirkonium &9.85+-0.1 & 17.99+-0.17 & 18.01& 3.62+-0.17&3.65\\
      \bottomrule
  \end{tblr}
\end{table}
\begin{figure}
    \centering
    \caption{Ermittlung des Strontium Bragwinkels}
    \label{fig:zink}
    \includegraphics{build/strontium.pdf}
\end{figure}

\begin{figure}
    \centering
    \caption{Ermittlung des Gallium Braggwinkels}
    \label{fig:}
    \includegraphics{build/gallium.pdf}
\end{figure}

\begin{figure}
    \centering
    \caption{Ermittlung des Zink Braggwinkels}
    \label{fig:}
    \includegraphics{build/zink.pdf}
\end{figure}

\begin{figure}
    \centering
    \caption{Ermittlung des Zirkonium Braggwinkels}
    \label{fig:}
    \includegraphics{build/zr.pdf}
\end{figure}

\begin{figure}
    \centering
    \caption{Ermittlung des Brom Braggwinkels}
    \label{fig:}
    \includegraphics{build/brom.pdf}
\end{figure}


