\section{Diskussion}
\label{sec:Diskussion}

\subsection{Statische Methode}
In diesem Abschnitt sollen die Graphen der statischen Methode diskutiert werden.
Festzuhalten ist, dass es zu einem auffälligen Messfehler nach etwa 630s kam.
Dies ist durch die Apparatur zu begründen. Da die Messaparatur mehrere Jahre 
alt ist, kam es zu zahlreichen falschen Messungen. Es brauchte mehrere Versuche, 
um Werte zu erhalten, welche sich wissenschaftlich wiederverwerten lassen ließen.
Jenes zeichnet sich in den Graphen ab. Tatsächlich sollten die Graphen aus 
\autoref{fig:f1} und \autoref{fig:f2} keine Beugung aufweisen, sondern linear 
weiterverlaufen. Analog sollten die Graphen aus \autoref{fig:f3} und \autoref{fig:f4}
verlaufen.
\par\vspace{0.5em}
\noindent Jedoch liefern die Graphen sinnvolle Ergebnisse. Beide Graphen 
haben einen Peak, bei welchem die Differenz der Thermoelemente maximal wird.
Das kommt vermutlich daher, dass der innere Sensor stärker erhitzt wurde als 
der Äußere, da dieser näher am Peltier-Element liegt. Eigenschaften wie die 
Stauchung/Streckung des Graphen oder etwa die Position hängen dabei von der 
Wärmeleitung ab. Da bei $T1$ und $T2$ Wärme schneller geleitet wird, ist ein 
gestreckter Graph zu erwarten, was hier zutriftt. Der darauffolgende asymptotische 
Abfall des Graphen ist mit des Heizvorgangs des äußeren Thermoelements zu verstehen;
Dieses heizt nach und die Temperaturdifferenz mit dem innen liegenden Sensor 
verringert sich, sodass sich der Graph auf einen Abfall einstellt.

\subsection{Dynamische Methode}

\subsection{Messing}
\noindent Unser Ergebnis für die Wärmeleitfähigkeit bei Messing lautet $
\kappa_{Messing} = \qty{103(11)}{\watt\per\meter\kelvin}$. In der 
Literaatur wird für Messing eine Wärmeleitfähigkeit von $81-105 \unit{\watt\per\meter\kelvin}
$ angegeben. Somit würde das Ergebnis in den Bereich passen, wenn auch in das obere Ende. 
\subsection{Aluminium}
\noindent Aluminium hat in der Literatur einen Wert von ca $235 \unit{\watt\per\meter\kelvin}$ 
Was einer Abweichung von $5.5$ \% zu dem von uns Berechneten Wert $\kappa_{Aluminium} = \qty{248(23)}{\watt\per\meter\kelvin}$ 
entspricht. Jedoch ist der Fehler, welcher aus den Mittelwertsbildungen und der
Fehlerfortpflanzung resultiert mit $ 23 \unit{\watt\per\meter\kelvin}$ziemlich groß, sodass der literaturwert 
im Fehlerbereich der ermittelten Größe liegt. 
\subsection{Edelstahl}
\noindent Der Literaturwert der Wärmeleitfähigkeit von Edelstahl beträgt $15 \unit{\watt\per\meter\kelvin}$
Der hier ermittelte Wert beträgt $\kappa_{Edelstahl} = \qty{15.9(1.4)}{\watt\per\meter\kelvin}$.Das entspricht einer 
Abweichung von $1.6 \% $. Diese Abweichung ist sehr gering und liegt auch im Fehlerbereich von $1.4 \unit{\watt\per\meter\kelvin}$.\\[0.2cm]

\noindent Die Ermittelten Größen geben einen guten Richtwert für die Leitfähigkeiten der verschiedenen Materialien ab.
Durch Mittelwertbildungen und ungenauigkeiten wie eine nicht perfekte Wärmeisolation und offensichtliche defizite im 
versuchsaufbau (Statische Methode hat nicht funktioniert) kommen jedoch etwas größere Fehler 
in die Ermittelte Größe, wesshalb sie kein genauen wert zur angabe der Wärmeleitfähigkeiten liefern.