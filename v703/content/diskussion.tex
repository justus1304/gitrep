\section{Diskussion}
\label{sec:Diskussion}

\subsection{Kennlienie}
Bei zunehmendem Alter des Zählrohres kommt es zu einer höheren Warscheinlichkeit 
für Nachentladungen im Zählrohr. Dies macht sich in einer höheren Plateausteigung 
bemerkbar, welche einen Wert für die Güte des Zählrohres angibt.
Diese Steigung beträgt hier \qty{0.178}{\%}. Das ist eine sehr niedrige Steigung,
jenes ist auch in \autoref{fig:10} zu erkennen. Das Ergebnis passt nicht zu 
den Erwartungen, passt aber zur Theorie: Idealerweise sollte die Steigung so 
gering wie möglich sein.

\subsection{Bestimmung der Totzeit}
Die Totzeit wurde auf zwei unterschiedlichen Arten berechnet.
Es ergab sich: 
\begin{equation}
    \tau_{Zwei-Quellen} = \qty{129.0}{\micro\second} 
\end{equation}
\begin{equation}
    \tau_{Oszilloskop} = \qty{150}{\micro\second} 
\end{equation}
Das Ablesen vom Oszilloskop hat eine Abweichung von $16 \%$ von dem 
ermittelten Wert durch die zwei Quellen-Mehode, die tatsächliche 
Totzeit wird also in dem Bereich liegen.