\section{Diskussion}
\label{sec:Diskussion}

\subsection{Bragg-Bedingung}
Zunächst ist festzuhalten, dass der auffällige plötzliche Fall der geplotteten 
Daten auf einen systematischen Fehler zurückzuführen ist. Bei einer Detektion 
der Intensität von Null handelt es sich um einen Fehler des Messgeräts, welcher 
für die Bestimmung des Bragg-Winkels ignoriert werden kann. Der im Experiment
gemessene Beugungswinkel von 28° entspricht exakt dem Doppelten des eingestellten
Einfallswinkels (14°) zur Netzebene. Dies steht in Einklang mit der Bragg-Theorie,
nach der der Beugungswinkel $2 \theta$ beträgt, wenn $\theta$ der Winkel zwischen
einfallendem Strahl und den Kristallebenen ist. Da hier $\theta$ = 14° sowohl
als Einfallswinkel (zur Netzebene) gewählt wurde als auch aus dem Beugungswinkel 
folgt, bestätigt dies die Gültigkeit der Bragg-Bedingung für diese Messung.

\subsection{Emissionsspektrum}
Als experimentelle Größen ergaben sich
\begin{align*}
    \theta_\text{G}    &= \qty{5.7(0.1)}{\degree}     \\
    \lambda_\text{min} &= \qty{40(0.7)}{\pico\meter}  \\
    E_\text{max}       &= \qty{49.7(0.7)}{\kilo\electronvolt},
\end{align*}
sowie mit einer gleichen Energiedifferenz von $0.26 \pm 0.07$ für das Auflösungsvermögen
\begin{align*}
    A_{\alpha} &= 31 \pm 8 \\
    A_{\beta}  &= 34 \pm 9 \\
\end{align*}
Jene Werte sind als Quantifizierung des Experiments, die Energien zu unterscheiden, 
zu verstehen, Werte der vorliegenden Größenordnung bedeuten eine erkennbare Differenz, 
was mit dem Plot übereinstimmt, da zwei Maxima zu erkennen sind.
\vspace{0.5em}
\\
\noindent Für die Abschirmkonstanten können nur Vergleiche für $\sigma_2$ und 
$\sigma_3$ aufgestellt werden, da sich $\sigma_1$ nicht experimentell bestimmt 
wurde. Für die beiden Werte ergaben sich Abweichungen von $5.53 \%$ und $0.04 \%$,
was auf das Auflösevermögen zurückzuführen sein könnte. Aus der geringen Auflösung 
resultiert eine größere Unsicherheit von $E_K$. Ebenso führt eine geringe 
Quantität an Messwerten zu Abweichungen von $E_K$, so überträgt sich ein möglicher 
Fehler weiterhin auf $\sigma_K$.

\subsection{Absorptionsspektrum}
Die Abweichungen der Bragg-Winkel, der Absorptionsenergie und der Abschirmkonstanten 
zu den Materialien Strontium, Gallium, Brom, Zink und Zirkonium befinden sich 
in \autoref{tab:abspek}.
\begin{table}[H]
    \centering
    \caption{Abweichung der Bragg-Winkel, der Absorptionsenergie und der Abschirmkonstanten.}
    \label{tab:abspek}
    \begin{tblr}{
            colspec = {S S S S},
            row{1} = {guard, mode = math},
        }
        \toprule
        \text{Material} & \Delta \theta_{\text{K}} / \% & 
        \Delta E_{\text{K}} / \% & 
        \sigma_{\text{K}} / \% \\
        \midrule
        Sr & 0.54 & 0.43 & 3.05 \\
        Ga & 1.04 & 5.85 & 4.4 \\
        Br & 0.76 & 0.74 & 10.97\\
        Zn & 0.81 & 0.92 & 20.89 \\
        Zr & 0.10 & 0.11 & 0.82 \\
        \bottomrule
    \end{tblr}
\end{table}
\noindent Es fällt auf, dass die Messwerte größtenteils sehr nahe am Literaturwert
liegen und einen Prozentwert von 1 idR. nicht überschreitet. Allerdings gibt es 
einige Abweichungen, die herausstechen, so ist die Abschirmkonstante von Zink 
offenbar sehr fehlerbelastet. Als Ursache könnte Unreinheit der Probe gemutmaßt 
werden, das hohe Alter und die oftmalige Benutzung kann zu einer Abnutzung 
beigesteuert haben, was sich in den Abweichungen manifestiert.

\subsection{Rydbergkonstante}
Ausgehend von einer Rydbergkonstante von $R_{\infty,\text{lit}} = 13.605 \,
\unit{\eV}$ \cite{ryd} beläuft sich die Abweichung zu $R_{\infty,\text{exp}} 
= 11.76\,\text{eV}$ auf $13.56 \%$. In Anbetracht der vorhergegangen Fehler 
von Absorptionsenergie und Abschirmungskonstante (siehe \autoref{tab:abspek})
war eine derartige Abweichung zu erwarten, da die Rydberg-Energie aus jenen 
resultiert.