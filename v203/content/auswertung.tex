\section{Auswertung}
\label{sec:Auswertung}

\subsection{Verdampfungswärme L im bereich bis 1 Bar}
Hier soll die Verdampfungswärme $\symbf{L}$ für Wasser im Bereich von 
$30 - 1000$ mBar experiementell ermittelt werden. Bei der Verdampfung von Wasser
wurden folgende Werte für die Dampftemperatur $\symbf{T}/\unit{\celsius}$ und
den zugehörigen Druck $\symbf{p}/\unit{\milli\bar}$ gemessen.  

  \begin{longtblr}{
      colspec = {S S },
      row{1} = {guard, mode=math},
      %vline{4} = {2}{-}{text=\clap{$\pm$}},
    }
    \toprule
    T/\unit{\celsius} & p/\unit{\milli\bar}\\
    \midrule
    19  & 48      \\   
    20  & 51      \\   
    21  & 53      \\   
    22  & 56      \\   
    24  & 59      \\   
    24  & 60      \\   
    25  & 63      \\   
    26  & 64      \\   
    27  & 66      \\   
    28  & 67      \\   
    29  & 67      \\   
    26  & 71      \\   
    27  & 75      \\   
    28  & 80      \\   
    29  & 83      \\   
    30  & 85      \\   
    31  & 87      \\   
    32  & 90      \\   
    33  & 96      \\   
    34  & 99      \\   
    35  & 102     \\   
    36  & 105     \\   
    37  & 109     \\   
    38  & 113     \\   
    39  & 117     \\   
    40  & 121     \\   
    41  & 125     \\   
    42  & 129     \\   
    43  & 134     \\   
    44  & 137     \\   
    45  & 142     \\   
    46  & 147     \\   
    47  & 151     \\   
    48  & 156     \\   
    49  & 162     \\   
    50  & 167  \\
    51  & 172\\
    52  & 177\\
    53  & 183\\
    54  & 189\\
    55  & 195\\
    56  & 200\\
    57  & 206\\
    58  & 212\\
    59  & 220\\
    60  & 226\\
    61  & 234\\
    62  & 242\\
    63  & 250\\
    64  & 259\\
    65  & 268\\
    66  & 277\\
    67  & 287\\
    68  & 298\\
    69  & 309\\
    70  & 322\\
    71  & 336\\
    72  & 349\\
    73  & 363\\
    74  & 378\\
    75  & 394\\
    76  & 410\\
    77  & 427\\
    78  & 443\\
    79  & 460\\
    80  & 476\\
    81  & 497\\
    82  & 517\\
    83  & 536\\
    84  & 557\\
    85  & 575\\
    86  & 594\\
    87  & 633\\
    88  & 646\\
    89  & 658\\
    90  & 690\\
    91  & 713\\
    92  & 741\\
    93  & 765\\
    94  & 794\\
    95  & 821\\
    96  & 852\\
    97  & 881\\
    98  & 911\\
    99  & 948\\
    100 & 975     \\   
   \bottomrule
  \end{longtblr}

\noindent Um aus diesen Messwerten einen Ausdruck für die Verdampfungswärme
 $\symbf{L}$ zu bekommen, nutzen wir die in der Theorie erwähnte 
 Gleichung
 \begin{equation}
  \ln \left(\frac{p}{p_0}\right) = -\frac{L}{R} \cdot \frac{1}{T}.
 \end{equation}
Da angenommen wird, dass $\symbf{L}$ im Druckbereich von bis zu ein Bar 
als konstant angesehen werden kann, kann Verdmpfungswärme über
eine lineare Ausgleichsrechnung mit $\ln\left(\frac{p}{p_0}\right)$ in 
Abhängigkeit von $\frac{1}{T}$ berechnet werden.
Für den Parameter $a$ der Ausgleichsgeraden mit der Geradengleichung
\begin{equation}
  y = m \cdot x + b
\end{equation}
gilt dann 
\begin{equation*}
  \label{eqn:a}
  a = -\frac{L}{R}
\end{equation*}

\begin{figure}[H]
  \centering
  \includegraphics{linreg.pdf}
  \label{fig:1}
  \caption{Lineare Ausgleichsrechnung}
\end{figure}

\noindent Mit den Parametern der Ausgleichsgeraden $a = \num{-3874.9574 +- 33.1933}
$ und $b = \num{21.787 +- 0.10}$ ergibt sich mit \autoref{eqn:a}
umgestellt zu
\begin{equation}
  L = -Ra
\end{equation}
mit der idealen Gaskonstante $R =  8,314[\unit{\joule\per\mol\per\kelvin}] $ 
folgender Wert für die Verdampfungswärme:
\begin{equation}
  \label{eqn:vdw}
  L = \num{3.22 +- 0.02 e4} \unit{\joule\per\mol}
\end{equation}
Nun wird die äußere Verdampfungswärme $\symbf{L_a}$ mithilfe der idealen 
Gasgleichung (\autoref{eqn:igg}) abgeschätzt. $L_a$ ist die Arbeit, welche 
in Form von Wärme an einem Mol Wasser verrichtet werden muss, um es
in den gasförmigen Aggregatzustand zu überführen. 
\begin{equation}
  L_a = RT
\end{equation}
Für eine Temperatur von $373 \unit{\kelvin}$ ergibt sich eine äußere 
Verdampfungswärme von 
\begin{equation*}
  L_a = \qty{3101.12}{\joule\per\mol}.
\end{equation*}
Damit lässt sich durch Einsetzen in \autoref{eqn:l} jene Arbeit $\symbf{L_i}$
berechnen, welche notwendig ist um die molekularen Anziehungskräfte 
bei der Verdampfung zu überwinden.
\begin{equation}
  \label{eqn:l}
  L_i = L - L_a
\end{equation}
Einsetzen der Werte für die Verdampfungswärme aus \autoref{eqn:vdw} und der 
eben berechneten äußeren Verdampfungswärme ergibt den Wert $L_i = \qty{2.912(0.028)e04}{\joule\per\mol}$.
Um das Ergebnis in Elektronenvolt $(\unit{\eV})$ pro Molekül angeben zu können, wird es durch die 
Avogadro Konstante $N_A = 6.022 \cdot 10^23 \quad \unit{\per\mol}$,
die Anzahl der Moleküle pro einem Mol, geteilt, was den Wert den Wert 
\qty{0.301(0.002)}{\eV} ergibt (mit $\qty{1}{\eV} = \qty{1.602e-19}{\joule}$)

Zusätzlich die ideale Gasgleichung:
\begin{equation}
  \label{eqn:igg}
  pV = RT
\end{equation}



\subsection{Verdampfungswärme im Bereich 1-15 Bar}
Nun wird die Verdampfungswärme im Bereich von 1 - 15 Bar bestimmt. 
Da in diesem Bereich $V_D$ in \autoref{eqn:2} nicht mehr durch die ideale 
Gasgleichung beschrieben werden kann, wird die Verdampfungswärme in diesem 
Druckbereich temperaturabhängig. $V_D$ kann nun über folgende Formel bestimmt 
werden.
\begin{equation}
  \left(p + \frac{a}{V^2}\right)V = RT \Leftrightarrow V_D = \frac{RT}{2p} \pm \sqrt{\left(\frac{RT}{2p}\right)^2-\frac{a}{p}}
\end{equation}
Dabei ist $a$ eine Konstante der Größe $0.9 \unit{\joule\cubic\meter\per\square\mole}$.

  %Tabelle Messwerte
\begin{table}[h]
  \centering 
  \caption{}
  \label{tab:messwerte2}
  \begin{tblr}{
  colspec = {S S },
  row{1} = {guard, mode=math},
    %vline{4} = {2}{-}{text=\clap{$\pm$}},
  }
  \toprule
  T/\unit{\celsius} & p/\unit{\bar}\\
  \midrule    
  118 & 1 \\
  131 & 2\\
  148 & 3\\
  155 & 4\\
  161 & 5\\
  166 & 6\\
  172 & 7\\
  176 & 8\\
  181 & 9\\
  185 & 10\\
  188 & 11\\
  194 & 12\\
  194 & 13\\
  203 & 14\\                                                                                              
 \bottomrule
  \end{tblr}
\end{table}

\noindent Die Messwerte aus Tabelle \autoref{tab:messwerte2} werden nun in einem T/p 
Diagramm durch ein Ausgleichspolynom vom Grad 3 genähert. um daraus den 
Differentialkoeffizienten $dp/dT$ bestimmen zu können, wird das 
Ausgleichpolynom nach der Temperatur abgeleitet.

\begin{figure}[H]
  \centering
  \includegraphics[width=0.7\textwidth]{ausgleichspolynom.pdf}
  \label{fig:2}
  \caption{Ausgleichspolynom 3. Grades}
\end{figure}

\noindent Es ergibt sich das Ausgleichspolynom der Form 
\begin{equation*}
  p(T) = a \cdot x^3 + b \cdot x^2 + c \cdot x + d 
\end{equation*}
\noindent und die resultierende Ableitung
\begin{equation*}
  \frac{dp(T)}{dT} = 3a \cdot x^2 + 2b \cdot x + c 
\end{equation*}
mit den Werten der Parameter des Ausgleichspolynoms:
\begin{align*}
  a &= \num{-1.057 +- 0.737}\\
  b &= \num{1531.298 +- 959.786}\\
  c &= \num{-714342.798 +- 415959.850}\\
  d &= \num{108512685.516 +- 59968812.507}\\
\end{align*}

Damit ist der Differentialkoeffizienten $dp/dT$ über die
Ausgleichsrechnung dargestellt. Nun ist es möglich, durch Einsetzen in
\autoref{eqn:2}, umgestellt nach der Verdampfungswärme $L$,
diese in Abhängigkeit der Temperatur bestimmen.
\begin{equation}
L(T) = T \left(\frac{RT}{2p}\pm\sqrt{\left(\frac{RT}{2p}\right)^2-\frac{k}{p}}\right) \frac{dp}{dT}
\end{equation}
\noindent Diese Werte sind in \autoref{fig:e} und \autoref{fig:3} 
für die Fälle der Addition, sowie der Subtraktion der 
Wurzel aufgetragen.

% \begin{figure}[h]
%   \centering
%   \begin{minipage}{0.45\textwidth}
%       \centering
%       \includegraphics[width=\textwidth]{minus.pdf}
%       \caption{Verdampfungswärme bei Subrtaktion der Wurzel}
%       \label{fig:e}
%   \end{minipage}
%   \hfill
%   \begin{minipage}{0.45\textwidth}
%       \centering
%       \includegraphics[width=\textwidth]{plus.jpg}
%       \caption{Verdampfungswärme bei Addition der Wurzel}
%       \label{fig:3}
%   \end{minipage}
% \end{figure}






negative wurzel
\begin{figure}[H]
  \centering
  \includegraphics[width=0.5\textwidth]{minus.pdf}
  \caption{Verdampfungswärme bei Subrtaktion der Wurzel}
  \label{fig:e}
\end{figure}

%positive wurzel
\begin{figure}[H]
  \centering
  \includegraphics[width=0.5\textwidth]{plus.pdf}
  \caption{Verdampfungswärme bei Addition der Wurzel}
  \label{fig:3}
\end{figure}


