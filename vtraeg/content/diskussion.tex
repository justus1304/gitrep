\section{Diskussion}
\label{sec:Diskussion}

Im Folgenden wird eine Diskussion des gesamten Versuchs durchgeführt. Dabei
ist nicht bekannt, ob es sich um zwangsläufig signifikante Werte handelt, was
größtenteils auf die Tatsache zurückgeht, dass keine Literaturwerte zum 
Vergleich herangezogen werden können.

\subsection{Winkelrichtgröße}
\label{chap:winkelrichtgroesse}
Die experimentelle Messung der Winkelrichtgröße $D$ gibt einen Wert von
\begin{equation}
    D = 0.00002790 \pm 0.000003259 \frac{N \cdot m}{rad}.
\end{equation}
Dieser Wert ist fehlerbehaftet, was auf einige Fehlerquellen zurückzuführen
ist. So ist eine präzise Auslenkung der Federwaage nur sehr erschwerlich
realisierbar. Das liegt daran, dass zum Einen die Platte mit den Winkeln
nicht mit der Apparatur befestigt ist, sie liegt eben auf. Demnach kann es
dazu kommen, dass diese selbst ausgelenkt wird oder sogar abrutscht. Genau
so anspruchsvoll ist das genaue Ablesen eines Winkels auf der Kreisscheibe.
Da der ausgelenkte Hebelarm einen Radius von mehreren Millimetern hat und
somit einen verhältnismäßig großen Teil der Scheibe abdeckt, ist eine 
genaue Angabe eines Ausrichtungswinkels aus der Draufsicht kaum möglich. Ein
ähnliches Problem ergibt sich bei der Kraftmessung; die Bestimmung durch
Augenmaß erzeugt bereits Fehler. Da die Federwaage selbst dem statistischen
Fehler unterliegt, ergeben sich noch größere Fehler. Darüber hinaus erschwert
sich das Messen der Kraft durch die Ausrichtung der Federwaage. Zum Einen
kann die senkrechte Ausrichtung (zur Drillachse) nur ungefähr erfolgen, da
mit Augenmaß gearbeitet wird. Andererseits verändert sich die Kraft, je nach
Winkel auf horizontaler Ebene. Jenes ist umso komplizierter zu berücksichtigen,
da eine horizontale Auslenkung bei einer Draufsicht kaum zu identifizieren ist.

\subsection{Eigenträgheitsmoment}
\label{chap:eigentraeg}
Im Versuch wurde für das Eigenträgheitsmoment ein Wert von
\begin{equation*}
    I_D = -0.005072 \pm 0.000004 \quad kg \cdot m^2
\end{equation*}
bestimmt, was keinen physikalischen Sinn ergibt und demnach auf einen Fehler
in der Messung hinweist. Um $I_D$ zu bestimmen, ist vor Allem die gemessene 
Zeit T sowie die Winkelrichtgröße D ausschlaggebend. Die Messung von T 
unterliegt dem statistischen Fehler, da sich die Reaktionszeit des Messers 
nicht eliminieren lässt. Andererseits ist Komponente D wie oben beschrieben 
mit Fehlern behaftet.

\subsection{Trägheitsmoment Kugel und Zylinder}
\label{chap:kuzi}
Die Bestimmung des theoretischen Trägheitsmoments der Kugel beläuft sich auf
\begin{equation*}
    I_K = 0.0009350 \pm 0.0000034 \quad kg \cdot m^2
\end{equation*}
und für den Zylinder auf
\begin{equation*}
    I_Z = 0.000444 \pm 0.0000022 \quad kg \cdot m^2.
\end{equation*}

Die experimentellen ergebnisse liefern folgende Trägheitsmomente.
\begin{equation*}
    I_{Kugel} = \qty{0.97(0.1)e-3}{\kilo\gram\meter\squared}
\end{equation*}
\begin{equation*}
    I_{zylinder} = \qty{0.39(0.05)e-3}{\kilo\gram\meter\squared}
\end{equation*}

Die Fehler bei der Bestimmung des Moments entstehen durch ähnliche Gründe
wie \ref{chap:winkelrichtgroesse} ; Das Bestimmen von $\varphi$ ist unpräzise,
was eine Auslenkung ebenso ungenau macht. Hinzu kommt eine Ungenauigkeit 
durch die fehlerbehaftete Zeitmessung T wie in \ref{chap:eigentraeg} erläutert.
Die abweichungen zwischen experimentellem und Thoretischen Ergebnissen
sind jedoch ziemlich gering Kugel(4.2\%) und Zylinder(12.8\%). Das liegt daran, dass es ziemlich gut möglich ist die Theoretischen 
Trägheitsmomente dieser symmetrischen Körper auszurechnen, sovern die Masse und 
das Volumen bekannt ist. Des weiteren wurden die Umlaufzeiten 10fach gemessen und 
gemittelt, um fehler Möglichst klein zu halten.

\subsection{Trägheitsmoment Puppe}
Für die Puppe ergibt sich nach theoretischer Berechnung das Trägheitsmoment
in Position 1 zu 
\begin{equation*}
    I_{pos1}= 0.000292 \pm 0.000018 \quad kg \cdot m^2
\end{equation*}
und in Position 2 zu 
\begin{equation*}
    I_{pos2}= 0.001090 \pm 0.000085 \quad kg \cdot m^2.
\end{equation*}
Währenddessen ergibt sich für das Trägheitsmoment nach experimenteller
Bestimmung in Position 1 der Wert
\begin{equation*}
    I_{pos1}= 0.00044 \pm 0.00005 \quad kg \cdot m^2
\end{equation*}
und für Position 2
\begin{equation*}
    I_{pos2}= 0.00189 \pm 0.00023 \quad kg \cdot m^2.
\end{equation*}
Das bedeutet eine Abweichung um 50,68\% des theoretischen Werts zum
experimentellen Wert in Position 1 und eine 73,39\%-ige Abweichung zwischen
dem experimentellen und theoretischen Wert in Position 2. Diese Abweichungen
sind von erheblichem Ausmaß, jedoch zu erwarten. Einerseits ist die Berechnung 
der theoretischen Werte schwierig zu realisieren. Da jedes Körperteil als 
Zylinder oder Kugel angenommen werden soll, entstehen dezente Abweichungen.
Bei zunehmender Komplexität der Pose verschlimmert sich der Fehler folglich.
In Anbetracht dessen und der größeren prozentualen Abweichung von experimentellem 
und theoretischem Messwert ist der Fehler bei Pose 2 größer als bei Pose 1.
Ähnlich wie bei \ref{chap:winkelrichtgroesse}, \ref{chap:eigentraeg} und 
\ref{chap:kuzi} ergibt sich ein Fehler für T. Im Gegensatz dazu ist der Fehler
hier größer, das liegt daran, dass bei der Puppe das Gewicht ungleich verteilt
ist, wodurch die Schwingung beeinflusst und eine Messung von genauen Zeiten
erschwert wird.