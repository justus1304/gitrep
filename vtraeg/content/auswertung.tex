\section{Auswertung}
\label{sec:Auswertung}


\subsection{Winkelrichtgröße D}
Im Folgenden wird zunächst die Winkelrichtgröße $D$ und das 
Eigenträgheitsmoment der Drehachse bestimmt. Dazu wird die
rücktreibende Kraft $F$ der Spiralfeder für verschiedene 
Auslenkungen $\varphi$ gemessen. \\
\begin{table}[H]
  \centering
  \caption{Messwerte zur Bestimmung von $D$.}
  \label{tab:tabelle}
  %\sisetup{table-format=1.1, per-mode=reciprocal}
  \begin{tblr}{
      colspec = {S S },
      row{1} = {guard, mode=math},
      %vline{4} = {2}{-}{text=\clap{$\pm$}},
    }
    \toprule
    \varphi(°) &  F(N)\\
    \midrule
    22.5  & 0\\
    45.0  & 0.04\\
    67.5  & 0.068\\
    90.0  & 0.09\\
    112.5 & 0.126\\
    135.0 & 0.164\\
    157.5 & 0.172\\
    180.0 & 0.18\\
    202.5 & 0.2\\
    225.0 & 0.28\\
   % \midrule
   % $\overline{\varphi}$ & $\overline{F}$\\
   % 123.75 & 0.132\\
   % \midrule
    \bottomrule
  \end{tblr}
\end{table}

Mithilfe von \eqref{eqn:gl9} und $r= 29,2cm$ kann die Winkelrichtgröße der 
Spiralfeder als Mittelwert von den errechneten
Größen für die einzelnen Messwerte zu %ich weiß nicht, was ich falsch mache, er will einfach nicht die gleichung im pdf referenzieren.
0.00002790 \pm 0.000003259 $\frac{N \cdot m}{rad}$ bestimmt werden. 

\subsection{Eigenträgheitsmoment $I_D$}
Als nächstes wird das Eigenträgheitsmoment $I_D$ der Drillachse bestimmt.
Um das zu tun, wird das Trägheitsmoment des gesamten System berechnet.
Dafür wird das Trägheitsmoment des Stabs um die Drehachse ermittelt, ebenso
wie das Trägheitsmoment der beiden Hohlzylinder, welche jeweils im Abstand
a zu der Drehachse stehen. Im Anschluss wird das Trägheitsmoment des Stabes
und der Hohlzylinder von dem Gesamtergebnis abgezogen, sodass sich als 
Ergebnis das gesuchte Trägheitsmoment der Drillachse ergibt.

das theoretische Trägheitsmoment 

 \begin{table}[H]
   \centering
   \caption{Messwerte T/a.}
   \label{tab:tabelle}
   %\sisetup{table-format=1.1, per-mode=reciprocal}
   \begin{tblr}{
       colspec = {S S },
       row{1} = {guard, mode=math},
       %vline{4} = {2}{-}{text=\clap{$\pm$}},
     }
     \toprule
     a(cm) & T(s)\\
     \midrule
     10    & 3.28\\
     11.5  & 3.71\\
     13    & 4.28\\
     14.5  & 4.75\\
     16    & 5.25\\
     17.5  & 5.90\\
     19    & 6.28\\
     20.5  & 6.78\\
     22    & 7.28\\
     23.5  & 7.97\\
     \midrule
     $\overline{T}$ & $\overline{a}$\\
     \midrule
     \bottomrule
   \end{tblr}
 \end{table}

\begin{figure}[H]
  \centering
  \includegraphics{plot.pdf}
  \caption{Plot.}
  \label{fig:plot}
\end{figure}

\subsection{Trägheitsmoment Kugel und Zylinder}
%Kugel
\begin{table}[H]
  \centering
  \caption{Abmessungen Holzkugel.}
  \label{tab:kugel}
  %\sisetup{table-format=1.1, per-mode=reciprocal}
  \begin{tblr}{
      colspec = {S S },
      row{1} = {guard, mode=math},
      %vline{4} = {2}{-}{text=\clap{$\pm$}},
    }
    \toprule
    r(m) & m(kg)\\
    \midrule
    0.0605\pm 0.0001  & 0.638\pm 0.001 \\
    \bottomrule
  \end{tblr}
\end{table}
%zylinder
\begin{table}[H]
  \centering
  \caption{Abmessungen Holzzylinder.}
  \label{tab:zylinder}
  %\sisetup{table-format=1.1, per-mode=reciprocal}
  \begin{tblr}{
      colspec = {S S S},
      row{1} = {guard, mode=math},
      %vline{4} = {2}{-}{text=\clap{$\pm$}},
    }
    \toprule
    r(m) & h(m) & m(kg)\\
    \midrule
    0.0493\pm 0.0001 & 0.0995\pm 0.0001 & 0.366\pm 0.001 \\
    \bottomrule
  \end{tblr}
\end{table}

Hier sollte das Trägheitsmoment von einer kugel und eines Zylinders berechnet werden. mit den maßen aus \autoref{tab:kugel}
und \autoref{eqn:gl5} berechnet sich $I_{Holzkugel} = (0.0009350\pm0.0000034)kg ⋅ m²$. und Das Trägheitsmoment des Zylinder
mit den abmessungen \autoref{tab:zylinder} eingesetzt in \autoref{eqn:gl6} zu $I_Holzzylinder = 
0.000444\pm0.0000022 kg ⋅ m²$.

\subsection{Experimentelles Trägheitsmoment Kugel}
Dann wird wir in \autoref{sec:zwei körper} das System zum schwingen gebracht und die Periodendauer 
dokumentiert. Daraus berechnen wir mit hilfe der \autoref{equ:gl10} nach dem Trägheitsmoment I umgestellt zu
\begin {equation}
\frac{T^2}{4*\pi^2}*D = \symbf{I}
\end{equation}
das experimentelle Trägheitsmoment der Kugel. Identische Vorgehensweise auch beim Zylinder.

%Messwerte T Kugel
\begin{table}[H]
  \centering
  \caption{Schwingungsdauer Kugel}
  \label{tab:Tkugel}
  %\sisetup{table-format=1.1, per-mode=reciprocal}
  \begin{tblr}{
      colspec = {S S},
      row{1} = {guard, mode=math},
      %vline{4} = {2}{-}{text=\clap{$\pm$}},
    }
    \toprule
    T_k(s) Kugel & T_Z(s) Zylinder\\
    \midrule
    1.13 & 0.78\\
    1.22 & 0.65\\
    1.15 & 0.78\\
    1.16 & 0.78\\
    1.16 & 0.78\\
    1.22 & 0.75\\
    1.22 & 0.72\\
    1.25 & 0.79\\
    1.16 & 0.75\\
    1.19 & 0.78\\
    \midrule
    $\overline{T_k}$ & $\overline{T_z}$\\
    \midrule
    1.186 \pm 0.0124 & 0.756 \pm 0.0135\\
    \bottomrule
  \end{tblr}
\end{table}

bei berücksichtigung der mittelwerten und dem zugehörigen Standardehler des Mittelwerts beläuft 
sich das Trägheitsmoment fur die Kugel auf $I_k = (0.00097 \pm 0.00011)$ und das Trägheitsmoment 
des Zylinders auf $I_z = (0.00039 \pm 0.00005)$


\subsection{Trägheitsmoment von zwei Puppen}
Um das Theoretische Trägheitsmoment einer Puppe in ihrer eingenommenen 
Position zuberechnen benötigen wir die Abmessungen der einzelnen körperteile, 
sowie derren Masse, Ausrichting und die Entfernung von der Rotationsaxe. 
Zuerst haben wir die Volumina der Einzelkörperteile der Puppe bestimmt. Wenn man diese Teilvolumina 
durch das Gesamtvolumen der Puppe teilt und erhällt man den anteil der Masse des Volumenstückes 
an der Gesamtmasse. so kann man alle größen vereinfacht angeben und zur theoretischen berechnung nutzen.



\begin{table}[H]
  \centering
  \caption{Abmessungen Puppe}
  \label{tab:maßePuppe}
  %\sisetup{table-format=1.1, per-mode=reciprocal}
  \begin{tblr}{
      colspec = {S S S S S},
      row{1} = {guard, mode=math},
      %vline{4} = {2}{-}{text=\clap{$\pm$}},
    }
    \toprule
    \SetCell[c=2]{c} R_{Arme} & & R_{Beine} & R_{Torso} & R_{Kopf}\\
    \midrule
    0.985  &0.875 & 1.25   &  1.885  & 1.580\\
    0.925  &0.905 & 1.305  &  2.415  & 1.725\\
    1.045  &0.95  & 1.19   &  2.195  & 1.715\\
    1.055  &0.91  & 1      &  2.03   & 1.65\\
    1.05   &0.81  & 0.95   &  3.67   & 1.59\\
    0.995  &0.775 & 1.010  &  2.14   & 1.425\\
    0.88   &0.71  & 1.075  &  2.485  & \\
    0.8    &0.655 & 0.97   &  2.495  & \\
    0.795  &0.8   & 0.825  &  2.36   & \\
    0.79   &1.95  & 0.78   &  1.925  & \\
    \midrule
    &$\overline{R_A}$  & $\overline{R_B}$ & $\overline{R_T}$ & $\overline{R_K}$\\
    \midrule
      &   &   &   \\
    \midrule
    $h_A$&&$h_B$&$h_T$&$h_K$\\
    \midrule
    \bottomrule
  \end{tblr}
\end{table}

In der \autoref{tab:maßePuppe} sind die gemittelten radien und die Höhen der Körperteile 
aufgetragen. (Am Anfang ein bischen übertriebene 10 Messwerte) wir Teilen die Puppe in ihre 
sechs Körperteile ein, welche alle als Zylinder genähert werden. Also können wir die Volumina 
der Körperteile mit der Formel 

\begin{equation}
  \symbf{V} = h * \pi * r^2
\end{equation}



