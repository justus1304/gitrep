\section{Diskussion}
\label{sec:Diskussion}

Berechnungen der Abweichung werden im Folgenden mithilfe von 
\begin{equation*}
    \increment x = \left| \frac{x_{exp}-x_{theo}}{x_{theo}} \right|
\end{equation*}
durchgeführt.

\subsection{Programm und Geräteeinstellungen}
Der experimentell bestimmte Wert für die Wellenlänge der Schwingungen in Acryl 
bei einer Periode von $T = \qty{0.61(0.04)}{\micro\second}$ berechnet sich zu
\begin{align*}
    \lambda &= \qty{1.69(0.11)}{\milli\meter}.
\end{align*}
Der vergleichbare im Vorhinein berechnete Wert lässt sich mit 
% \begin{align*}
%     \lambda_\text{Theo} &= \qty{1.36()}{\milli\meter}
% \end{align*}
\begin{align*}
    \lambda_\text{Theo} &= \qty{1.36(0.0)}{\milli\meter}.
\end{align*}
beziffern, was in einer Differenz von $19.53 \%$ resultiert. Das ist eine 
vergleichsweise hohe Abweichung, welche auf die händische Durchführung zurückgeführt 
werden könnte. Die Auswertung der Daten am Computer könnte an falschen Stellen 
erfolgt sein, sodass die Peaks nicht exakt getroffen wurden, sodass die Differenziation 
der Peaks mangelhaft war. Ebenso könnte zu wenig destilliertes Wasser als 
Gleitmittel genutzt worden sein, sodass die Signale mit Rauschen versehen waren, 
was das Ergebnis nachträglich verfälscht.

\subsection{Schallgeschwindigkeit}
Die experimentell bestimmten Schallgeschwindigkeiten beliefen sich auf:
\begin{align*}
    c_\text{Acryl} &= \qty{2744(14)}{\meter\per\second} \\
    c_\text{Aluminium} &= \qty{6213(1107)}{\meter\per\second}
\end{align*}
Die dazugehörigen Literaturwerte betragen:
\begin{align*}
    c_\text{Acryl,Lit} &= \qty{2730}{\meter\per\second} \\
    c_\text{Aluminium,Lit} &= \qty{6320}{\meter\per\second}
\end{align*}
\noindent Damit lassen sich die Abweichungen berechnen: $0.51 \%$ und $1.69 \%$. 
Das spricht für eine sehr gelungene Messung und eine erfolgreiche Durchführung. 
Trotz erwarter Abweichungen durch die Kopplung von zwei Zylindern sind die 
Abschweifungen zur Literatur vernachlässigbar gering.

\subsection{Dämpfung}
Die Dämpfungsfaktoren ließen sich mit 
\begin{align*}
    d_{2 \unit{\mega\hertz},angepasst} &= \qty{11.6(2.9)}{1 \per\meter} \\
    d_{1 \unit{\mega\hertz},angepasst} &= \qty{1.31(0.92)}{1 \per\meter} \\
    d_{2 \unit{\mega\hertz}} &= \qty{1.47 \pm 2.37}{1 \per\meter} \\
    d_{1 \unit{\mega\hertz}} &= \qty{-2.46 \pm 2.06}{1 \per\meter}
\end{align*}
beziffern. Die unterschiedlichen Werte lassen sich einerseits durch die Eliminierung 
von Messwerten begründen, andererseits durch Wechselwirkungseffekte der Sonde 
mit dem Medium. Bei einer geringeren Frequenz wechselwirkt die Sonde weniger, 
wodurch der Koeffiziente hier grundsätzlich geringer ausfällt. Ebenso ist der 
negative Wert von $d_{1 \unit{\mega\hertz}}$ nicht physikalisch sinnvoll, was 
auf eine sehr fehlerbehaftete Messung hinweist. Inwiefern die Werte allerdings 
realistisch sind, steht weiterhin aus, da keine Literaturwerte oder anderweitige 
Quellen zum Vergleich stehen.

\subsection{Kallibrierkurve}
Aufgrund der zulaufenden Form sollte eine quadratische Näherung stattfinden. 
Durch die Messungen kann auf einen misslungenen Versuch hingedeutet werden: 
Es ergibt sich eine Gerade. Das könnte auf eine schlechte Durchführung 
zurückzuführen sein. Möglicherweise wurden die Messabstände nicht korrekt eingehalten.
Hauptfehlerquelle wird sehr wahrscheinlich die ungünstige Peak-Detektion gewesen 
sein. Dadurch, dass die Apparatur ständig gewissen Unterschwankungen ausgesetzt 
war, schwankten auch die Peaks. Schlussendlich bleibt nichts von einem 
quadratischen Verlauf übrig und es kann von einer erfolglosen Messung gesprochen 
werden.