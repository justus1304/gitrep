\section{Auswertung}
\label{sec:Auswertung}

\subsection{Gauß'sche Fehlerfortpflanzung}
Da es in Experimenten zu unumgänglichen Messunsicherheiten kommt, ist eine 
Fehlerkalkulation obligatorisch. In der folgenden Auswertung wird eine 
Fehlerrechnung mithilfe der Gauß'schen Fehlerfortpflanzung durchgeführt, welche
lautet:
\begin{equation}
    \sigma_f = \sqrt{\sum\limits_{i=1}^N \left( \frac{\partial f}{\partial x_i} \sigma_f \right) ^2}.
\end{equation}

\section{Fehlerfortpflanzung}
Für die Fehlerfortpflanzung nach Gauß ergibt sich für die Linearisierung
$f_e(x)=Lx^2-\frac{x^3}{3}$ der einsitigen Einspannung dann folgender Term:
\begin{equation}
  \Delta f_e(x) = \sqrt{(2Lx-x^2)^2 (\Delta x)^2}.
\end{equation}
Des Weiteren ist die Fehlerfortpflanzung für die beidseitige Einspannung 
mit Linearisierung $f_b(x)=3L^2x-4x^3$ gegeben durch:
\begin{equation}
  \Delta f_b(x) = \sqrt{(3L^2-12x^2)^2 (\Delta x)^2}.
\end{equation}

%runder Stab-einseitige Einspannung plot und Messwerte
\subsection{runder Stab-einseitige Einspannung}
\begin{figure}
  \centering
  \includegraphics{plotrSeinseit.pdf}
  \caption{Einseitiger Aufhang-runder Stab.}
  \label{fig:plotrSeinseit}
\end{figure}

\begin{table}[H]
  \centering
  \caption{Messwerte x, D\textsubscript{0x}, D(x)}
  \label{tab:at}
  %\sisetup{table-format=1.1, per-mode=reciprocal}
  \begin{tblr}{
      colspec = {S S S},
      row{1} = {guard, mode=math},
      %vline{4} = {2}{-}{text=\clap{$\pm$}},
    }
    \toprule
    x(mm) & D\textsubscript{0x}(mm) & D(x)(mm)\\
    \midrule
    500 & 4.31 & 2    \\
    475 & 4.69 & 2.55 \\
    450 & 5.07 & 3    \\
    425 & 5.48 & 3.56 \\
    400 & 5.85 & 4.1  \\
    375 & 6.22 & 4.61 \\
    350 & 6.59 & 5.16 \\
    325 & 6.9  & 5.64 \\
    300 & 7.23 & 6.12 \\
    275 & 7.55 & 6.59 \\
    250 & 7.87 & 7.04 \\
    \midrule
    $\overline{x}$ & $\overline{D_{0 x}}$ & $\overline{D(x)}$\\
    \midrule
    1 \pm 2 & 1 \pm 2 & 1 \pm 2\\
    \bottomrule
  \end{tblr}
\end{table}

%runder Stab-beidseitige Einspannung plot und Messwerte
\subsection{runder Stab-beidseitige Bespannung}
\begin{figure}
  \centering
  \includegraphics{beidseitigLinks.pdf}
  \caption{Beidseitige Aufhängung-runder Stab.}
  \label{fig:beidseitigLinks}
\end{figure}

\begin{table}[H]
  \centering
  \caption{Messwerte a, D\textsubscript{0,l}, D\textsubscript{x,l}, D\textsubscript{0,r}, D\textsubscript{x,r}}
  \label{tab:at}
  %\sisetup{table-format=1.1, per-mode=reciprocal}
  \begin{tblr}{
      colspec = {S S S S S},
      row{1} = {guard, mode=math},
      %vline{4} = {2}{-}{text=\clap{$\pm$}},
    }
    \toprule
    a(mm), D\textsubscript{0,l}(mm), D\textsubscript{x,l}(mm), D\textsubscript{0,r}(mm), D\textsubscript{x,r}(mm)\\
    \midrule
    20  & 8.44 & 7.88 & 9.51  & 8.89  \\
    40  & 8.34 & 7.75 & 9.66  & 9.05  \\
    60  & 8.21 & 7.61 & 9.80  & 9.25  \\
    80  & 8.08 & 7.52 & 9.96  & 9.45  \\
    100 & 7.98 & 7.44 & 10.11 & 9.63  \\
    120 & 7.86 & 7.37 & 10.26 & 9.81  \\
    140 & 7.76 & 7.29 & 10.40 & 10.00 \\
    160 & 7.63 & 7.23 & 10.55 & 10.21 \\
    180 & 7.54 & 7.16 & 10.68 & 10.40 \\
    200 & 7.43 & 7.12 & 10.81 & 10.61 \\
    \midrule
    $\overline{a}$ & $\overline{D_{0,l}}$ & $\overline{D_{x,l}}$ & $\overline{D_{0,r}}$ & $\overline{D_{x,r}}$\\
    \midrule
    1 \pm 2 & 1 \pm 2 & 1 \pm 2 & 1 \pm 2 & 1 \pm 2\\
    \bottomrule
  \end{tblr}
\end{table}

%eckiger Stab-einseitige Einspannung plot und Messwerte
\subsection{eckiger Stab-einseitige Einspannung}
\begin{figure}
  \centering
  \includegraphics{ploteSeinseit.pdf}
  \caption{Einseitiger Aufhang-eckiger Stab.}
  \label{fig:ploteSeinseit}
\end{figure}

\begin{table}[H]
  \centering
  \caption{Messwerte x, D\textsubscript{0x}, D(x)}
  \label{tab:at}
  %\sisetup{table-format=1.1, per-mode=reciprocal}
  \begin{tblr}{
      colspec = {S S S},
      row{1} = {guard, mode=math},
      %vline{4} = {2}{-}{text=\clap{$\pm$}},
    }
    \toprule
    x(mm) & D\textsubscript{0x}(mm) & D(x)(mm)\\
    \midrule
    500 & 6.01 & 4.71 \\
    475 & 6.26 & 5.00 \\
    450 & 6.41 & 5.27 \\
    425 & 6.65 & 5.57 \\
    400 & 6.83 & 5.88 \\
    375 & 7.00 & 6.21 \\
    350 & 7.22 & 6.47 \\
    325 & 7.55 & 6.79 \\
    300 & 7.75 & 7.13 \\
    275 & 7.89 & 7.44 \\
    250 & 8.14 & 7.67 \\
    \midrule
    $\overline{x}$ & $\overline{D_{0 x}}$ & $\overline{D(x)}$\\
    \midrule
    1 \pm 2 & 1 \pm 2 & 1 \pm 2\\
    \bottomrule
  \end{tblr}
\end{table}

%eckiger Stab-beidseitige Einspannung plot und Messwerte
% \subsection{eckiger Stab-beidseitige Bespannung}
% \begin{figure}
%   \centering
%   \includegraphics{beidseitigLinks.pdf}
%   \caption{Einseitiger Aufhang-eckiger Stab.}
%   \label{fig:ploteSeinseit}
% \end{figure}

\begin{table}[H]
  \centering
  \caption{Messwerte a, D\textsubscript{0,l}, D\textsubscript{x,l}, D\textsubscript{0,r}, D\textsubscript{x,r}}
  \label{tab:at}
  %\sisetup{table-format=1.1, per-mode=reciprocal}
  \begin{tblr}{
      colspec = {S S S S S},
      row{1} = {guard, mode=math},
      %vline{4} = {2}{-}{text=\clap{$\pm$}},
    }
    \toprule
    a(mm), D\textsubscript{0,l}(mm), D\textsubscript{x,l}(mm), D\textsubscript{0,r}(mm), D\textsubscript{x,r}(mm)\\
    \midrule
    20  & 9.59 & 9.13 & 10.52 & 10.15 \\
    40  & 9.55 & 9.12 & 10.56 & 10.19 \\
    60  & 9.49 & 9.07 & 10.55 & 10.19 \\
    80  & 9.45 & 9.03 & 10.53 & 10.21 \\
    100 & 9.42 & 9.03 & 10.54 & 10.25 \\
    120 & 9.39 & 9.03 & 10.55 & 10.29 \\
    140 & 9.34 & 9.01 & 10.57 & 10.34 \\
    160 & 9.30 & 8.99 & 10.58 & 10.41 \\
    180 & 9.23 & 8.96 & 10.65 & 10.50 \\
    200 & 9.17 & 8.94 & 10.68 & 10.60 \\
    \midrule
    $\overline{a}$ & $\overline{D_{0,l}}$ & $\overline{D_{x,l}}$ & $\overline{D_{0,r}}$ & $\overline{D_{x,r}}$\\
    \midrule
    1 \pm 2 & 1 \pm 2 & 1 \pm 2 & 1 \pm 2 & 1 \pm 2\\
    \bottomrule
  \end{tblr}
\end{table}

%\begin{figure}
%  \centering
%  \includegraphics{"eckigbeidseit.pdf"}
%  \caption{Plot.}
%  \label{fig:plotbeidseitig}
%\end{figure}



