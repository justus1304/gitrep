% Hiermit sagst du LaTeX, dass es eine Beamer-Präsentation ist
\documentclass{beamer}
\usepackage{graphicx}
\usepackage{tikz} % Brauchst du für diese Methode
\usetikzlibrary{positioning} % Für einfache Positionierung (z.B. "right=of")

% --- HINTERGRUNDBILD (ROBUSTE TIKZ-METHODE) ---
%\setbeamertemplate{background canvas}{%
%  \begin{tikzpicture}[overlay, remember picture]
%    \node[opacity=.5, at=(current page.center)] {
%      \includegraphics[width=\paperwidth, height=\paperheight]{bg3.png}
%    };
%  \end{tikzpicture}
%}
% ---Wichtige Pakete für Umlaute und Schriftkodierung
\usepackage[utf8]{inputenc}
\usepackage[T1]{fontenc}
\usepackage[ngerman]{babel} % Für deutsche Silbentrennung und Begriffe
% Fügt 1em (ca. eine Zeilenhöhe) Abstand über/unter JEDER figure-Umgebung ein
\setbeamertemplate{figure}[text margin=1em]

% --- TIKZ-STILE DEFINIEREN (Für sauberen Code) ---
% Hier definieren wir, wie unsere Boxen und Pfeile aussehen sollen
\tikzset{
  % Stil für unsere Boxen
  mybox/.style={
    draw=structure!80,  % Randfarbe (nimmt die Theme-Farbe)
    fill=structure!10,  % Füllfarbe (10% der Theme-Farbe)
    rounded corners,      % Abgerundete Ecken
    minimum height=1.5cm,   % Mindesthöhe
    minimum width=4cm,    % Mindestbreite
    text centered,        % Text zentrieren
    thick                 % Dickerer Rand
  },
  % Stil für unsere Pfeile
  myarrow/.style={
    ->,                   % Pfeilspitze
    very thick,           % Dicker Pfeil
    color=structure!90,   % Pfeilfarbe
    shorten >=1pt,        % Pfeil stoppt 1pt vor der Box
    shorten <=1pt         % Pfeil startet 1pt nach der Box
  }
}

% --- Titel-Informationen ---
\title{Wasserkraft}
\author{Justus Weber}
\date{\today} % Setzt das heutige Datum

% Füge dies nach den \usepackage-Befehlen ein
\usetheme{metropolis} % Ein sehr beliebtes Theme mit Navigationsleiste
\logo{\includegraphics[height=1mm]{tud_logo_rgb.jpg}}
\setbeamercolor{structure}{fg=green!80!black}
%\usecolortheme{beaver} % Ändert die Farben (meist rot/braun)
% 1. Ebene (standard: Dreieck) wird zu einem Gedankenstrich
%\setbeamertemplate{itemize item}{\large $\circ$} 

% 2. Ebene (standard: kleineres Dreieck) wird zu einem Kreis
%\setbeamertemplate{itemize subitem}{$\circ$}

% 3. Ebene (standard: Quadrat) wird zu einem Stern
%\setbeamertemplate{itemize subsubitem}{$\star$}



\begin{document}

% --- 1. Die Titelfolie generieren ---
% Die \titlepage nutzt die Infos aus der Präambel
\begin{frame}
    \titlepage
\end{frame}

% --- 2. Eine Inhaltsverzeichnis-Folie (optional) ---
\begin{frame}
    \frametitle{Übersicht} % Titel für diese Folie
    \tableofcontents % Erstellt automatisch ein Inhaltsverzeichnis
\end{frame}

% --- 3. Eine normale Inhaltsfolie ---
\section{Einleitung} % Fügt einen Eintrag zum Inhaltsverzeichnis hinzu

\begin{frame}
  \frametitle{Das Grundprinzip der Wasserkraft}
  Hier ist das Prinzip:

  \begin{itemize}
        \item Umwandlung von potenzieller in Elektrische Energie 
        \item Höher gelagertes Wasser $\rightarrow$ Fluss in nierdigere Lagen 
        \item Kinetische Energie des Wassers in Elektrische Energie umwandeln 
    \end{itemize}
    \vfill
    %\begin{figure}
        
        \includegraphics[width=0.5\textwidth]{bilder/grundaufbau.png}
        
    %\end{figure}
    \vfill
\end{frame}

% --- 4. Eine weitere Folie ---
\section{Hauptteil} % Nächster Abschnitt

\begin{frame}
    \frametitle{Warum LaTeX?}
    
    Weil es für Formeln $E = mc^2$ und Textsatz unschlagbar ist.
\end{frame}

\end{document}