\section{Auswertung}
\label{sec:Auswertung}

\subsection{Fourier-Synthese}


\subsection{Fourier-Analyse}
Bei der Fourier-Analyse werden am Oberwellengeneratur die Rechtecksfunktion,
die Dreiecksfunktion und die Sägezahnfunktion generiert. Am Amperemeter werden 
die Amplituden bei den jewailigen Frequenzen der Oberwelle abgelesen, um so
auf die Abhängigkeit der Spannung der $n$-ten Oberwelle von $n$ schließen zu
können.
\subsubsection{Rechteckspannung}
\begin{table}[H]
    \centering
    \caption{Amplituden der Oberschwingungen Rechtecksfunktion.}
    \label{tab:j1}
    \begin{tblr}{
        colspec = {S S S},
        row{1} = {guard, mode=math},
      }
    \toprule
    f (\unit{\hertz}) &  A (\unit{\deci\bel})\\
    \midrule
    10 & 17.8\\
    30  & 9.01\\
    50  & 4.61\\
    70  & 1.81\\
    90  &-0.19\\
    110 &-1.79\\
    130 &-3.39\\
    150 &-4.59\\
    170 &-6.19\\
    190 &-8.19\\
    \bottomrule
    \end{tblr}
\end{table}

\subsubsection{Dreieckspannung}
\begin{table}[H]
    \centering
    \caption{Amplituden der Oberschwingungen Dreiecksfunktion.}
    \label{tab:j1}
    \begin{tblr}{
        colspec = {S S S},
        row{1} = {guard, mode=math},
      }
    \toprule
    f (\unit{\hertz}) &  A (\unit{\deci\bel})\\
    \midrule
    10  & 14.2  \\
    30  & -4.59 \\
    50  & -13.4 \\
    70  & -19.4 \\
    90  & -23.4 \\
    110 & -27.0 \\
    130 & -29.4 \\
    \bottomrule
    \end{tblr}
\end{table}

\subsubsection{Sägezahnspannung}
\begin{table}[H]
    \centering
    \caption{Amplituden der Oberschwingungen Sägezahnfunktion.}
    \label{tab:j1}
    \begin{tblr}{
        colspec = {S S S},
        row{1} = {guard, mode=math},
      }
    \toprule
    f (\unit{\hertz}) &  A (\unit{\deci\bel})\\
    \midrule
    10  & 11.8  \\
    20  &  7.01 \\
    30  &  3.01 \\
    40  &  1.01 \\
    50  & -1.39 \\
    60  & -2.99 \\
    70  & -4.00 \\
    80  & -5.79 \\
    90  & -6.19 \\
    100 & -8.99 \\
    110 & -7.79 \\
    120 & -9.79 \\
    130 & -9.39 \\
    \bottomrule
    \end{tblr}
\end{table}

%Auskommentiert, da es irgendwie Probleme mit "\includegraphics{rechteck.pdf}"
%gab (bereits bei make nachdem ich direkt gepullt habe)
% \begin{figure}
%     \centering
%     \caption{}
%     \includegraphics{rechteck.pdf}
% \end{figure}

\subsubsection{Dreiecksfunktion}
\begin{figure}
    \centering
    \caption{}
    \includegraphics{dreieck.pdf}
\end{figure}

\subsubsection{Sägezahnfunktion}
\begin{figure}
    \centering
    \caption{}
    \includegraphics{saegezahn.pdf}
\end{figure}


