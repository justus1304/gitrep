\section{Theorie}
\label{sec:Theorie}

Die Messung von durch Widerstände ausdrückbare Größen kann durch Brückenschaltungen 
möglich gemacht werden. In diesem Versuch geht es demnach um ohmsche Widerstände
und komplexe Widerstände, wie es bei Kondensatoren und Spulen der Fall ist.
Um das Prinzip von Brückenschaltungen näherzubringen, ist eine in \autoref{fig:f1}
dargestellt.
\begin{figure}[H]
    \centering
        \centering
        \includegraphics[width=0.35\textwidth]{Bilder/Brueckenschaltung.png}
        \caption{Allgemeine Brückenschaltung. \cite{anleitung}}
    \hfill
    \label{fig:f1}
\end{figure}
\noindent Zwischen den Punkten A und B, welche wiederum zwischen den vier Widerständen 
$R1$, $R2$, $R3$ und $R4$ liegen besteht eine Verbindung über ein Messgerät. 
Diese Verbindung wird als Brücke bezeichnet. Um Fehlströme zu vermeiden und damit 
die Messung nicht fälschlicherweise beeinflusst werden kann, wird eine erdungsfreie 
Speisespannung genutzt. Es gelten die Kirchhoffschen Gesetze:
\begin{enumerate}
    \item Die Summe aller Ströme, die an einem Knoten ein- und ausgehen, ist Null.
    \begin{equation}
        \sum\limits_{n}^N I_n = 0
    \end{equation}
    \item Die Summe aller Spannungen in einem abgeschlossenen Stromkreis ist Null 
    (unter Berücksichtigung der Vorzeichen).
    \begin{equation}
        \sum\limits_{n}^N U_n = 0
    \end{equation}
\end{enumerate}
\par\vspace{0.5em}
\noindent Die Punkte A und B befinden sich im Falle des Nullabgleichs (Das 
Messgerät dokumentiert keinen Strom, I=0) auf gleichem Potential.
Es kann eine Abgleichbedingung aufgestellt werden:
\begin{equation}
    R_1 R_4 = R_2 R_3
\end{equation}
Das gleiche gilt analog für komplexe Widerstände. Ist einer der Widerstände 
von unbekannter Größe, kann er trivialerweise mithilfe der anderen Komponenten 
bestimmt werden. Jenes wird als Kompensationsmethode bezeichnet und soll im 
Folgenden gezeigt werden.

\subsection{Fehlerrechnung}
Die gemessenen Werte unterliegen Messunsicherheiten und werden demnach im
Folgenden nicht als fehlerfrei angesehen. Die Fehler entstehen bei der
Bildung der Mittelwerte durch den Fehler des Mittelwerts und bei der
Regressionsrechnung sowie der Fehlerforpflanzung durch Python.
Der Fehler des Mittelwerts ist gegeben durch 
\begin{equation}
    \begin{aligned}
        \Delta \overline{x} &= \sqrt{\overline{x^2} - \overline{x}^2} \\
                            &= \frac{\sqrt{\frac{1}{N-1} \sum\limits_{i=1}^N (x_i - \overline{x})^2}}{\sqrt{N}}.
    \end{aligned}
\end{equation}

Um Fehler einzubeziehen, wird die Gauß'sche Fehlerfortpflanzung verwendet:
\begin{equation}
    \label{eqn:8}
    \Delta f = \sqrt{\left(\frac{\partial f}{\partial x}\right)^2 \cdot \left(\Delta x\right)^2 + \left(\frac{\partial f}{\partial y}\right)^2 \cdot \left(\Delta y\right)^2 + .... + \left(\frac{\partial f}{\partial z}\right)^2 \cdot \left(\Delta z\right)^2}
\end{equation}