\section{Diskussion}
\label{sec:Diskussion}

Als Hauptfehlerquellen sind einige Aspekte des Aufbaus und der Geräte heranzuziehen.
Es ist anzunehmen, dass der Polarisationsfilter nicht rein parallel- oder 
senkrecht polarisiertes Licht filtern kann, das soll im Folgenden noch weiter 
thematisiert werden. Weiterhin ist die Ausrichtung von Laser und Spiegel inklusive 
Intensitätsmesser anzumängeln. Zum einen war es mühsam, diese einigermaßen präzise 
aufzubauen, die Ausrichtung erfolgt nach Augenmaß. Darüber hinaus lässt sich 
vermuten, dass das Luxmeter fehlerbehaftet ist, so ergaben sich 
Intensitätsveränderungen bei Kontakt der Apparatur, was nicht der Fall sein 
durfte.

\subsection{Brechungsindizes}
Aus den Berechnungen ergaben sich die Brechungsindizes
\begin{align*}
    \overline{n_\text{parallel}}  &= \qty{2.50(0.32)}{} \\
    \overline{n_\text{senkrecht}} &= \qty{1.48(0.05)}{}
\end{align*}
\noindent Es ist nicht möglich Vergleichswerte heranzuziehen, jedoch sind diese 
Werte als fehlerbehaftet anzunehmen. Die signifikante Abweichung zwischen den
beiden Werten ($\increment n = 1.02$) deutet auf eine starke Doppelbrechung des
Materials hin. Die vergleichsweise große Unsicherheit bei $n_\text{parallel}$
von $\pm 0.32$ lässt sich auf mehrere Faktoren zurückführen, wie die bereits 
angesprochenen Polarisationsfilterprobleme oder die manuelle Ausrichtung von 
Instrumenten, was zu was zu einer Vermischung der Polarisationsebenen führt.
Dies äußert sich besonders stark bei der Messung der parallelen Komponente.

\subsection{Brewsterwinkel}