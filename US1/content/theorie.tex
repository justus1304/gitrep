\section{Theorie}
\label{sec:Theorie}

Unter Schall versteht sich eine, durch Druckschwankungen fortbewegende, 
longitudinale Welle.
In dem Versuch relevanten Schall handelt es sich um Ultraschall, dieser liegt
im Bereich von $20 \unit{\kilo\hertz} - 1 \unit{\giga\hertz}$.
Ultraschallwellen verfügen über ähnliche Eigenschaften wie elektromagnetische
Wellen und finden demnach breite Anwendung in der Medizin und der
zerstörungsfreien Werkstoffprüfung. Die Phasengeschwindigkeit ist materialabhängig,
was im späteren Verlauf (siehe \autoref{sec:Auswertung}) auch noch untersucht
werden soll. Grund dafür sind Druck- sowie Dichteänderungen.
Eine Ultraschallwelle (inklusive akustischer Impedanz von $Z=c \rho$) hat die 
Form
\begin{equation}
    \label{eqn:1}
    p(x,t) = p_0 + v_0 c \rho \cos(\omega t - kx).
\end{equation}
Bei einer Schallwelle, welche an einem Medium gebrochen wird, kommt es zu einem 
Reflexionsanteil, dessen Koeffizient bestimmt werden kann. Mithilfe der akustischen
Impedanzen beläuft sich dieser auf
\begin{equation}
    R = \left( \frac{Z_1 - Z_2}{Z_1 + Z_2} \right)^2.
\end{equation}
Für die unterschiedlichen Schallgeschwindigkeiten $c_{Fl}$ in einer Flüssigkeit 
und $c_{Fe}$ in einem Festkörper sind die Zusammenhänge gegeben:
\begin{align}
    \label{eqn:2}
    c_{Fl} &= \sqrt{\frac{1}{\kappa \cdot \rho}}, & c_{Fe} &= \sqrt{\frac{E}{\rho}}
\end{align}
\noindent Dabei seien $\kappa$ die Kompressibilität und $E$ die das 
Elastizitätsmodul. Zusätzlich geht bei der Schallausbreitung Energie durch 
Absorbtion verloren, die Intensität nimmt exponentiell mit der Strecke ab.
Da Luft bereits stark absorbiert, wird als Präventativmaßnahme ein 
Kontaktmittel verwendet.
\vspace{0.5em}
\\
\noindent Ultraschall wird häufig mithilfe des piezoelektrischen Effekts erzeugt.
Piezokristalle wie Quarz schwingen in einem elektrischen Wechselfeld und
emittieren bei Resonanzfrequenz intensive Ultraschallwellen. Umgekehrt können
diese Kristalle auch als Empfänger dienen, indem sie eintreffende Schallwellen
in elektrische Signale umwandeln.
\vspace{0.5em}
\\
\noindent In der Medizin wird Ultraschall, etwa durch Laufzeitmessungen, zur Diagnostik
genutzt. Bei dem Durchschallungs-Verfahren gibt es einen Ultraschallsender,
der das Gewebe durchdringt. Trifft das Wellensignal auf eine Fehlstelle $s$, so 
ist ein Echo gesehen. Dieses wird an den Empfänger gesendet. Die Schallgeschwindigkeit 
an der Fehlstelle kann als
\begin{equation}
    \label{eqn:3}
    c = \frac{2s}{t}
\end{equation}
beschrieben werden.
Bei dem Impuls-Echo-Verfahren werden Reflexionen an der Grenzfläche gemessen, hier 
ist der Sender gleich dem Empfänger. Beide Methoden liefern Informationen über
die Struktur und Beschaffenheit des untersuchten Materials, ohne es zu schädigen.
In diesem Versuch soll letzteres verwendet werden, dabei wird erneut zwischen 
drei Arten von Scans unterschieden.
\vspace{0.5em}
\\
\noindent Der A-Scan (Amplituden-Scan) wird genutzt um ein Tiefenprofil des 
Materials zu erstellen. Er stellt die Echoamplituden als Funktion der Laufzeit
der.
\vspace{0.5em}
\\
\noindent Der B-Scan (Brightness-Scan) stellt die Messdaten als 2-dimensionales 
Bild dar, wobei die höheren Amplituden als helle Pixel und die tieferen als 
dunklere Pixel dargestellt werden. Für ein Schnittbild wird die Sonde also 
über die Probe bewegt.
\vspace{0.5em}
\\
\noindent Der TM-Scan (Time-Motion-Scan) stellt Bewegungen als Bildfolge in 
Abhängigkeit der Zeit dar. Demnach können beispielsweise Organregungen 
detektiert werden.
\vspace{0.5em}
\\
In diesem Versuch wird jedoch nur der A-Scan von Relevanz sein.

\subsection{Fehlerrechnung}
Die gemessenen Werte unterliegen Messunsicherheiten und werden demnach im
Folgenden nicht als fehlerfrei angesehen. Die Fehler entstehen bei der
Bildung der Mittelwerte durch den Fehler des Mittelwerts und bei der
Regressionsrechnung sowie der Fehlerforpflanzung durch Python.
Der Fehler des Mittelwerts ist gegeben durch 
\begin{equation}
    \begin{aligned}
        \increment \overline{x} &= \sqrt{\overline{x^2\kern-0.1em} - \overline{x}^2} \\
                            &= \frac{\sqrt{\frac{1}{N-1} \sum\limits_{i=1}^N (x_i - \overline{x})^2}}{\sqrt{N}}.
    \end{aligned}
\end{equation}

Um Fehler einzubeziehen, wird die Gauß'sche Fehlerfortpflanzung verwendet:
\begin{equation}
    \label{eqn:9}
    \increment f = \sqrt{\left(\frac{\partial f}{\partial x}\right)^2 \cdot \left(\increment x\right)^2 + \left(\frac{\partial f}{\partial y}\right)^2 \cdot \left(\increment y\right)^2 + .... + \left(\frac{\partial f}{\partial z}\right)^2 \cdot \left(\increment z\right)^2}
\end{equation}