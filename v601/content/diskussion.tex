\section{Diskussion}
\label{sec:Diskussion}
\subsection{Mittlere freie Weglänge}
Im Ersten unterabschnitt der Auswertung \autoref{sec:mfw} kommt man zu dem schluss, dass die Franck Hertz Kurven bei 
niedrigen Temperaturen bis ca. 150 Grad Celsius nicht entstehen. Dies ist auch logisch, da Dort der Dampfdruck im Gas 
noch nicht so hoch ist und somit weniger Teilchen in der Bahn des Elektrons liegen, wesshalb 
es nicht so viele Stöße gibt. Dies zeigt sich in der Hohen Mittleren freien Weglänge des Elektrons im Niedrigtempratur-Bereich.

\subsection{Franck-Hertz Kurven}
Als Literaturwert für die Anregungsenergie von $4.9 \unit{e\volt}$ angegeben. Daraus würde eine Mittlere Differenz der Maxima
 von $4.9 \unit{\volt}$ gellten. Zu unserem berechneten Wert von $5.5 \unit{\volt}$ ergibt sich eine Abweichung von $11 \%"$ . Dies Kann mehrere Ursachen haben.
 Zum einen die Ungenauigkeit durch Ablesen der Steigung von Millimeterpapier sowie die Unsicherheit durch den x-y Schreiber und als größte 
 Fehlerquelle kann die nicht fest zu regulierende Temperatur des Quecksilbergases herangezogen werden.