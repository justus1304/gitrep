\section{Diskussion}
\label{sec:Diskussion}

\subsection{Magnetfeld einer langen und kurzen Spule}
Für das Magnetfeld von zwei Spulen unterschiedlicher Länge lässt sich zunächst 
festhalten, dass sich das Ausrichten der longitudinalen Sonde mit dem Stativ
nicht besonders präzise bewerkstelligen lässt. Einerseits sitzt die Sonde selbst 
nicht fest auf dem als Metermaß dienenden Lineal und kann leicht ausgelenkt werden, 
was bei der Messung große Unterschiede machen kann, da der Messer empfindlich ist.
Darüber hinaus erfolgen jegliche Justierungen unter Augenmaß. Demnach ist es 
höchstunwahrscheinlich, dass sich die Hall-Sonde bei jeder Messung genau 
zentral in der Spule befand. Dieses Problem vergrößert sich noch weiter bei der 
langen Spule, da bei dieser die Länge der Hall-Sonde nicht ausreicht und beidseitig 
eingeführt werden muss. Bei Drehungen von Instrumenten sind Abweichungen unvermeidbar.
\par\vspace{0.5em}
Trotzdem werden mit den gemessenen Werten (für die 
kurze Spule) eine Flussdichte mit $1.63 \pm 0.18 \unit{\milli\tesla}$ bestimmt,
welche dem theoretischen Wert mit $1.57 \unit{\milli\tesla}$ sehr nahe kommen.
Das entspricht einer Abweichung von 3.75 \%.
Auch der Graph passt zu theoretischen Überlegungen; Zu erkennen ist ein klares 
Maximum, was eine Homogenität in der Mitte ausschließt.
\par\vspace{0.5em}
Bei der langen Spule ergibt sich für die experimentell bestimmte Flussdichte 
ein Wert von $1.90 \pm 0.13 \unit{\milli\tesla}$ und weicht damit nur um 4.04 \%
vom theoretischen Wert mit $1.98 \unit{\milli\tesla}$ ab. Ein verhältnismäßig 
geringer Fehler, jedoch zu erwarten. Vor Allem unter Einbezug der Tatsache, dass 
die Justierung der Sonde nach Drehung der Spule erwartend unpräzise erfolgt. 
Jenes ist auch dem Graphen zu der magnetischen Feldstärke zu entnehmen. Nach 
Drehung und erneuter Einführung in die Spule kommt es zu einem Sprung der 
Messwerte, sodass ab 20 cm eine erhöhte Flussdichte detektiert wird. Weiterhin 
ist am Graphen abzulesen, dass das Feld im Inneren homogen scheint, zumal der 
Graph im x-Bereich zwischen 10 und 20 cm annähernd linear verläuft. Abweichungen 
sind durch systematische Fehler zu begründen.

\subsection{Hysteresekurve}
Die gemessenen Punkte, welche in \autoref{fig:3} vorliegen, benötigen keinen 
Hilfsgraphen um die Hysteresekurve zu erkennen. Die charakteristischen Punkte 
Remananz $B_r$, Koerzitivkraft $H_c$ und Sättigungswert $B_S$ sind auch deutlich 
abzulesen, ebenso die Neukurve. Für diese ist als Fehlerquelle bei minimalen 
Abweichungen die Umpolung heranzuziehen, je langsamer diese stattfand, desto 
eher waren Abweichungen zu erwarten. Lediglich ein Wert am Ende der Messungen
entfällt der Norm was vermutlich auf einen Messfehler zurückzuführen ist. Für
keinen der Graphen hat dieser jedoch einen Einfluss und kann daher vernachlässigt
werden. Die experimentell bestimmten Werte für die Remananz ergaben
\begin{align*}
    B_{R1} =& 122 \unit{\milli\tesla}\\
    B_{R2} =& -117 \unit{\milli\tesla}\\
\end{align*}
und 
\begin{align*}
    H_{S1} =& 540 \unit{\mega\ampere\per\meter}\\
    H_{S2} =& -540 \unit{\mega\ampere\per\meter}\\
\end{align*}
für den Sättigungswert. Letztlich ergab sich für die Koerzitivkraft.
\begin{align*}
    H_{K} =& 0.344 \unit{\kilo\ampere\per\meter}\\
    H_{K} =& 0.324 \unit{\kilo\ampere\per\meter}\\
\end{align*}
Als differentielle Permeabilität ergab sich ein Faktor von $\mu_{diff}=1.223$,
was ein sinniges Ergebnis ist, da die Spule sowohl einen Eisenkern als auch
einen Luftspalt enthält. Jenes verringert die Gesamtpermeabilität.

\subsection{Helmholtz-Spule}
Die beiden Graphen zur Helmholtz-Spule entsprechen näherungsweise der theoretischen 
Vorhersage, wenn es auch eher in \autoref{fig:10} danach aussieht. Der Tiefpunkt 
liegt bei der gemessenen Magnetfeldstärke bei x = 0 cm, wie die Theoriekurve 
bestätigt. Das liegt daran, dass sich die Hall-Sonde dann genau zwischen dem 
Spulenpaar befindet. Einige Abweichungen sind dennoch erkennbar, allerdings 
auch unvermeidbar, Richtung Ende des Experiments liegen die gemessenen Größen 
nahezu exakt auf der Theoriekurve.
\par\vspace{0.5em}
Auch die zweite Kurve, also bei einem Abstand von 8 cm beider Spulen unterscheiden 
sich die gemessenen Werte nicht allzu stark von der Theoriekurve. Die einzige 
Abweichung findet bei den ersten 15 Werten statt, diese liegen um 0.2 
\unit{\milli\tesla} über dem Theoriewert des entsprechenden Abstands. Restliche 
Werte nähern sich der Theoriekurve plausibel nahe an. Die ungleiche Verteilung 
an Messwerten ist mit dem geringen Abstand der Spulen verbunden. Aufgrund dieser 
Tatsache erlangen die gemessenen Werte keine erkennbar ähnliche Gestalt zur 
Theoriekurve (auf den ersten Blick). Bei anderen Werten mit einem besseren 
Abstand wäre eine Ähnlichkeit offensichtlicher gewesen.