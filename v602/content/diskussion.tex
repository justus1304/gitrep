\section{Diskussion}
\label{sec:Diskussion}

\subsection{Bragg-Bedingung}
Zunächst ist festzuhalten, dass der auffällige plötzliche Fall der geplotteten 
Daten auf einen systematischen Fehler zurückzuführen ist. Bei einer Detektion 
der Intensität von Null handelt es sich um einen Fehler des Messgeräts, welcher 
für die Bestimmung des Bragg-Winkels ignoriert werden kann. Der im Experiment
gemessene Beugungswinkel von 28° entspricht exakt dem Doppelten des eingestellten
Einfallswinkels (14°) zur Netzebene. Dies steht in Einklang mit der Bragg-Theorie,
nach der der Beugungswinkel $2 \theta$ beträgt, wenn $\theta$ der Winkel zwischen
einfallendem Strahl und den Kristallebenen ist. Da hier $\theta$ = 14° sowohl
als Einfallswinkel (zur Netzebene) gewählt wurde als auch aus dem Beugungswinkel 
folgt, bestätigt dies die Gültigkeit der Bragg-Bedingung für diese Messung.

