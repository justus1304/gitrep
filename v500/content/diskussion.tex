\section{Diskussion}
\label{sec:Diskussion}

\subsection{Kennlinien der Lichtspektren}
Um über die Kennlinien des violetten Lichts zu urteilen bedarf es der 
Betrachtung der Fermi-Dirac-Statistik, welche die Besetzungswahrscheinlichkeit 
von Quantenzuständen durch Fermionen bei einer Temperatur $T$ beschreibt.
Wird der Theorie gefolgt, so nimmt die Wahrscheinlichkeit für Fermionen, sich
im Metallgitter zu befinden mit abnehmender Energie bei $T>0$ zu. Wird die 
Fermi-Energie erreicht, stellt sich eine Sättigung ein. Diese ist auch in 
\autoref{fig:10} zu erkennen. Der Photostrom steigt zunächst exponentiell, 
bis er sich nach und nach zu einem ca. linearen Graphen abflacht.
\vspace{0.5em}
\\
Für den Graphen mit halber Intensität gilt das gleiche; der Graph verhält sich 
entsprechend der Fermi-Dirac Prognose. Der Graph ist lediglich gestaucht, da 
die Intensität nicht mehr so hoch ist. Es lässt sich zusammenschließen, dass 
die Anzahl der austretenden Elektronen proportional zur Intensität des Lichts 
ist.

\subsection{Grenzspannungen}
Für die gemessenen Grenzspannungen ergaben sich die folgenden Werte:
\begin{align*}
    U_\text{g,regr,violett} &= \qty{1.175(0.01)}{\volt} \\
    U_\text{g,regr,grün}   &= \qty{1.175(0.01)}{\volt} \\
    U_\text{g,regr,türkis} &= \qty{-1.0(0.12)}{\volt}  \\
    U_\text{g,regr,gelb}    &= \qty{-0.55(0.05)}{\volt} \\
\end{align*}
Es sind keine Theoriewerte oder Literaturwerte zum Vergleich vorhanden, alledings 
ist durch \autoref{fig:aa} erkennbar, dass die Grenzspannungen fehlerbehaftet
sind und somit von der Ausgleichsrechnung abweichen.

\subsection{Planck'sches Wirkungsquantum}
Der gemessene Wert des Planck'schen Wirkungsquantum aus \autoref{sec:planckWa}
beläuft sich auf
\begin{equation*}
    h_{exp} = \qty{6.3(1.1)e-34}{\joule\second},
\end{equation*}
\noindent wohingegen der Wert aus der Literatur als 
\begin{equation*}
    h_{Lit} = \SI{6.626e-34}{\joule\second}
\end{equation*}
bekannt ist. Das enstpricht einer Abweichung von $4.92 \%$. Als Fehlerquelle
könnte die Regression herangezogen werden. Hier handelt es sich um eine
Ausgleichsrechnung aus vier Werten, welche (wie in \autoref{fig:aa} zu sehen)
fehlerbehaftet sind. Folglich ist auch die experimentell bestimmte Planck'sche 
Konstante mit Fehlern versehen. Diese sind jedoch sehr gering, der Literaturwert
liegt im Fehlerbereich des experimentell bestimmten Werts. Es kann also von
einer erfolgreichen Messung gesprochen werden.

\subsection{Austrittsarbeit}
Der Anodenring der Photozelle besteht aus einer Platin-Rhodium Legierung, 
demnach wird die gemessene Austrittsarbeit mit den Literaturwerten beider 
Elemente verglichen, diese belaufen sich auf:
\begin{align*}
    W_{A,\text{Pt}} &= \SIrange{5.32}{5.66}{\eV}, \quad \text{\cite{pt}} \\
    W_{A,\text{Rh}} &= \SI{4.58}{\eV} \quad \text{\cite{rh}}.
\end{align*}
Verglichen mit dem gemessenen Wert von 
\begin{align*}
    W_\text{A,exp} =& \qty{1.5(0.4)}{\eV}
\end{align*}
wäre das eine Abweichung von mindestens $67.25 \%$. Das könnte durch eine 
verunreinigte Metallschicht kommen. Andererseits können hier nur Vermutungen 
angestellt werden zumal es sich einerseits um eine Legierung handelt und 
anderersetis kein direkter Vergleich vorliegt, sondern nur der mit dein einzelnen 
Elementen selbst. Letztendlich lässt sich jedoch vermuten, dass ein vergleichsweise 
großer Fehler vorliegt, da die lineare Regression aus lediglich vier Werten 
besteht, welche selbst wiederum einem Fehler unterliegen.