\section{Diskussion}
\label{sec:Diskussion}

Zunächst ist festzuhalten, dass jegliche bestimmte Größen nicht mit Literaturwerten 
zu vergleichen sind. Die tatsächlichen Werte der Widerstände sind unbekannt.
Jedoch liegen die Werte der Mittelwertbildung größtenteils verhältnismäßig nah
beieinander, was auf eine gute Genauigkeit bei den Messungen hindeutet. 
Lediglich bei der Berechnung der Induktivität des "Wert16" wichen die berechneten
Werte des Widerstands $R_x$ mit $R_{x1} = 464, R_{x2} = 422$ und $R_{x3} = 394$ 
entsprechend stark voneinander ab.
Es ist anzunehmen, dass diese Abweichung durch eine Ungenauigkeit am Widerstand 
zurückzuführen ist, da auch bei mehrfacher Durchführung des Versuches mit dem 
erwähnten Widerstand große Abweichungen zustande kamen.
\par\vspace{0.5em}
\noindent Für die Bauelemente wurde bei Wiederständen, Kapazitäten,
sowie Induktivitäten jeweils ein Fehler von $2 \%$ angenommen. Dieser Fehler
wurde auf Grundlage vorhergeganger Berechnungen abgeschätzt, sodass dieser
annähernd mit den Abweichungen zusammenpasst.

\noindent Für die unbekannten Widerstände ergab sich bei der Wheatstone Brücke:
\begin{align*}
    R_{x,11} &= \qty{629(13.0)}{\ohm}\\
    R_{x,14} &= \qty{901(18.0)}{\ohm}\\
    R_{x,12} &= \qty{391(8.0)}{\ohm}
\end{align*}
Bei der Kapazitätsmessbrücke ergaben sich die unbekannten Größen zu
\begin{align*}
    R_{x,8}  &= \qty{865(19.0)}{\ohm}\\
    R_{x,15} &= \qty{469(9.0)}{\ohm}\\
    R_{x,9}  &= \qty{461(9.0)}{\ohm}
\end{align*}
und
\begin{align*}
    C_{x,8}  &= \qty{294(6.0)}{\nano\farad}\\
    C_{x,15} &= \qty{639(13.0)}{\nano\farad}\\
    C_{x,9}  &= \qty{435(9.0)}{\nano\farad}.
\end{align*}
Bei der Induktivitätsmessbrücke ergaben sich die unbekannten Größen zu 
\begin{align*}
    R_{x,19} &= \qty{104.3(2.1)}{\ohm}\\
    R_{x,16} &= \qty{427(9.0)}{\ohm}\\
    R_{x,18} &= \qty{349(7.0)}{\ohm}
\end{align*}
und
\begin{align*}
    L_{x,19} &= \qty{26.9(0.5)}{\milli\henry}\\
    L_{x,16} &= \qty{134.2(2.7)}{\milli\henry}\\
    L_{x,18} &= \qty{49.9(1.0)}{\milli\henry}.
\end{align*}
Die Fehlertoleranz liegt bei allen gemessenen Größen bei etwa $2 \%$, was zu 
dem Fehler der Geräte passt, in dieser Hinsicht kann der Versuch als gelungen 
interpretiert werden.
\par\vspace{0.5em}
\noindent Der gemessene Klirrfaktor liegt bei $3,57 \%$, demnach ist das Signal
dezent verzerrt. Hauptursache für diesen hohen Wert ergibt sich aus der Berechnung: 
Der Klirrfaktor ergibt sich unter der Annahme, dass die zweite Oberwelle den 
größten Beitrag liefert, die restlichen Wellen vernachlässigt. Jenes impliziert, 
dass die Signalverzerrung größtenteils durch diese Hauptwelle entsteht, trotzdem 
könnten gegebenenfalls Oberwellen zu diesem Werte geführt haben.
\par\vspace{0.5em}
\noindent Grundsätzlich kann der Versuch als gelungen gewertet werden, äußerliche 
Bedingungen wie thermodynamische Aspekte haben keinen Einfluss auf den Versuch.
Das spiegelt sich im Ergebnis wieder, die gemessenen Fehler passen zueinander 
und die Messwerte liegen verhältnismäßig in der jeweiligen Größenordnung.