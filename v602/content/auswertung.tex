\section{Auswertung}
\label{sec:Auswertung}

\subsection{Verifizierung der Bragg-Bedingung.}
\begin{figure}
    \centering
    \caption{Messwerte zur Bragg-Überprüfung}
    \label{fig:brag}
    \includegraphics{build/bragg.pdf}
\end{figure}
Anhand von \autoref{fig:brag} lässt sich das Maximum ziemlich genau bei 
$\theta = 28\unit{\degree}$ erkennen. Da das Medium, an welchen 
die Photonen reflektiert werden in einem Winkel von $14\unit{\degree}$ stand, 
ist das Maximum bei dem Winkel erwartbar. (Einfallswinkel = Ausfallswinkel).

\subsection{Das Emissionsspektrum einer Cu-Roentgenroehre}
In \autoref{fig:spektrum} sind die Messwerte geplotet. Eingezeichnet sind auch der Bremsbergm sowie die $K_\alpha $ und $K_\beta $-Linien.
Der Grenzwinkel bestimmt sich durch ablesen zu 
\begin{equation}
    \Theta_\text{G} = \qty{5.7(0.1)}{\degree}
\end{equation}
daraus ergibt sich nach \autoref{eqn:} und \autoref{eqn:}
eine minimale Wellenlänge, bzw eine maximale Energie ableiten,
 \begin{align}
    \lambda_\text{min} &= \qty{40(0.7)}{\pico\meter}\\
    E_\text{max} &= \qty{49.7(0.7)}{\kilo\electronvolt}
 \end{align}
\begin{figure}
    \centering
    \caption{Emmisionsspektrum einer Cu-Röhre.}
    \label{fig:spektrum}
    \includegraphics{build/wertejust.pdf}
\end{figure}
Um nun das Auflösevermögen der Apparatur zu bestimmen Werden die K-Linien noch einmal 
vergrößert in \autoref{fig:zoom} dargestellt. Des Weiteren wurden die Wargerechten linien 
auf der höhe der beiden Halbwertsbreiten der K-Linien eingezeichnet. Daraus ergeben sich jewails zwei 
Winkel aus den Schnittpunkten mit der $K_\alpha$ und $K_\beta$-Linie.
Für $K_\alpha$ ergibt sich $\Theta_{min,\alpha} = \qty{22(0.1)}{\degree}$ 
, $\Theta_{max,\alpha} = \qty{22.7(0.1)}{\degree}$ sowie für $K_\beta
$ $\Theta_{min,\beta} = \qty{19.8(0.1)}{\degree}$ und $\Theta_{max,\alpha} = \qty{20.4(0.1)}{\degree}$
Durch \autoref{} werden daraus erneut die Energien ermittelt. Das Auflösungsvermögen lässt sich dann 
durch folgende Gleichung bestimmen.Die Ergebnisse sind \autoref{tab:auf} zu entnehmen.

\begin{equation}
    A = \frac{E}{\Delta E}
\end{equation}
\begin{table}[H]
  \centering
  \caption{Messwerte violettes Farbspektrom halbe Intensität.}
  \label{tab:11}
  \begin{tblr}{
          colspec = {S S S S S S},
          row{1} = {guard, mode = math},
      }
      \toprule
      Linie&IE/\unit{\kilo\electronvolt}& \Delta E /\unit{\kilo\electronvolt}&A\\
      \midrule
      
      \bottomrule
  \end{tblr}
\end{table}
\begin{figure}
    \centering
    \caption{Zoom auf $K_{\alpha} $und $ K_{\beta}$ Lienien des Emissionsspektrums}
    \label{fig:zoom}
    \includegraphics{build/zoom.pdf}
\end{figure}