\section{Theorie}
\label{sec:Theorie}

Beim Durchlaufen von Materie verlieren $\alpha$-Teilchen Energie durch Anregung 
oder Dissoziation von Molekülen, also der Aufspaltung eines Moleküls. Dieser 
Energieverlust $\frac{-dE_{\alpha}}{dx}$ ist abhängig von der Energie der 
Strahlung und der Dichte des Materials. Je schneller ein $\alpha$-Teilchen ist, 
desto kürzer ist die Wechselwirkungszeit, folglich ist die Wahrscheinlichkeit 
für Wechselwirkungen gering und der Energieverlust ist minimal. Für große 
Energien kann die Bethe-Boch-Gleichung genutzt werden, sie gibt an, wie der 
Energieverlust aussieht.
\begin{equation}
    \label{eqn:1}
    -\frac{dE_\alpha}{dx} = \frac{z^2 e^4}{4 \pi \epsilon_0 m_e} \cdot 
                           \frac{n Z}{v^2} \cdot \ln\left(\frac{2 m_e v^2}{I}\right)
\end{equation}
Die Bestandteile sind die Ladung $z$, $v$ die Geschwindigkeit der $\alpha$-
Teilchen, $Z$ die Ordnungszahl, $n$ als die Teilchendichte und $I$ die 
Ionisierungsenergie des Targetgases. Nach dem Ladungsaustausch bei hinreichend
kleinen Energien kann die Reichweite $R$ bis zur vollkommenen Abbremsung durch 
Energieverlust mittels 
\begin{equation}
    \label{eqn:2}
    R = \int_0^{E_\alpha} \frac{dE_{\alpha}}{-\frac{dE_\alpha}{dx}}
\end{equation}
bestimmt werden. Wie bereits erwähnt nimmt die Wahrscheinlichkeit für 
Wechselwirkungen bei hohen Geschwindigkeiten ab. Im Umkehrzug sind Materie-
Wechselwirkungen in niedrigen Geschwindigkeitsbereichen demnach eher zu erwarten.
Die Bethe-Boch-Gleichung versagt, da sie die Ladungsänderungen vernachlässigt.
Als Lösung werden empirische Kurven genutzt, um die mittlere Reichweite (im 
$\unit{\milli\meter}$-Bereich)
\begin{equation}
    \label{eqn:3}
    R_m = 3.1 \cdot E_{\alpha}^{\frac{3}{2}}
\end{equation}
zu bestimmen. Die Energie der $\alpha$-Strahlung ist dabei $E_{\alpha} \leq 
2.5 \unit{\mega\eV}$. Es wird also die effektive Weglänge gegen die Anzahl der 
Pulse aufgetragen, um die mittlere Reichweite der Strahlung zu ermitteln.
Diese effektive Weglänge $x$ kann bei einem Druck $p$ und einem Abstand zwischen 
Detektor und $\alpha$-Strahler von $x_0$, sowie einem Normaldruck $p_0$ (letzteres 
sei beides konstant mit $x_0 = 6 \unit{\centi\meter}$ und $p_0 = 1.013 \unit{\bar}$
\cite{normaldruck}) bestimmt werden als:
\begin{equation}
    \label{eqn:4}
    x = x_0 \frac{p}{p_0}
\end{equation}
Diese Gleichung gilt, da die Reichweite von $\alpha$-Teilchen bei $T$=const. und 
$V$=const. proportional zu $p$ ist. Demzufolge ist eine Absorptionsmessung mit 
Variierung des Drucks sinnvoll.
\cite{anleitung11}

\subsection{Fehlerrechnung}
\label{subsec:fehler}
Die gemessenen Werte unterliegen Messunsicherheiten und werden demnach im
Folgenden nicht als fehlerfrei angesehen. Die Fehler entstehen bei der
Bildung der Mittelwerte durch den Fehler des Mittelwerts und bei der
Regressionsrechnung sowie der Fehlerforpflanzung durch Python.
Der Mittelwert ist gegeben durch:
\begin{equation}
    \overline{x} = \frac{1}{N} \sum\limits_{k=1}^N x_K
\end{equation}
Der Fehler des Mittelwerts ist gegeben durch 
\begin{equation}
    \begin{aligned}
        \increment \overline{x} &= \sqrt{\overline{x^2\kern-0.1em} - \overline{x}^2} \\
                            &= \frac{\sqrt{\frac{1}{N-1} \sum\limits_{i=1}^N (x_i - \overline{x})^2}}{\sqrt{N}}.
    \end{aligned}
\end{equation}

Um Fehler einzubeziehen, wird die Gauß'sche Fehlerfortpflanzung verwendet:
\begin{equation}
    \label{eqn:9}
    \increment f = \sqrt{\left(\frac{\partial f}{\partial x}\right)^2 \cdot \left(\increment x\right)^2 + \left(\frac{\partial f}{\partial y}\right)^2 \cdot \left(\increment y\right)^2 + .... + \left(\frac{\partial f}{\partial z}\right)^2 \cdot \left(\increment z\right)^2}
\end{equation}