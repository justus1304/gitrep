\section{Theorie}
\label{sec:Theorie}


Wir lenken eine Feder um enen bestimmten weg aus der Ruhelage aus 
und Messen daraufhin die Rückstellkraft der Feder, um über die 
Formel

\begin{equation}
    \symbf{F} = \symbf{D} \times \Delta x 
    \label{eq:example}
\end{equation}

einen wert für die Ferderkonstante $\symbf{D}$ zu bekommen.

