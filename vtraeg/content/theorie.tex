\section{Theorie}
\label{sec:Theorie}

Das Trägheitsmoment entspricht der Kraft, welche gegen die Änderung der
Winkelgeschwindigkeit $\omega$ wirkt. Ausgehend von einer punktförmigen 
Masse m mit einem Radius r (der Abstand zur Drehachse) lässt sich das 
Trägheitsmoment berechnen durch
\begin{equation}
    I = m r^2 \quad \text{,[I]=} kg \cdot m^2.
    \label{eqn:gl1}
\end{equation}
\noindent Dreht sich ein Körper um eine (beliebige) Achse, so bewegen sich auch alle
Elemente dieses Körpers mit identischer Winkelgeschwindigkeit um diese. Das
gesamte Trägheitsmoment eines Körpers berechnet sich somit aus der Summe
\begin{equation}
    I = \sum\limits_{i} m_i r_i^2
    \label{eqn:gl2}
\end{equation}
\noindent der einzelnen Trägheitsmomente der verschiedenen Masseelementen. Für
infinitesimale Massen gilt
\begin{equation}
    I = \int r^2 dm .
    \label{eqn:gl3}
\end{equation}
\noindent Das Trägheitsmoment hängt dabei von Lage und Objekt ab. Sei die
Achse, um die sich das Objekt drehe, jeweils in positiver z-Richtung liegend.
Dann wären die Trägheitsmomente folgender Objekte folgendermaßen definiert:\\
\noindent Langer, infinitesimal dünner Stab mit Länge l
\begin{equation}
    I = \frac{ml^2}{12}
    \label{eqn:gl4}
\end{equation}
Kugel
\begin{equation}
    I = \frac{2}{5} m r^2
    \label{eqn:gl5}
\end{equation}
Zylinder (stehend)
\begin{equation}
    I = \frac{1}{2} m r^2
    \label{eqn:gl6}
\end{equation}
Zylinder (liegend) mit Höhe h
\begin{equation}
    I = m \left(\frac{r^2}{4} + \frac{h^2}{12} \right) 
    \label{eqn:gl7}
\end{equation}
Wenn eine Kraft $\vec{F}$ auf einen sich drehbaren Körper (dieser befinde
sich im Abstand $\vec{r}$ zur Drehachse) wirkt, entsteht das Drehmoment.
Dieses ist definiert durch:
\begin{equation}
    \vec{M} = \vec{M} \times \vec{r} \quad \text{oder betragsmäßig:} 
    \quad M = F \cdot r \quad \text{,[M]=} J.\end{equation}
    \label{eqn:gl8}
Allerdings lässt es sich auch über die Winkelrichtgröße $D$ (rücktreibendes
Moment der Feder) darstellen:
\begin{equation}
    M = D \cdot \varphi \quad \text{,[D]=} \frac{N \cdot m}{rad}
    \label{eqn:gl9}
\end{equation}
Wird der Körper um einen Winkel $\varphi$ aus seiner Ruhelage ausgelenkt,
so wirkt ein rücktreibendes Drehmoment, durch eine Feder verursacht, 
entgegen. Es kommmt zur harmonischen Schwingung, dessen Schwingungsdauer
sich als
\begin{equation}
    T = 2 \pi \sqrt{\frac{I}{D}} \quad \text{,[T]=} s.
    \label{eqn:gl10}
\end{equation}
beschreiben lässt.
\cite{sample}

\subsection{Fehlerrechnung}
Die gemessenen Werte für Längen, Periodendauern und Massen werden im folgenden als nicht fehlerbehaftet 
angesehen. Die Fehler entstehen bei der bildung der Mittelwerte durch den Fehler des Mittelwertes und
bei der regressionsrechnung sowie der Fehlerfortpflanzung durch Python.

der Fehler des Mittelwertes beträgt
\begin{equation}
  \sqrt{\overline{x^2} - \overline{x}^2}
\end{equation}

Die Formel für die Feherfortpflanzung lautet
\begin{equation}
  \Delta f = \sqrt{\left(\frac{\partial f}{\partial x}\right)^2 \cdot \left(\Delta x\right)^2 + \left(\frac{\partial f}{\partial y}\right)^2 \cdot \left(\Delta y\right)^2 + .... + \left(\frac{\partial f}{\partial z}\right)^2 \cdot \left(\Delta z\right)^2}
\end{equation}