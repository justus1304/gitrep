\section{Zielsetzung}
\label{sec:Zielsetzung}

Unter der thermischen Elektronenemission versteht man den Effekt des 
Elektronenaustritts aus Metalloberflächen, wenn diese erhitzt werden. In 
diesem Versuch soll die Abhängigkeit des Elektronenstroms von der Temperatur 
der Metalloberfläche erforscht werden. Dazu wird eine Kennlinienschar erstellt, 
aus der der Sättigungsstrom abgelesen werden soll. Weiterhin soll herausgefunden 
werden, für welchen Bereich das Langmuir-Schottkysche Raumladugnsgesetz 
Gültigkeit findet. Ebenso soll für die maximale Heizleistung das Anlaufstromgebiet
untersucht werden und so die Kathodentemperatur ermittelt werden. Diese soll 
außerdem mittels einer Leistungsbilanz des Heizstromkreises herausgefunden 
werden. Aus den Werten für die Kathodentemperatur und den Sättigungsstrom soll 
zuletzt die Austrittsarbeit von Wolfram herausgefunden werden.