\section{Theorie}
\label{sec:Theorie}

Der gesamte Versuch basiert auf der Tatsache, dass bewegte Ladungen magnetische 
Felder erzeugen. Aufgrund von Elektronenbewegungen verfügen Atome über ein 
magnetisches Moment. Es gilt die Verknüpfung
\begin{equation}
    \label{eqn:1}
    \vec{B} = \mu \vec{H},
\end{equation}
wenn sich diese Momente statistisch im Raum verteilen. Der Faktor $\mu$ wird 
dabei als Permeabilität bezeichnet und setzt sich durch die Vakuum-Permeabilität
$\mu_0$ und die relative Permeabilität $\mu_r$ (in Materie) mittels des Produkts
Beider zusammen.
Bei jedem stromdurchflossenen Leiter sind die Feldlinien konzentrische Kreise, 
welche senkrecht zu dem Stromfluss stehen. Die dazugehörige Magnetfeldstärke 
$\vec{B}$ mit einem Abstand $r$ zum Leiter lässt sich über das Biot-Savart-Gesetz
bestimmen.
\begin{equation}
    \label{eqn:2}
    \vec{B}(\vec{r}) = \frac{\mu_0}{4 \pi}
        \int_V j(\vec{r'}) \times \frac{\vec{r}-\vec{r'}}{(\lvert \vec{r}- \vec{r'} \rvert)^3} dV'
\end{equation}
Damit lässt sich ein Zusammenhang für die magnetische Flussdichte im Mittelpunkt 
eines Leiterrings mit $n$ Windungen erstellen:
\begin{equation}
    \label{eqn:3}
    |\vec{B}(x)| = \frac{\mu_0 n I}{2} \frac{R^2}{(R^2+x^2)^{\frac{3}{2}}}
\end{equation}
Das Magnetfeld innerhalb eines Solenoids gilt als homogen, solange die Bedingung 
$R<<l$ erfüllt ist (R: Radius, l: Länge). Die Feldlinien verlaufen parallel zu 
der Symmetrieachse und sing geschlossen. Es lässt sich festhalten:
\begin{equation}
    \label{eqn:4}
    B = \mu_r \mu_0 \frac{n}{l} I
\end{equation}
Handelt es sich um eine Spule mit Eisenkern, so fließt noch die Magnetisierung 
$\vec{M}$ mit ein. Der Magnetische Fluss ergibt sich dann als 
\begin{equation}
    \vec{B} = \mu_0 (\vec{H} + \vec{M}).
\end{equation}

\subsection{Hysteresekurve}
Die Hysteresekurve zeigt das Verhalten eines ferromagnetischen Materials unter
Einfluss eines wechselnden Magnetfelds. Bei Anlegung eines äußeren Magnetfeldes 
steigt die Magnetisierung des Materials an. Der anschließend erreichte Wert wird
auch Sättigungswert $B_S$ genannt.

Verschwindet das Feld, bleibt die Remanenz $B_R$, eine Restmagnetisierung, welche 
bei Visualisierung als y-Achsenabschnitt zu verstehen wäre. Um diese aufzuheben,
kann die Koerzitivkraft $H_C$, ein Gegenfeld, genutzt werden. Durch diese Umpolung 
zeigt diese Magnetisierung symmetrisch in die andere Richtung, effektiv wird 
$-B_S$ erreicht. Es ergibt sich die Hysteresekurve.

Dabei ist wichtig, dass diese nicht allgemeingültig ist und von Material zu 
Material unterschiedliches Ausmaß annimmt. Die Kurve hängt vom Ausganszustand
der Probe ab, so sollte das Material auch zu Beginn nicht magnetisiert sein.
Die relative Permeabilität $\mu_r$ ferromagnetischer Materialien abhängig von 
der magnetischen Feldstärke $H$, welche das Außenfeld erzeugt. Demfolgend wird 
die differentielle Permeabilität eingeführt.
\begin{equation}
    \mu_diff = \frac{1}{\mu_0} \frac{dB}{dH}
\end{equation}
