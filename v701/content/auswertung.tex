\section{Auswertung}
\label{sec:Auswertung}

\subsection{Energieverlust}
Um den Energieverlust zu berechnen wird die effektive Weglänge gegen die Energie aufgetragen und der Energieverlust
durch eine lineare Ausgleichsrechnung bestimmt.
Um die effektive Weglänge zu berechnen, werden die Werte für den Druck aus \autoref{tab:11}
und \autoref{tab:12} für die Abstände von $x=4$ und $x=7\unit{\centi\meter}$ in \autoref{eqn:4}
eingesetzt. Die Energie wird in einer Dreisatzrechnung bestimmt, wobei von einer maximalen Energie $E = \qty{4}{\mega\electronvolt}$
bei einem Druck $p = \qty{0}{\bar}$ ausgegangen wird. Es wird von einer linearen 
Energieskala ausgegangen. Die Ausgleichsrechnungen sind \autoref{fig:11} und
\autoref{fig:12} zu entnehmen.
\begin{table}[H]
  \centering
  \caption{Druck, Anzahl der Pulse und Channel für einen Abstand von $x=4$.}
  \label{tab:11}
  \begin{tblr}{
          colspec = {S S S S S S},
          row{1} = {guard, mode = math},
      }
      \toprule
      p/\unit{\bar}&\text{Anzahl Impulse}&\text{Channel}\\
      \midrule
      0   & 59397  & 1159 \\
      50  & 58355  & 1082\\
      100 & 57520  & 1039\\
      150 & 56528  & 1023\\
      200 & 56276  & 967\\
      250 & 56068  & 896\\
      300 & 55243  & 870\\
      350 & 55010  & 879\\
      400 & 54460  & 768\\
      450 & 53865  & 768\\
      500 & 52403  & 687\\
      550 & 51494  & 666\\
      600 & 50589  & 612\\
      650 & 48451  & 534\\
      700 & 44760  & 474\\
      750 & 36158  & 391\\
      800 & 19362  & 366\\
      850 & 3498   & 363\\
      900 & 501    & 367\\
      \bottomrule
  \end{tblr}
\end{table}
\begin{table}[H]
  \centering
  \caption{Druck, Anzahl der Pulse und Channel für einen Abstand von $x=7$.}
  \label{tab:12}
  \begin{tblr}{
          colspec = {S S S S S S},
          row{1} = {guard, mode = math},
      }
      \toprule
      p/\unit{\bar}&\text{Anzahl Impulse}&\text{Channel}\\
      \midrule
      0   & 11903 & 1140\\
      50  & 11615 & 1000\\
      100 & 11442 & 943\\
      150 & 10859 & 805\\
      200 & 11039 & 763\\
      250 & 10634 & 640\\
      300 & 9397  & 514\\
      350 & 5680  & 386\\
      400 & 33    & 363\\
      \bottomrule
  \end{tblr}
\end{table}

\begin{figure}[H]
  \centering
  \caption{Bestimmung des Energieverlusts bei einem Abstand von $x = \qty{4}{\centi\meter}$}
  \label{fig:11}
  \includegraphics{build/abstand4.pdf}
\end{figure}
\noindent Der Steigungsparameter der Ausgleichsrechnung gibt den Energieverlust
an, welcher sich damit bei einem Abstand von 
$x = \qty{4}{\centi\meter}$ zu 
\begin{equation}
  m_\text{4cm} = \qty{-81.26(2.24)}{\mega\electronvolt\per\meter}
\end{equation}
ergibt.
\begin{figure}[H]
  \centering
  \caption{Bestimmung des Energieverlusts bei einem Abstand von $x = \qty{7}{\centi\meter}$}
  \label{fig:12}
  \includegraphics{build/abstand4.pdf}
\end{figure}
\noindent Bei einem Abstand von $x = \qty{7}{\centi\meter}$ ergibt sich durch
den Steigungsparameter ein Energieverlust von 
\begin{equation}
  m_\text{7cm} = \qty{-99.14(3.83)}{\mega\electronvolt\per\meter}
\end{equation}

\subsection{Mittlere Reichweite}
Um die mittlere Reichweite zu bestimmen, wird die Anzahl der Pulse gegen die
effektive Weglänge aufgetragen. Durch den stärksten Abfall wird eine Ausgleisgerade
gezogen und dann der Schnittpunkt dieser Geraden mit einer Geraden auf 
Höhe der halben maximalen Pulszahl ($N_\text{max,4cm}/2 = 29698.5$, $N_\text{max,7cm}/2 = 5951.5$) gesucht. Die Ausgleichsrechnungen und die
eignezeichnete Gerade sowie der Schnittpunkt befinden sich für die jeweiligen
Abstände in \autoref{fig:13} und \autoref{fig:14} abgebildet.

\begin{figure}[H]
  \centering
  \caption{Messwerte der Pulse pro $120\unit{\second}$ und Bestimmung der mittleren Reichweite für den Abstand $x = \qty{4}{\centi\meter}$}
  \label{fig:13}
  \includegraphics{build/anzahl4.pdf}
\end{figure}

\begin{figure}[H]
  \centering
  \caption{Messwerte der Pulse pro $120\unit{\second}$ und Bestimmung der mittleren Reichweite für den Abstand $x = \qty{7}{\centi\meter}$}
  \label{fig:14}
  \includegraphics{build/anzahl7.pdf}
\end{figure}
\noindent Die Schnittpunkte der Ausgleichgeraden mit der Geraden auf Höhe der
halben maximalen Pulsrate geben die mittlere Reichweite an. Die Reichweiten
ergeben sich damit zu 
\begin{align}
  R_\text{4cm} = &\qty{5.3(0.9)}{\centi\meter} \\
  R_\text{7cm} = &\qty{2.4(0.4)}{\centi\meter}.
\end{align}
und nach \autoref{eqn:3} entsprechen diese einer Energie von
\begin{align}
  E_\text{4cm} = &\qty{6.6(0.8)}{\mega\electronvolt} \\
  E_\text{7cm} = &\qty{3.9(0.4)}{\mega\electronvolt}.
\end{align}

\subsection{Statistische Messung}
Es wird $100$ mal eine Statistische Messung der Pulsrate (Pulse pro $10\unit{\second}$) in gleichem Abstand $x=\qty{7}{\centi\meter}$ 
durchgeführt. Die ermittelten Werte werden in einem Histogramm in \autoref{fig:05} in $10$ Bins und in \autoref{fig:06} in $5$ Bins Dargestellt. 
Außerdem wird zum Vergleich eine Poisson und Gaußverteilung mit dem aus den Messwerten ermittelten Mittelwert von $\overline{N} = \qty{1792(8)}{}$ und einer Varianz von 
$\sigma = \qty{78.11}{}$, welche nach \autoref{subsec:fehler} ermittelt wurden, 
als Histogramm gegenübergestellt.
\begin{figure}[H]
  \centering
  \caption{Messwerte sowie Gauß- und Poissonverteilung in $10$ Bins}
  \label{fig:05}
  \includegraphics[width=\textwidth]{build/histogram_1.pdf}
\end{figure}
\begin{figure}[H]
  \centering
  \caption{Messwerte sowie Gauß- und Poissonverteilung in $10$ Bins}
  \label{fig:06}
  \includegraphics{build/histogram_2.pdf}
\end{figure}