\section{Auswertung}
\label{sec:Auswertung}

\subsection{Gauß'sche Fehlerfortpflanzung}
Da es in Experimenten zu unumgänglichen Messunsicherheiten kommt, ist eine 
Fehlerkalkulation obligatorisch. In der folgenden Auswertung wird eine 
Fehlerrechnung mithilfe der Gauß'schen Fehlerfortpflanzung durchgeführt, welche
lautet:
\begin{equation}
    \sigma_f = \sqrt{\sum\limits_{i=1}^N \left( \frac{\partial f}{\partial x_i} \sigma_f \right) ^2}.
\end{equation}

\section{Fehlerfortpflanzung}
Für die Fehlerfortpflanzung nach Gauß ergibt sich für die Linearisierung
$f_e(x)=Lx^2-\frac{x^3}{3}$ der einsitigen Einspannung dann folgender Term:
\begin{equation}
  \Delta f_e(x) = \sqrt{(2Lx-x^2)^2 (\Delta x)^2}.
\end{equation}
Des Weiteren ist die Fehlerfortpflanzung für die beidseitige Einspannung 
mit Linearisierung $f_b(x)=3L^2x-4x^3$ gegeben durch:
\begin{equation}
  \Delta f_b(x) = \sqrt{(3L^2-12x^2)^2 (\Delta x)^2}.
\end{equation}



%runder Stab-einseitige Einspannung plot und Messwerte
\subsection{Runder Stab-einseitige Einspannung}
\label{sec:RunderStabEinseitig}
Hier wird das Elastizitätsmodul für einen einseitig eingespanten Runden Stab
der Länge $L = 0,59m$ und Radius $r = 0,005m$ sowie einer Masse $m = 0,39kg$
ermittelt. Dazu sind folgende Formeln gegeben:
\begin{equation}
  \label{eqn:dEinseitigeEinspannung}
    \symbf{D}(x) = \frac{F}{2 E \symbf{I}}\left(L x^2 - \frac{x^3}{3}\right)
\end{equation}
und
\begin{equation}
  \label{eqn:FlächenträgheitKreis}
    \symbf{I_r} = \int_{0}^{2 \pi} \int_{0}^{r} r'^3 dr'd\varphi = \frac{r^4 \pi}{2}.
\end{equation}
Darin ist $L$ die Stablänge, $E$ das Elastizitätsmodul, $F$ die angreifende
Kraft und $\symbf{I}$ das Flächenträgheitsmoment. Nun wird die Auslenkung gegen 
$\left(L x^2 - \frac{x^3}{3}\right)$ aufgetragen. Es wird eine 
Ausgleichsrechnung mit Python durchgeführt und mit dem gewonnenen Parameter a,
welcher die Steigung der Ausgleichsgeraden \autoref{fig:plotrSeinseit} angibt, 
kann das Elastizitätsmodul bestimmt werden durch Gleichsetzung der Steigung
a mit dem Vorfaktor aus \autoref{eqn:dEinseitigeEinspannung} und umgestellt zu:
 \begin{equation*}
  \label{eqn:elastizitätsmodulEinseitig}
    \symbf{E} = \frac{F}{2a \symbf{I}}.
\end{equation*}

Die Ausgleichsrechnung zu den Werten aus \autoref{tab:t1} ergibt die Werte 
\begin{equation*}
  a = 0.0100 \pm 0.0005 \quad \text{und} \quad b = 0.0005 \pm 0.0001.
\end{equation*}
Es ergibt sich der folgender Graph:
\begin{figure}[H]
  \centering
  \includegraphics{plotrSeinseit.pdf}
  \caption{Einseitiger Aufhang-runder Stab.}
  \label{fig:plotrSeinseit}
\end{figure}

\begin{table}[H]
  \centering
  \caption{Werte zur bestimmung von E(Runder Stab)}
  \label{tab:t1}
  %\sisetup{table-format=1.1, per-mode=reciprocal}
  \begin{tblr}{
      colspec = {S S S S S},
      row{1} = {guard, mode=math},
      row{2} = {guard, mode=math},
      %vline{4} = {2}{-}{text=\clap{$\pm$}},
    }
    \toprule
    m_{gewicht}\unit{\kilo\gram} & g_{erde}\unit{\meter\per\second\squared} & r\unit{\meter} & I & a \\
    \midrule
    0.2 & 9.81 &0.0025 & \num{3.06e-11} & \num{0.0100 +- 0.0005 } \\
    \bottomrule
  \end{tblr}
\end{table}

\noindent Die Werte aus \autoref{tab:t1} sind jene, welche zur Bestimmung des
Elastizitätsmoduls des runden Stabes bei einseitiger Aufhängung nötig sind.
Das Elastizitätsmodul ergibt sich als
\begin{equation*}
  E = \qty{1.99(0.1)e11}{\newton\per\meter\squared}.
\end{equation*}

\begin{table}[H]
  \centering
  \caption{Messwerte x, D\textsubscript{0x}, D(x)}
  \label{tab:at}
  %\sisetup{table-format=1.1, per-mode=reciprocal}
  \begin{tblr}{
      colspec = {S S S},
      row{1} = {guard, mode=math},
      %vline{4} = {2}{-}{text=\clap{$\pm$}},
    }
    \toprule
    x(mm) & D\textsubscript{0x}(mm) & D(x)(mm)\\
    \midrule
    500 & 4.31 & 2    \\
    475 & 4.69 & 2.55 \\
    450 & 5.07 & 3    \\
    425 & 5.48 & 3.56 \\
    400 & 5.85 & 4.1  \\
    375 & 6.22 & 4.61 \\
    350 & 6.59 & 5.16 \\
    325 & 6.9  & 5.64 \\
    300 & 7.23 & 6.12 \\
    275 & 7.55 & 6.59 \\
    250 & 7.87 & 7.04 \\
    \midrule
    \bottomrule
  \end{tblr}
\end{table}




%Eckiger Stab-einseitige Einspannung plot und Messwerte
\subsection{Eckiger Stab-einseitige Einspannung}
Die Berrechnung des Elastizitätsmoduls des eckigen, einseitig 
eingespannten Stabs wird analog zu der in\autoref{sec:RunderStabEinseitig} 
durchgeführten Berechnung bestimmt. Es wird ein Stab mit einer Kantenlänge von
$a = 0,01m$, einer Länge von $L = 0,6m$ und einer Masse von $m = 0,5025kg$ 
betrachtet. Das Flächenträgheitsmoment eines eckigen
Stabes ist ergibt sich mit einer quadratischen Grundfläche als:
\begin{equation}
  \symbf{I_e} = \int_{0}^{1} \int_{0}^{1} a^2 da da = \int_{0}^{1} \frac{1}{3}a^3 da = \frac{h^4}{12}.
\end{equation}
Aus der Ausgleichsrechnung mit den Werte aus \autoref{tab:at} folgern die Werte:
\begin{equation*}
  a = 0.0059 ± 0.0003 \quad \text{und} \quad b = 0.0002 ± 0.0000.
\end{equation*}
\\
\\
\\
\\
\\
\\
\\
\\
\\
\\
\\
\\
\\
\\
\\
Daraus kommt die Ausgleichsgerade zustande:
\begin{figure}[H]
  \centering
  \includegraphics{ploteSeinseit.pdf}
  \caption{Einseitiger Aufhang-eckiger Stab.}
  \label{fig:ploteSeinseit}
\end{figure}

\begin{table}[H]
  \centering
  \caption{Werte zur bestimmung von E(Eckiger Stab)}
  \label{tab:t2}
  %\sisetup{table-format=1.1, per-mode=reciprocal}
  \begin{tblr}{
      colspec = {S S S S S},
      row{1} = {guard, mode=math},
      row{2} = {guard, mode=math},
      %vline{4} = {2}{-}{text=\clap{$\pm$}},
    }
    \toprule
    m_{gewicht}\unit{\kilo\gram} & g_{erde}\unit{\meter\per\second\squared} & h\unit{\meter} & I & a \\
    \midrule
    0.2 & 9.81 &0.01 & \num{8.33e10} & \num{ 0.0059 +- 0.0003} \\
    \bottomrule
  \end{tblr}
\end{table}

\noindent Mit dem Parameter a der Ausgleichsrechnung in \autoref{fig:ploteSeinseit}
und den Werten aus \autoref{tab:t2} ergibt sich der Elastizitätsmodul
\begin{equation*}
    E = \qty{1.99(0.1)e11}{\newton\per\meter\squared}
\end{equation*}

\begin{table}[H]
  \centering
  \caption{Messwerte x, D\textsubscript{0x}, D(x)}
  \label{tab:at}
  %\sisetup{table-format=1.1, per-mode=reciprocal}
  \begin{tblr}{
      colspec = {S S S},
      row{1} = {guard, mode=math},
      %vline{4} = {2}{-}{text=\clap{$\pm$}},
    }
    \toprule
    x(mm) & D\textsubscript{0x}(mm) & D(x)(mm)\\
    \midrule
    500 & 6.01 & 4.71 \\
    475 & 6.26 & 5.00 \\
    450 & 6.41 & 5.27 \\
    425 & 6.65 & 5.57 \\
    400 & 6.83 & 5.88 \\
    375 & 7.00 & 6.21 \\
    350 & 7.22 & 6.47 \\
    325 & 7.55 & 6.79 \\
    300 & 7.75 & 7.13 \\
    275 & 7.89 & 7.44 \\
    250 & 8.14 & 7.67 \\
    \midrule
    \bottomrule
  \end{tblr}
\end{table}

%eckiger Stab-beidseitige Einspannung plot und Messwerte
% \subsection{eckiger Stab-beidseitige Bespannung}
% \begin{figure}
%   \centering
%   \includegraphics{beidseitigLinks.pdf}
%   \caption{Einseitiger Aufhang-eckiger Stab.}
%   \label{fig:ploteSeinseit}
% \end{figure}



%%%%%%%%%%%%%%%%%%%%%%%%%%%%%%%%%%%%%%%%%%%%%%%%%%%%%%%%%%%%%%%%%%%%%%%%%%%%%%%%%%%%%%%%%%%%%%%%%%%%%%%%%%%%%%%%%%%%%%%%%%%%%%%%%%%%
\subsection{Beidseitige Aufhängung}
Bei der Bestimmung des Elastizitätsmoduls bei beidseitiger Aufhängung werden
die Ausgleichsrechnungen in eine Rechnung für die linke Seite und eine für
die rechte Seite (von dem Þunkt aus, an dem das Gewicht hängt) aufgeteilt.
Das resultiert aus den unterschiedlichen Ausdrücken für $D(x)$, wie man es
\autoref{eqn:linksDx} und \autoref{eqn:RechtsDx} entnehmen kann.
Für den Bereich $(x\leq\frac{L}{2})$ wird die x Achse mit dem Term
\begin{equation*}
    3L^2-4x^3
\end{equation*}
linearisiert, sowie im bereich $(\frac{L}{2}\leq x \leq L)$ mit dem Term
\begin{equation}
  4x^3-12Lx^3+9L^2x-L^3
\end{equation}

\noindent Für den Teil, welcher links des Gewichts liegt, ergaben sich die 
Messwerte für den runden sowie den eckigen Stab wie in \autoref{tab:t5} 
dokumentiert.

\noindent Die Werte der Ausgleichsrechnung belaufen sich auf 
\begin{equation*}
  a = 0.00243 \pm 0.00018 \quad \text{und} \quad b = 0.00008 \pm 0.00003
\end{equation*}
für den runden Stab sowie
\begin{equation*}
  a = 0.00189 \pm 0.00006 \quad \text{und} \quad b = 0.00005 \pm 0.00001
\end{equation*}
für den eckigen Stab.
\\
\\
\\
\\
\noindent Die Regressionen dazu sehen wie folgt aus:
\begin{figure}[H]
  \centering
  \includegraphics{beidseitigLinks.pdf}
  \caption{Beidseitige aufhängung-linke Seite.}
  \label{fig:linksBeidseitig}
\end{figure}
%%%%%%%%%%%%%%%%%%%%
Für die linke Seite Ergeben sich die Elastizitätsmodule zu 
\begin{equation*}
  E_{links,rund} = \qty{1.71(0.12)e11}{\newton\per\meter\squared}
\end{equation*}
und
\begin{equation*}
  E_{links,eckig} = \qty{1.29(0.04)e11}{\newton\per\meter\squared}.
\end{equation*}


\noindent Der rechte Teil vom Gewicht ergibt sich ähnlich analog. Die Messwerte
befinden sich in \autoref{tab:t6}. Es ergeben sich für die Parameter 
\begin{equation*}
  a = 0.00374 \pm 0.00014 \quad \text{und} \quad b = -0.00020 \pm 0.00003
\end{equation*}
für den runden Stab sowie
\begin{equation*}
  a = 0.00272 \pm 0.00008 \quad \text{und} \quad b = -0.00021 \pm 0.00001
\end{equation*}
für den eckigen Stab.
\\
\\
\\
\\
\noindent Ebenso sehen die Graphen dazu ähnlich aus.
\begin{figure}[H]
  \centering
  \includegraphics{"rechtsBeidseitig.pdf"}
  \caption{Beidseitige aufhängung-rechte Seite.}
  \label{fig:rechtsBeidseitig}
\end{figure}
%%%%%%%%%%%%%%%%%%%%
\noindent Für die rechte Seite Ergeben sich die Elastizitätsmodule zu 
\begin{equation*}
  E_{Rechts,rund} = \qty{1.11(0.04)e11}{\newton\per\meter\squared}
\end{equation*}
und
\begin{equation*}
  E_{Rechts,eckig} = \qty{9.01(0.26)e10}{\newton\per\meter\squared}.
\end{equation*}



%\begin{figure}
%  \centering
%  \includegraphics{beidseitigRechts.pdf}
%  \caption{Beidseitige Aufhängung-runder Stab rechte Seite.}
%  \label{fig:rundBeidseitigRechts}
%\end{figure}


\begin{table}[H]
  \centering
  \caption{Messwerte a, D\textsubscript{0,r}, D\textsubscript{x,r}, D\textsubscript{0,e}, D\textsubscript{x,e}}
  \label{tab:t5}
  %\sisetup{table-format=1.1, per-mode=reciprocal}
  \begin{tblr}{
      colspec = {S S S S S},
      row{1} = {guard, mode=math},
      %vline{4} = {2}{-}{text=\clap{$\pm$}},
    }
    \toprule
    a(mm) & D\textsubscript{0,r}(mm) & D\textsubscript{x,r}(mm) & D\textsubscript{0,e}(mm) & D\textsubscript{x,e}(mm)\\
    \midrule
    20  & 8.44 & 7.88 & 9.51  & 8.89  \\
    40  & 8.34 & 7.75 & 9.66  & 9.05  \\
    60  & 8.21 & 7.61 & 9.80  & 9.25  \\
    80  & 8.08 & 7.52 & 9.96  & 9.45  \\
    100 & 7.98 & 7.44 & 10.11 & 9.63  \\
    120 & 7.86 & 7.37 & 10.26 & 9.81  \\
    140 & 7.76 & 7.29 & 10.40 & 10.00 \\
    160 & 7.63 & 7.23 & 10.55 & 10.21 \\
    180 & 7.54 & 7.16 & 10.68 & 10.40 \\
    200 & 7.43 & 7.12 & 10.81 & 10.61 \\
    \midrule
    \bottomrule
  \end{tblr}
\end{table}

xx
\begin{table}[H]
  \centering
  \caption{Messwerte a, D\textsubscript{0,r}, D\textsubscript{x,r}, D\textsubscript{0,e}, D\textsubscript{x,e}}
  \label{tab:t6}
  %\sisetup{table-format=1.1, per-mode=reciprocal}
  \begin{tblr}{
      colspec = {S S S S S},
      row{1} = {guard, mode=math},
      %vline{4} = {2}{-}{text=\clap{$\pm$}},
    }
    \toprule
    a(mm) & D\textsubscript{0,r}(mm) & D\textsubscript{x,r}(mm) & D\textsubscript{0,e}(mm) & D\textsubscript{x,e}(mm)\\
    \midrule
    20  & 9.51  & 8.89  & 10.52 & 10.15 \\
    40  & 9.66  & 9.05  & 10.56 & 10.19 \\
    60  & 9.80  & 9.25  & 10.55 & 10.19 \\
    80  & 9.96  & 9.45  & 10.53 & 10.21 \\
    100 & 10.11 & 9.63  & 10.54 & 10.25 \\
    120 & 10.26 & 9.81  & 10.55 & 10.29 \\
    140 & 10.40 & 10.00 & 10.57 & 10.34 \\
    160 & 10.55 & 10.21 & 10.58 & 10.41 \\
    180 & 10.68 & 10.40 & 10.65 & 10.50 \\
    200 & 10.81 & 10.61 & 10.68 & 10.60 \\
    \midrule
    \bottomrule
  \end{tblr}
\end{table}

