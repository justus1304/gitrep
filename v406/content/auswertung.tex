\section{Auswertung}
\subsection{Einfachspalt}
Es soll die Spaltbreite ermittelt werden, indem man die Apperturfunktion der Blende auf das 
Beugungsbild zurückführt. Das Intensitätsmuster des Beugungsbildes wurde Punktweise aufgenommen und 
die Messwerte befinden sich in \autoref{tab:10}. Die Meswerte,sowie die Ausgleichsfunktion nach \autoref{eqn:8},
sind \autoref{abb:10} zu entnehmen.
\begin{table}[H]
    \centering
    \caption{Messwerte Intensitätsverteilung für den festen Einfachspalt}
    \label{tab:10}
    \begin{tblr}{
        colspec = {S S | S S | S S | S S},
        row{1} = {guard, mode=math},}
           \toprule
            \text{x}/ \unit{\milli\meter} & \text{I}/ \unit{\micro\ampere}& \text{x} /\unit{\milli\meter} & \text{I} /\unit{\micro\ampere}& \text{x} /\unit{\milli\meter} & \text{I} /\unit{\micro\ampere} & \text{x} /\unit{\milli\meter} & \text{I}/ \unit{\micro\ampere}\\
           \midrule
           -25   &0.002 & -7      & 0.012   & 4   & 0.060 &20 &0.002 \\              
           -24   &0.002 & -6.5    & 0.008   & 4.5 & 0.040 &21 &0.002 \\    
           -23   &0.000 & -6      & 0.006   & 5   & 0.014 &22 &0.000 \\    
           -22   &0.000 & -5.5    & 0.010   & 5.5 & 0.006 &23 &0.000 \\    
           -21   &0.002 & -5      & 0.020   & 6   & 0.004 &24 &0.002 \\    
           -20   &0.002 & -4.5    & 0.040   & 6.5 & 0.008 &25 &0.002 \\    
           -19   &0.002 & -4      & 0.090   & 7   & 0.014 & & \\    
           -18   &0.000 & -3.5    & 0.140   & 7.5 & 0.018 & & \\    
           -17   &0.000 & -3      & 0.200   & 8   & 0.020 & & \\
           -16   &0.002 & -2.5    & 0.280   & 8.5 & 0.018 & & \\
           -15   &0.004 & -2      & 0.340   & 9   & 0.016 & & \\
           -14   &0.006 & -1.5    & 0.420   & 9.5 & 0.012 & & \\
           -13   &0.006 & -1      & 0.460   & 10  & 0.008 & & \\
           -12   &0.004 & -0.5    & 0.500   & 11  & 0.002 & & \\    
           -11   &0.002 & 0       & 0.520   & 12  & 0.004 & & \\
           -10.5 &0.004 & 0.5     & 0.500   & 13  & 0.006 & & \\
           -10   &0.006 & 1       & 0.460   & 14  & 0.006 & & \\
           -9.5  &0.010 & 1.5     & 0.400   & 15  & 0.004 & & \\
           -9    &0.014 & 2       & 0.340   & 16  & 0.002 & & \\    
           -8.5  &0.016 & 2.5     & 0.260   & 17  & 0.002 & & \\    
           -8    &0.017 & 3       & 0.200   & 18  & 0.004 & & \\    
           -7.5  &0.016 & 3.5     & 0.120   & 19  & 0.004 & & \\
            \bottomrule
    \end{tblr}
\end{table}

\label{sec:Auswertung}
\begin{figure}[H]    
    \centering
    \caption{Ausgleichsrechnung Intensitätsverteilung des festen Einfachspaltes}
    \label{abb:10}
    \includegraphics{teil1.pdf}
\end{figure}

Durch die Ausgleichsrechnung erhalten wir folgende Parameter 
\begin{align}
    \text{A} = & 0.715 \unit{\micro\ampere} \\
    \text{b} = & 0.21 \unit{\milli\meter}
\end{align}
Dabei gibt der Parameter b die breite des Spaltes an an dem das Licht gebeugt wurde. 
Somit beträgt die experimentell ermittelte breite des Spaltes $0.21 \unit{\milli\meter}$.



\subsection{Doppelspalt}
Ein ähnliches Vorgehen wird bei der Bestimmung der Breite des Doppelspaltes angewandt. Dort wird 
jedoch eine etwas andere Funktion für die Ausgleichsrechnung verwndet. Diese befindet sich in \autoref{eqn:10}.

\begin{equation}
    \label{eqn:10}
    I\left(\Phi\right) = A^2 \cdot \text{sinc}^2\left(b \cdot \left(x - d\right)\right) ^2 \cdot \cos^2{\left(c \cdot \left(x - d\right)\right)}
\end{equation}

Die Messdaten sind \autoref{tab:11} zu entnehmen und die Ausgleichsrechnung ist in \autoref{abb:11} visualisiert.
Die Ausgleichsrechnung durch einen ""curve Fit" des python paketes skipy ergibt die Parameter 
\begin{align}
    \text{A} = & 0.61 \unit{\micro\ampere} \\
    \text{b} = & 0.09 \unit{\milli\meter}\\
    \text{c} = & 2.23     \\
    \text{d} = & -0.20   \unit{\milli\meter}
\end{align}

\begin{table}[H]
    \centering
    \caption{Messwerte Intensitätsverteilung für den festen Doppelspalt}
    \label{tab:11}
    \begin{tblr}{
        colspec = {S S | S S | S S | S S},
        row{1} = {guard, mode=math},}
           \toprule
            \text{x}/ \unit{\milli\meter} & \text{I}/ \unit{\micro\ampere}& \text{x} /\unit{\milli\meter} & \text{I} /\unit{\micro\ampere}& \text{x} /\unit{\milli\meter} & \text{I} /\unit{\micro\ampere} & \text{x} /\unit{\milli\meter} & \text{I}/ \unit{\micro\ampere}\\
           \midrule
           -25   & 0.003      &    -8.5        &      0.018        &      0.5    & 0.020       &    9.5    & 0.004       \\              
           -24.5 & 0.002      &    -8.25       &      0.014        &      0.75   & 0.100       &    9.75   & 0.002       \\    
           -24   & 0.001      &    -8          &      0.004        &      1      & 0.300       &    10     & 0.002       \\    
           -23.5 & 0.003      &    -7.75       &      0.010        &      1.25   & 0.400       &    10.5   & 0.004       \\    
           -23   & 0.000      &    -7.5        &      0.030        &      1.5    & 0.240       &    11     & 0.008       \\    
           -22.5 & 0.001      &    -7.25       &      0.056        &      1.75   & 0.060       &    11.5   & 0.004       \\    
           -22   & 0.002      &    -7          &      0.050        &      2      & 0.040       &    12     & 0.004       \\    
           -21.5 & 0.001      &    -6.75       &      0.020        &      2.25   & 0.200       &    12.5   & 0.012       \\    
           -21   & 0.002      &    -6.5        &      0.006        &      2.5    & 0.300       &    13     & 0.004       \\
           -20.5 & 0.003      &    -6.25       &      0.040        &      2.75   & 0.280       &    13.5   & 0.006       \\
           -20   & 0.002      &    -6          &      0.100        &      3      & 0.140       &    14     & 0.012       \\
           -19.5 & 0.002      &    -5.75       &      0.140        &      3.25   & 0.020       &    14.5   & 0.002       \\
           -19   & 0.001      &    -5.5        &      0.100        &      3.5    & 0.040       &    15     & 0.006       \\
           -18.5 & 0.001      &    -5.25       &      0.020        &      3.75   & 0.140       &    15.5   & 0.002       \\    
           -18   & 0.002      &    -5          &      0.020        &      4      & 0.222       &    16     & 0.003       \\
           -17.5 & 0.001      &    -4.75       &      0.120        &      4.25   & 0.140       &    16.5   & 0.001       \\
           -17   & 0.000      &    -4.5        &      0.220        &      4.5    & 0.040       &    17     & 0.002       \\
           -16.5 & 0.002      &    -4.25       &      0.200        &      4.75   & 0.000       &    17.5   & 0.003       \\
           -16   & 0.006      &    -4          &      0.100        &      5      & 0.040       &    18     & 0.003       \\    
           -15.5 & 0.006      &    -3.75       &      0.020        &      5.25   & 0.100       &    18.5   & 0.003       \\    
           -15   & 0.002      &    -3.5        &      0.060        &      5.5    & 0.100       &    19     & 0.003       \\    
           -14.5 & 0.012      &    -3.25       &      0.240        &      5.75   & 0.060       &    19.5   & 0.003       \\
           -14   & 0.006      &    -3          &      0.320        &      6      & 0.020       &    20     & 0.003       \\
           -13.5 & 0.002      &    -2.75       &      0.200        &      6.25   & 0.000       &    20.5   & 0.003      \\
           -13   & 0.012      &    -2.5        &      0.060        &      6.5    & 0.040       &    21     & 0.002      \\
           -12.5 & 0.004      &    -2.25       &      0.020        &      6.75   & 0.060       &    21.5   & 0.003      \\
           -12   & 0.004      &    -2          &      0.140        &      7      & 0.040       &    22     & 0.002      \\
           -11.5 & 0.006      &    -1.75       &      0.400        &      7.25   & 0.020       &    22.5   & 0.002      \\
           -11   & 0.002      &    -1.5        &      0.400        &      7.5    & 0.000       &    23     & 0.003      \\
           -10.5 & 0.002      &    -1.25       &      0.200        &      7.75   & 0.000       &    23.5   & 0.002      \\
           -10   & 0.004      &    -1          &      0.020        &      8      & 0.020       &    24     & 0.002      \\
           -9.75 & 0.004      &    -0.75       &      0.060        &      8.25   & 0.020       &    24.5   & 0.003      \\
           -9.5  & 0.002      &    -0.5        &      0.300        &      8.5    & 0.000       &    25     & 0.002      \\
           -9.25 & 0.002      &    -0.25       &      0.140        &      8.75   & 0.000       &       &   \\
           -9    & 0.004      &    0           &      0.360        &      9      & 0.040       &        &  \\
           -8.75 & 0.012      &    0.25        &      0.140        &      9.25   & 0.004       &         & \\
        \bottomrule
    \end{tblr}
\end{table}

\begin{figure}[H]
    \centering
    \caption{Ausgleichsrechnung Intensitätsverteilung des festen Doppelspaltes}
    \includegraphics{teil2.pdf}
    \label{abb:11}
\end{figure}



