\section{Diskussion}
\label{sec:Diskussion}

Im Folgenden werden die ergebenen Werte für die Breite $b$ beim Einfach-und 
Doppelspalt diskutiert und mit den Hersteller Angaben verglichen. Laut Hersteller
betragen die Abstände:
\begin{align*}
    b_{Hersteller,Einfach} &= 150 \unit{\micro\meter} \\
    b_{Hersteller,Doppel}  &= 100 \unit{\micro\meter}
\end{align*}
\noindent Die experimentell bestimmten Werte belaufen sich auf:
\begin{align*}
    b_{Einfach} &= \num{216.0331+-0.9402} \unit{\micro\meter} \\
    b_{Doppel}  &= \num{121.6017+-3.0790} \unit{\micro\meter}
\end{align*}
\noindent Somit beträgt die Abweichung bei dem Einfachspalt $30,556 \%$. Die 
Abweichung der Breiten beim Doppelspalt hingegen liegt bei $17,764 \%$. Diese 
Werte sind groß, allerdings war ein derartiger Unterschied zwischen den Werten 
zu erwarten. Das liegt hauptsächlich an der Justierung des Intensitätsmessers 
und dem Amperemeter, bei dem es zu zahlreichen Komplikationen kam.
\\
\noindent Zunächst ist
festzuhalten, dass der Aufbau dem statistischen Fehler unterliegt, da die
exakte Ausrichtung des Lasers nicht garantiert ist. Darüber hinaus erfolgt 
die Einstellung des Intensitätsmessers auf der Schiebeleiste durch Augenmaß. 
Das Luxmeter wird auf der Schiene so positioniert, dass die Extrema des 
gebeugten Lichts von maximaler Größe und Schärfe sind. Jenes geschieht nicht 
ohne Ausschluss von Unsicherheiten, welche sich schlussendlich auf die 
Messwerte auswirken.
\\
\noindent Ein weiteres wesentlich größeres Problem ist die Detektion und 
Dokumentation von Lichtmessgerät und Amperemeter. Intensitätsänderungen
wurden durch das Luxmeter unmittelbar nach Kontakt mit der Vorrichtung
erfasst und zeigten sich in entsprechenden Stromänderungen am Amperemeter.
In Anbetracht dieser Tatsache, sind größere Messunsicherheiten nicht abzuschlagen.
\\
\noindent Trotz den großen prozentualen Abweichungen bestätigen die Ergebnisse
die theoretische Erwartung, dass die gemessenen Breiten des Doppelspalts präzisere
Werte liefern als die des Einfachspalts, eine Folge der schärferen 
Intensitätsmaxima im Interferenzmuster. Dies zeigt eine qualitative
Übereinstimmung mit den Vorhersagen der Beugungstheorie. Obwohl die absoluten 
Messwerte aufgrund experimenteller Unsicherheiten von der Theorie abweichen,
bleibt die physikalische Tendenz erkennbar.
\\
\noindent Letztlich ist in \autoref{abb:12} ein direkter Vergleich beider 
Intensitätsverteilungen zu sehen. Auffällig ist, dass die Verteilung des 
Einfachspalts im Vergleich zum Doppelspalt in gestreckter Form vorliegt. Sie 
bildet grob die Einhüllende zur Intensitätsverteilung des Doppelspalts, 
was wiederum die theoretische Erwartung bestätigt. Die Beugung bildet die 
Einhüllende des Doppelspaltmusters da die Interferenz des Doppelspalts als 
Modulation der Einfachspalt-Intensität aufgefasst werden kann. Erkennbare 
Abweichungen wie die Streckung des Graphen sowie austretende Peaks sind durch 
die unterschiedlichen Spaltbreiten zu erklären. Die dezente Asymmetrie in beiden
Verteilungen lässt auf eine minimale Fehljustierung des Strahlengangs schließen.
Dennoch bleibt die fundamentale Übereinstimmung mit der Fraunhoferschen
Beugungstheorie erkennbar, da es sich um eine minimale Verschiebung von weniger 
als $0,01\,\text{rad}$ handelt.