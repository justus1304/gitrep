\section{Diskussion}
\label{sec:Diskussion}
\subsection{Furier-Synthese}
Zu den Ergebnissen bei der Furier-Synthese kann nicht sehr viel diskutiert werden. 
Die Erzielten Funktionen sehen den gewünschten Funktionen(Sägezahn, Rechteck, Dreieck)
Sehr ähnllich. Deutlich zu erkennen sind die deutlichen abweichungen der Funktion in der Nähe der 
unstetigkeiten der jewailigen Funktionen. Beispielsweise ist die Rechteckfunktion an den Rechteckrändern immer sehr hoch.
Das wird in der Literatur als Gibbsches Phänomen bezeichnet. Da eine Furier-Reihe eine überlagerung von stetigen Funktionen 
ist, kommt es an diesen Unstetigkeits-Punkten zu großen Abweichungen.

\subsection{Furier-Analyse}
Bei der Furier Analyse liegt vor allem der Wert b, welcher den Grad der 
Abhängigkeit von n angibt, für die Sägezahnspannung mit $b_s = \num{0.94+-0.28} $sehr nah am
Theoretischen Wert $b_{sT} = 1$. Der Theoretische Wert liegt im Fehlerintervall von $b_s$. Auch für die Dreiecksspannung 
liegt der Theoretische Wert $b_{dT} = 2$ im Fehlerintervall von $b_d = \num{3.0+-2.1}$. Auffällig ist jedoch das ziemlich große Fehlerintervall von $b_d$
Nur der Experimentellle wert $b_r = \num{1.34+-0.26}$ weicht um $34 \% $ von dem Theoretischen Wert $b_{rT} = 1$ ab und der Theoretische Wert liegt nicht im
 Fehlerinterval des Experimentellen Wertes. 