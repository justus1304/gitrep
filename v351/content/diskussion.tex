\section{Diskussion}
\label{sec:Diskussion}
\subsection{Fourier-Synthese}
Grundsätzlich kann in diesem Bereich von einem gelungen Versuch gesprochen werden. 
Die erzielten Funktionen sehen den gewünschten Funktionen (Sägezahn, Rechteck,
Dreieck) sehr ähnllich. Deutlich zu erkennen sind die starken Abweichungen der
Funktion nahe der Unstetigkeiten. So ist etwa die Rechteckfunktion an den
Rechteckrändern immer sehr hoch.
Jenes ist auf das in der Theorie angesprochene Gibbsche Phänomen zurückzuführen.

\subsection{Fourier-Analyse}
Bei der Fourier Analyse liegt vor Allem der Wert $b$, welcher den Grad der 
Abhängigkeit von $n$ angibt, für die Sägezahnspannung mit $b_s = \num{0.94+-0.28}$
sehr nah am theoretischen Wert $b_{sT} = 1$. Dieser Theroriewert liegt im 
Fehlerintervall von $b_s$. Auch für die Dreiecksspannung 
liegt der theoretische Wert $b_{dT} = 2$ im Fehlerintervall von dem experimentell 
bestimmten Wert $b_d = \num{3.0+-2.1}$. Auffällig ist das ziemlich große
Fehlerintervall von $b_d$ für die Rechteckspannung. Der experimentelle Wert
$b_r = \num{1.34+-0.26}$ weicht um $34 \% $ von dem theoretischen Wert 
$b_{rT} = 1$ ab und fällt nicht mehr in dem 1$\sigma$ Toleranzintervall, was auf ein 
Rauschen am Messgerät zurückzuführen sein könnte. Vor Allem bei niedrigen 
Frequenzen haben sich die Amplituden verändert. 