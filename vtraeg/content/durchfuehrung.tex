\section{Durchführung}
\label{sec:Durchführung}

\subsection{Aufbau und Realisierung}
\label{sec:Aufbau und Realisierung}
Für diesen Versuch ist die Verwendung einer Drillachse Hauptbestandteil.
Mithilfe dieser und einer Scheibe mit aufgetragenen Winkeln soll das
Trägheitsmoment verschiedenster Körper bestimmt werden. Dazu werden diese
auf der drehbaren Achse befestigt, welche mit einer Feder versehen ist.
Diese sorgt bei einer Auslenkung des
an die Drehachse gebundenen Körpers für eine entgegenwirkende Kraft.
\begin{figure}[h]
    \centering
    \begin{minipage}{0.45\textwidth}
        \centering
        \includegraphics[width=\textwidth]{Bilder/abb1.jpg}
        \caption{Stativ von vorne}
    \end{minipage}
    \hfill
    \begin{minipage}{0.45\textwidth}
        \centering
        \includegraphics[width=\textwidth]{Bilder/abb2.jpg}
        \caption{Stativ von oben}
    \end{minipage}
\end{figure}
\subsection{Bestimmung der Winkelrichtgröße}
\label{sec:Durchführung}
Um die Winkelrichtgröße $D$ zu bestimmen, wird ein Hebelarm an der Apparatur
befestigt und um den Winkel $\varphi$ ausgelenkt. Mit einer orthogonalen
Ausrichtung der Federwaage zum Hebelarm wird die Kraft der Feder
gemessen.
\subsection{Bestimmung des Eigenträgheitsmoments der Drillachse}
\label{sec:Durchführung}
Für $I_D$ werden zwei identische Körper ($hier$: liegende Zylinder) an den
beiden Enden des Hebelarms befestigt. Nun erfolgt die Auslenkung und die
Messung der Schwingungsdauer T.
\subsection{Bestimmung des Trägheitsmoments zweier Körper}
\label{sec:zwei körper}
Die Bestimmung der Trägheitsmomente zweier verschiedener Körper ($hier$:
stehender Zylinder und Kugel) erfolgt analog zur Bestimmung des liegenden
Zylinders. Das System wird ausgelenkt und Schwingungsdauer T dokumentiert.
\subsection{Bestimmung des Trägheitsmoments einer Puppe}
\label{sec:Durchführung}
Die sich auf einem Stab befindende Puppe wird um den Winkel $\varphi$ 
ausgelenkt. Dabei wird analog wieder die Schwingungsdauer T gemessen. Dieses
Verfahren wird für 2 Posen ausgeführt (siehe Abbildungen). Der Kopf wird
hier als Vollkugel angenommen, die restlichen Bestandteile werden als Zylinder
betrachtet.
\begin{figure}[h]
    \centering
    \begin{minipage}{0.45\textwidth}
        \centering
        \includegraphics[width=\textwidth]{Bilder/abb3.jpg}
        \caption{Position 1}
        \label{fig:pos1}
    \end{minipage}
    \hfill
    \begin{minipage}{0.45\textwidth}
        \centering
        \includegraphics[width=\textwidth]{Bilder/abb4.jpg}
        \caption{Position 2}
        \label{fig:pos2}
    \end{minipage}
\end{figure}