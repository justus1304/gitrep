\section{Diskussion}
\label{sec:Diskussion}

\subsection{Statische Methode}
In diesem Abschnitt sollen die Graphen der statischen Methode diskutiert werden.
Festzuhalten ist, dass es zu einem auffälligen Messfehler nach etwa 630s kam.
Dies ist durch die Apparatur zu begründen. Da die Messaparatur mehrere Jahre 
alt ist, kam es zu zahlreichen falschen Messungen. Es brauchte mehrere Versuche, 
um Werte zu erhalten, welche sich wissenschaftlich wiederverwerten lassen ließen.
Jenes zeichnet sich in den Graphen ab. Tatsächlich sollten die Graphen aus 
\autoref{fig:f1} und \autoref{fig:f2} keine Beugung aufweisen, sondern linear 
weiterverlaufen. Analog sollten die Graphen aus \autoref{fig:f3} und \autoref{fig:f4}
verlaufen.
\par\vspace{0.5em}
\noindent Jedoch liefern die Graphen sinnvolle Ergebnisse. Beide Graphen 
haben einen Peak, bei welchem die Differenz der Thermoelemente maximal wird.
Das kommt vermutlich daher, dass der innere Sensor stärker erhitzt wurde als 
der Äußere, da dieser näher am Peltier-Element liegt. Eigenschaften wie die 
Stauchung/Streckung des Graphen oder etwa die Position hängen dabei von der 
Wärmeleitung ab. Da bei $T1$ und $T2$ Wärme schneller geleitet wird, ist ein 
gestreckter Graph zu erwarten, was hier zutriftt. Der darauffolgende asymptotische 
Abfall des Graphen ist mit des Heizvorgangs des äußeren Thermoelements zu verstehen;
Dieses heizt nach und die Temperaturdifferenz mit dem innen liegenden Sensor 
verringert sich, sodass sich der Graph auf einen Abfall einstellt.

\subsection{Dynamische Methode}