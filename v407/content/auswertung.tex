\section{Auswertung}
\label{sec:Auswertung}

Zunächst wird der Brechungsindex für die parallele , sowie die senkrechte
Polarisation berechnet. Dazu werden die Fresnelschen Formeln nach dem 
Brechungsindex $n$ umgestellt. Die Messwerte Der gemessenen Lichtintensität $I_\text{r}$
bei dem Winkel $\alpha$ sind \autoref{tab:10}  zu entnehmen. Die Berechnungen der Brechungsindizes 
finden in zwei unterabschnitten \autoref{sec:10} und \autoref{sec:11} statt und die Ergebnisse sind 
dann gemeinsam in \autoref{tab:11} zu finden.

\begin{table}[H]
  \centering
  \caption{Messwerte der Lichtreflexion}
  \label{tab:10}
  \begin{tblr}{
          colspec = {S S S S | S S S S},
          row{1} = {guard, mode = math},
          row{2} = {guard, mode = math},
      }
      \toprule
       \SetCell[c=4]{c} Parallel & & & & \SetCell[c=4]{c} Senkrecht & \\
      \alpha \, /\unit{\degree}& I \, /\unit{\milli\ampere}& \alpha \, /\unit{\degree} & I \, /\unit{\milli\ampere}&\alpha \, /\unit{\degree}& I \, /\unit{\milli\ampere}& \alpha \, /\unit{\degree} & I \, /\unit{\milli\ampere}\\
      \midrule
      85  &0.1400 &  53 & 0.0120  &  87 & 0.500 &  35 & 0.016\\
      83  &0.0900 &  51 & 0.0100  &  85 & 0.480 &  33 & 0.012\\
      82  &0.0680 &  49 & 0.0100  &  83 & 0.460 &  31 & 0.012\\
      81  &0.0440 &  47 & 0.0120  &  81 & 0.460 &  29 & 0.012\\
      80  &0.0280 &  45 & 0.0100  &  79 & 0.410 &  27 & 0.010\\
      79  &0.0180 &  43 & 0.0100  &  77 & 0.400 &  25 & 0.012\\
      78  &0.0120 &  41 & 0.0120  &  75 & 0.360 &  23 & 0.010\\
      77  &0.0074 &  39 & 0.0140  &  73 & 0.320 &  21 & 0.012\\
      76  &0.0042 &  37 & 0.0120  &  71 & 0.280 &  19 & 0.012\\
      75  &0.0030 &  35 & 0.0100  &  69 & 0.220 &  17 & 0.010 \\
      74  &0.0032 &  33 & 0.0080  &  67 & 0.200 &  15 & 0.010\\
      73  &0.0040 &  31 & 0.0100  &  65 & 0.180 &  13 & 0.010\\
      72  &0.0052 &  29 & 0.0100  &  63 & 0.120 &  11 & 0.010\\
      71  &0.0072 &  27 & 0.0120  &  61 & 0.100 &  9  & 0.010\\
      70  &0.0090 &  25 & 0.0100  &  59 & 0.100 &  7  & 0.010\\
      69  &0.0100 &  23 & 0.0100  &  57 & 0.072 &  5  & 0.010\\
      68  &0.0120 &  21 & 0.0100  &  55 & 0.060 &  3  & 0.010\\
      67  &0.0140 &  19 & 0.0100  &  53 & 0.040 &     &  \\
      66  &0.0160 &  17 & 0.0100  &  51 & 0.038 &     &  \\
      65  &0.0160 &  15 & 0.0120  &  49 & 0.030 &     &   \\
      64  &0.0160 &  13 & 0.0100  &  47 & 0.026 &     &  \\
      63  &0.0180 &  11 & 0.0100  &  45 & 0.022 &     &  \\
      61  &0.0180 &  9  & 0.0100  &  43 & 0.018 &     &  \\
      59  &0.0180 &  7  & 0.0100  &  41 & 0.016 &     &  \\
      57  &0.0180 &  5  & 0.0100  &  39 & 0.018 &     &  \\
      55  &0.0140 &               &  37 & 0.018 &     &  \\
      \bottomrule 
  \end{tblr}
\end{table}

\begin{table}[H]
  \centering
  \caption{Berechnete Brechungsindizes}
  \label{tab:10}
  \begin{tblr}{
          colspec = {S S S S | S S S S},
          row{1} = {guard, mode = math},
          row{2} = {guard, mode = math},
      }
      \toprule
       \SetCell[c=4]{c} Parallel & & & & \SetCell[c=4]{c} Senkrecht & \\
      \alpha \, /\unit{\degree}& n & \alpha \, /\unit{\degree} & n&\alpha \, /\unit{\degree}& n& \alpha \, /\unit{\degree} & n\\
      \midrule
      85  &0.1400 &  53 & 0.0120  &  87 & 0.500 &  35 & 0.016\\
      83  &0.0900 &  51 & 0.0100  &  85 & 0.480 &  33 & 0.012\\
      82  &0.0680 &  49 & 0.0100  &  83 & 0.460 &  31 & 0.012\\
      81  &0.0440 &  47 & 0.0120  &  81 & 0.460 &  29 & 0.012\\
      80  &0.0280 &  45 & 0.0100  &  79 & 0.410 &  27 & 0.010\\
      79  &0.0180 &  43 & 0.0100  &  77 & 0.400 &  25 & 0.012\\
      78  &0.0120 &  41 & 0.0120  &  75 & 0.360 &  23 & 0.010\\
      77  &0.0074 &  39 & 0.0140  &  73 & 0.320 &  21 & 0.012\\
      76  &0.0042 &  37 & 0.0120  &  71 & 0.280 &  19 & 0.012\\
      75  &0.0030 &  35 & 0.0100  &  69 & 0.220 &  17 & 0.010\\
      74  &0.0032 &  33 & 0.0080  &  67 & 0.200 &  15 & 0.010\\
      73  &0.0040 &  31 & 0.0100  &  65 & 0.180 &  13 & 0.010\\
      72  &0.0052 &  29 & 0.0100  &  63 & 0.120 &  11 & 0.010\\
      71  &0.0072 &  27 & 0.0120  &  61 & 0.100 &  9  & 0.010\\
      70  &0.0090 &  25 & 0.0100  &  59 & 0.100 &  7  & 0.010\\
      69  &0.0100 &  23 & 0.0100  &  57 & 0.072 &  5  & 0.010\\
      68  &0.0120 &  21 & 0.0100  &  55 & 0.060 &  3  & 0.010\\
      67  &0.0140 &  19 & 0.0100  &  53 & 0.040 &     &  \\
      66  &0.0160 &  17 & 0.0100  &  51 & 0.038 &     &  \\
      65  &0.0160 &  15 & 0.0120  &  49 & 0.030 &     &   \\
      64  &0.0160 &  13 & 0.0100  &  47 & 0.026 &     &  \\
      63  &0.0180 &  11 & 0.0100  &  45 & 0.022 &     &  \\
      61  &0.0180 &  9  & 0.0100  &  43 & 0.018 &     &  \\
      59  &0.0180 &  7  & 0.0100  &  41 & 0.016 &     &  \\
      57  &0.0180 &  5  & 0.0100  &  39 & 0.018 &     &  \\
      55  &0.0140 &               &  37 & 0.018 &     &  \\
      \bottomrule 
  \end{tblr}
\end{table}

\subsection{Paralleler Brechungsindex}
Die Fresnel Formel für Parallele polarisation ergibt umgestellt nach $n$
\begin{equation}
n = \sqrt{\frac{E}{2\cdot\cos{\alpha}^2} + \sqrt{\frac{E^2}{4\cdot\cos{\alpha}^4} - E \cdot \tan{\alpha}^2}}
\end{equation}
Dabei ist $E = \frac{(I+1)^2}{(I - 1)^2}$ für $\alpha < \alpha_\text{brewster}$ und  
$E = \frac{(I-1)^2}{(I + 1)^2}$ für $\alpha > \alpha_\text{brewster}$. 

\subsection{Senkrechte Polarisation}
Bei der Senkrechten polarisaton ergibt sich $n$ zu
\begin{equation}
n = \sqrt{1+\frac{4\cdot E \cos{alpha}^2}{(E-1)^2}}  , E = \sqrt{I}
\end{equation}


