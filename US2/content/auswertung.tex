\section{Auswertung}
\label{sec:Auswertung}

\subsection{Abmessung des Acrylblocks} 
Es werden die Abmessungen eines Acrylblocks Mit der Schiebleere und durch einem Amplituden Scan bestimmt.
Die ermittelten Abmessungen mit der Schiebleere sind \autoref{tab:10} zu entnehmen. Bei der Methode 
durch den A-Scan wird Der Acrylblock von oben und unten an der Stelle von jedem Loch gescannt und 
Die Laufzeit des Ultraschalls gemessen. Aus dieser laufzeit lässt sich durch \autoref{eqn:} die Strecke 
$s$ bis zur Fehlstelle bestimmen. Die Fehlstellen sind hier die Löcher im Acrylblock. Die Laufzeiten von beiden 
Seiten, sowie die umrechnung in die Abstände sind \autoref{tab:11} zu entnehmen. 
\begin{table}[H]
  \centering
  \caption{Abmessungen Schiebleere}
  \label{tab:10}
  \begin{tblr}{
          colspec = {S S S},
          row{1} = {guard, mode = math},
      }
      \toprule
      Loch & d \, \unit{\mm}& y-Position \, \unit{\mm}\\
      \midrule
      1  &   1.55 +- 0.05&   59.05 +- 0.05\\
      2  &   1.55 +- 0.05&   60.8  +- 0.05\\
      3  &   6.05 +- 0.05&   13.15 +- 0.05 \\
      4  &   5    +- 0.05&   21.8  +- 0.05\\
      5  &   4.95 +- 0.05&   30.3  +- 0.05\\
      6  &   3.05 +- 0.05&   38.7  +- 0.05\\
      7  &   3.1  +- 0.05&   46.7  +- 0.05\\
      8  &   3.05 +- 0.05&   54.75 +- 0.05 \\
      9  &   3.1  +- 0.05&   62.7  +- 0.05\\
      10  &  3.1  +- 0.05&   70.7  +- 0.05\\
      11  &  10.0 +- 0.05 &   15.3 +- 0.05 \\
      \bottomrule 
  \end{tblr}
\end{table}
Um Über die Laufzeit auf den Abstand zu schließen wird außerdem noch die Schallgeschwindigkeit im Medium benötigt.
Diese wird Über eine Lineare Ausgleichsrechnung des mit der Schiebleere ermessenen y-Abstandes der Trennlienie des 
jewailigen lochs in relation zu der Laufzeit die der Schall benötigt hat um zu dieser Trennlinie zu gelangen und wieder zurück 
zur Sonde. Damit bekommt man einen Ausdruck für die vom Schall zurückgelegte Strecke pro zeiteinheit. Die Ausgleichsrechnung ist 
in \autoref{fig:10} dargestellt.
\begin{figure}[H]
  \centering 
  \caption{Lineare Regression zur bestimmung der Schallgeschwindigkeit in Acryl}
  \label{fig:10}
  \includegraphics{build/plot1.pdf}
\end{figure}
Die parameter der Ausgleichsrechnung lauten 
\begin{align}
  m &= \qty{1374(14)}{\meter\per\second}\\
  b &= \qty{-0.0012(0.0004)}{\meter}
\end{align}
Daraus er gibt sich eine Schallgeschwindigkeit von $v_\text{s,acryl} = m \cdot 2 = \qty{2748(28)}{\meter\per\second}$

\begin{table}[H]
  \centering
  \caption{Laufzeiten im Acryl und daraus ermittelte Abmessungen.}
  \label{tab:10}
  \begin{tblr}{
          colspec = {S S S S S },
          row{1} = {guard, mode = math},
      }
      \toprule
      Loch & t_\text{Oben} \, \unit{\micro\second}& s_\text{Oben} \, \unit{\mm} & t_\text{Unten} \, \unit{\micro\second}& s_\text{Unten} \, \unit{\mm}\\
      \midrule
      1   &13.8  & & 44.0&  \\
      2   &15.3  & & 45.1&  \\
      3   &45.1  & & 10.4&  \\
      4   &39.7  & & 16.5&  \\
      5   &34.0  & & 23.6&  \\
      6   &28.6  & & 29.0&  \\
      7   &22.6  & & 34.9&  \\
      8   &16.8  & & 40.0&  \\
      9   &10.9  & & 46.8&  \\
      10  &5.1   & & 0   &  \\
      11  &0     & & 11.9&  \\
      \bottomrule
  \end{tblr}
\end{table}



