\section{Diskussion}
\label{sec:Diskussion}

Grundsätzlich ist festzuhalten, dass der A-Scan dem systematischen Fehler 
unterliegt, da die Übertragung der Sonde nicht rauschungsfrei verlief. Folglich 
sind Fehler durch falsches Ablesen zu erwarten, zudem traten auch (dezente) 
Verschiebungen einiger Maxima bei wiederholten Versuchen auf, was nicht der 
Fall sein sollte. Der B-Scan unterliegt ebenfalls dem systematischen Fehler, 
allerdings kommt der komplexe Gebrauch hinzu, was im Folgenden noch erläutert 
werden soll.

\subsection{Acrylblock}
Trotz erwarter Abweichung verhält sich diese sehr gering; ausgehend von einer 
Schallgeschwindigkeit in Acryl in der Literatur mit $2730 \unit{\meter}/\unit{\second}$
\cite{acryl} ist der experimentell bestimmte Wert von $v_\text{s,Acryl} =
\qty{2748(28)}{\meter\per\second}$ nicht weit entfernt, eher noch:
Der Literaturwert liegt im Fehlerbereich des Theoretischen. Die Abweichung 
beläuft sich auf $0.66 \%$, was unerwartet gering ist. Dementsprechend kann 
von einem gelungen Versuch gesprochen werden. Als Grund für eine derartig geringe 
Abweichung könnte der Gebrauch des Kontaktmittels von bidestilliertem Wasser 
herangezogen werden, dieses wurde in einem geregelten Maße für jede Messung 
verwendet.

\subsection{Auge}
Für diesen Teil liegen keine Referenzwerte vor. Folglich ist davon auszugehen, 
dass so wie beim Acrylblock der Fehler gering ist. Sollte es zu Abweichungen 
gekommen sein, lässt sich dazu die Hornhaut des Modells als Quelle heranzuziehen, 
diese war bereits vor dem Experiment durch Risse beschädigt. Ob jener Aspekt 
allerdings tatsächlich eine Rolle spielt oder von dem Kontaktmittel ausgeglichen 
wird bleibt unklar. Jedoch kann eine erfolgreiche Messung gemutmaßt werden, da
die bestimmten Werte zum Abstand in die Größenordnung des Modells passen.

\subsection{Tumore}
Die B-Scans sind als mittelmmäßig zu bewerten. Zwar sind Kontraste erkennbar, 
jedoch sind diese zumindest in \autoref{fig:12} sehr ungleich verteilt. Das 
ist problematisch, da die Tumore als kugelähnliche Masse ertastet wurden. 
Aufgrund der zudem mangelhaften Auflösung der Grafik sind Fehler beim Ablesen 
für die Bestimmung von der Position und Eindringtiefe nicht auszuschließen. 
Da keine Literaturwerte des Herstellers angegeben sind bleibt eine erfolgreiche 
Bestätigung zum Versuch also aus.

\subsection{Herzmodell}
Zu diesem Bereich lässt sich sagen, dass der TM-Scan nicht lief wie erwartet und 
sich der systematischer Fehler auch nicht mit einem Wechsel der Apparatur 
beheben lies. Dementsprechend sind die Herzschläge mäßig zu erkennen. Einen 
wirklichen Aufschluss liefert die Grafik jedoch nicht, außer dem Aspekt, dass 
ein Herzschlag simuliert wurde. Der ermittelte Wert liefert ein Herzschlagvolumen
von $V_\text{herz} = \qty{8.74(0.14)e-05}{\meter^3\per\second}$, was trotz der 
schlechten Grafik ein nicht sehr abwegiges Ergebnis ist, da das Herzmodell 
in der gleichen Größenordnung liegen könnte.