\section{Diskussion}
\label{sec:Diskussion}
Hier werden die ermittelten werte für Elastizitätsmodule diskutiert und mit Literaturwerten verglichen.
Für die einseitige Aufhängung ergab sich:
\begin{equation*}
    E_{rund,einseitig} = \qty{1.00(0.05)e11}{\newton\per\meter\squared}
  \end{equation*}
\begin{equation*}
    E_{eckig,einseitig} = \qty{2.00(0.1)e11}{\newton\per\meter\squared}
\end{equation*}

Für die beidseitige Aufhängung kamen die Werte zustande:
\begin{equation*}
    E_{links,rund} = \qty{8.6(0.6)e10}{\newton\per\meter\squared}
  \end{equation*}
\begin{equation*}
    E_{rechts,rund} = \qty{1.30(0.04)e11}{\newton\per\meter\squared}
\end{equation*}
\begin{equation*}
    E_{links,eckig} = \qty{5.57(0.21)e10}{\newton\per\meter\squared}
\end{equation*}
\begin{equation*}
    E_{rechts,eckig} = \qty{9.02(0.27)e10}{\newton\per\meter\squared}
\end{equation*}

Als Mittelwerte ergeben sich für den runden und eckigen Stab:
\begin{equation*}
    \overline{E_{rund}} = \qty{1.053(0.029)e11}{\newton\per\meter\squared} = \qty{105.30(2.90)}{\giga\pascal}
\end{equation*}
\begin{equation*}
    \overline{E_{eckig}} = \qty{1.153(0.012)e11}{\newton\per\meter\squared} = \qty{115.30(1.20)}{\giga\pascal}
\end{equation*}

\noindent Damit kann angenommen werden, dass die beiden Stäbe aus gleichem Material sind. 
Verglichen mit Literaturwerten kommen den experimentellen Werten den Materialien
Messing (mit einem Elastizitätsmodul von ca. $\qty{100}{\giga\pascal}$) und 
Kupfer (mit ca. $\qty{115}{\giga\pascal}$) nahe. Es wird davon ausgegangen,
dass es sich um Messing handelt. Die Dichte führt zu dem Schluss, dass es sich 
um Messing handeln könnte, die Dichten der beiden Stäbe ($\rho_{rund} = \qty{8.27}{\gram\per\cubic\centi\meter}$
$\rho_{eckig} = \qty{8.33}{\gram\per\cubic\centi\meter}$) kommen dem Literaturwert 
mit $\rho_{Messing} = \qty{8.73}{\gram\per\cubic\centi\meter}$ nahe. Jedoch
ist zu beachten, dass diese Näherung Schwächen aufweist, da die Dichte von Messing 
der von Kupfer nahe kommt. Die Farbe der Stäbe bestätigt allerdings den Verdacht,
dass es sich um das Material Messing handelt. 
Für die Elastizitätsmodule ergeben sich folgende 
Abweichungen von den Messergebnissen zu den Literaturwerten \cite{elastizitaetsmodul}:

\begin{align*}
    \symbf{\Delta}E_{rund} &= 5.3 \% \\
    \symbf{\Delta}E_{eckig} &= 15.3 \%
\end{align*}
\\

\noindent Alle den Tabellen zu entnehmenden Messwerte sind fehlerbehaftet, was 
zum Einendarauf zurückzufürhen ist, dass die Messuhren für die Auslenkung sehr 
leicht beeinflussbar waren. Ein einfacher Stoß an den Tisch mit der Messapparatur 
genügt, damit sich die Messuhr verstellt. Demnach wird dies eine Hauptquelle
für mögliche Fehler sein. Des Weiteren ist allerdings auch das Augenmaß als 
Fehlerpotential heranzuziehen, sowohl das Einstellen der Uhr auf dem Messband
als auch das Befestigen des Gewichts in der exakten Mitte des Stabes erfolgt
durch optische Abschätzung. Das Anbringen des Gewichtes in der Mitte des 
Stabes ist vor Allem problematisch, da der Stab über keine Markierung der Mitte
verfügt. Anders als bei der einseitigen Aufhängung, bei der der Haken, welcher 
das Gewicht hält, in eine Einkerbung gehängt wird. Darüber hinaus ist eine 
potentielle Fehlerquelle die Präzision der Messuhr; die beiden Messuhren waren
nicht exakt identisch, sodass die Messungen voneinander abwichen, obwohl beide 
Uhren das gleiche Ergebnis liefern sollten. Letztendlich ist anzumerken, dass 
die Länge der Stäbe, mit der gerechnet wurde, nicht die effektive Länge ist;
in den Rechnungen handelt es sich um eine Länge von $L = 0,6m$ bzw. $L = 0,59m$,
dabei handelt es sich jedoch um die exakte Länge des Stabes, der Bereich, welcher
eingeklemmt ist, wird hier nicht abgerechnet.