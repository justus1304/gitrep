% Hiermit sagst du LaTeX, dass es eine Beamer-Präsentation ist
\documentclass{beamer}

% Wichtige Pakete für Umlaute und Schriftkodierung
\usepackage[utf8]{inputenc}
\usepackage[T1]{fontenc}
\usepackage[ngerman]{babel} % Für deutsche Silbentrennung und Begriffe

% --- Titel-Informationen ---
\title{Meine erste Präsentation}
\author{Dein Name}
\date{\today} % Setzt das heutige Datum

% Füge dies nach den \usepackage-Befehlen ein
\usetheme{Madrid} % Ein sehr beliebtes Theme mit Navigationsleiste

\begin{document}

% --- 1. Die Titelfolie generieren ---
% Die \titlepage nutzt die Infos aus der Präambel
\begin{frame}
    \titlepage
\end{frame}

% --- 2. Eine Inhaltsverzeichnis-Folie (optional) ---
\begin{frame}
    \frametitle{Übersicht} % Titel für diese Folie
    \tableofcontents % Erstellt automatisch ein Inhaltsverzeichnis
\end{frame}

% --- 3. Eine normale Inhaltsfolie ---
\section{Einleitung} % Fügt einen Eintrag zum Inhaltsverzeichnis hinzu

\begin{frame}
    \frametitle{Was ist Beamer?} % Der Titel dieser Folie
    
    Das ist der Inhalt meiner ersten Folie.
    
    \begin{itemize}[-]
        \item Ein Punkt
        \item Noch ein Punkt
    \end{itemize}
\end{frame}

% --- 4. Eine weitere Folie ---
\section{Hauptteil} % Nächster Abschnitt

\begin{frame}
    \frametitle{Warum LaTeX?}
    
    Weil es für Formeln $E = mc^2$ und Textsatz unschlagbar ist.
\end{frame}

\end{document}