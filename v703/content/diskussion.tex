\section{Diskussion}
\label{sec:Diskussion}

\subsection{Kennlienie}
Bei zunehmendem Alter des Zählrohres kommt es zu einer höheren Warscheinlichkeit 
für Nachentladungen im Zählrohr. Dies macht sich in einer höheren Plateausteigung 
bemerkbar, welche einen Wert für die Güte des Zählrohres angibt.
Diese Steigung beträgt hier \qty{0.178}{\%}. Das ist eine sehr niedrige Steigung,
jenes ist auch in \autoref{fig:10} zu erkennen. Das Ergebnis passt nicht zu 
den Erwartungen, passt aber zur Theorie: Idealerweise sollte die Steigung so 
gering wie möglich sein. Als Grund könnte das Löschgas herangezogen werden, 
welche Nachentladungen trotz Alterung sehr unterdrücken. Die Messung würde somit 
auf ein sehr gutes Zählrohr hinweisen. Dabei ist festzuhalten, dass die wenigen 
aufgenommenen Werte größtenteils in einer gleichen Größenordnung von $0.2 \unit{\ampere}$
liegen. Dementsprechend liefert die Rechnung einen sehr geringen Median für die 
Steigung, was die statistische Unsicherheit der Messpunkte widerspiegeln könnte.

\subsection{Bestimmung der Totzeit}
Die Totzeit wurde auf zwei unterschiedlichen Arten berechnet.
Es ergab sich: 
\begin{equation}
    \tau_{Zwei-Quellen} = \qty{-129.0}{\micro\second} 
\end{equation}
\begin{equation}
    \tau_{Oszilloskop} = \qty{150}{\micro\second} 
\end{equation}
\noindent Dabei fällt auf, dass die theoretische Totzeit negativ ist, was 
keinen physikalischen Sinn ergibt. Als Ursache kann der Untergrund herangezogen 
werden, möglicherweise ist der Bereich mit der Probe und dem Geiger-Müller-Zählrohr 
nicht genug abgeschirmt gewesen und war weiterer Strahlung ausgesetzt. Andererseits 
wäre es möglich, dass sich die Quellen gegenseitig zu sehr beeinflusst haben oder 
statistische Schwankungen durch zu kurze Messzeit eine Fehlerquelle bilden.
Die relative Abweichung beider Werte berechnet sich durch 
\begin{equation}
    \frac{|\tau_{Oszilloskop} - \tau_{Zwei-Quellen}|}{\tau_{Oszilloskop}} \approx 16 \%.
\end{equation}
Die tatsächliche Totzeit wird demnach in diesem Bereich liegen.