\section{Theorie}
\label{sec:Theorie}

\subsection{Fouriersche Theorem}
Unter periodischen Funktionen definiert man jene, welche sich in einem bestimmten 
Maß wiederholen. So gilt für eine Funktion, welche räumlich periodisch ist die 
Gleichung
\begin{equation}
    \label{eqn:1}
    f(x+D) = f(x)
\end{equation}
mit einem Periodenabstand $T$. Diese Gleichung gilt ebenso für zeitlich periodische 
Funktionen, dabei wird das $x$ durch die Zeit $t$ ersetzt und der Periodenabstand 
mit der Periodendauer:
\begin{equation}
    \label{eqn:2}
    f(t+T) = f(t).
\end{equation}
Die beiden Funktionen, welche physikalisch am relevantesten sind, sind die 
geometrischen Funktionen Sinus und Cosinus. Mithilfe beider lässt sich das 
Fouriersche Theorem aufstellen. Darunter ist zu versetehen, dass jede Welle 
aus einer Reihe an sinusförmigen Komponenten besteht. Allgemein ist die 
Fourierreihe aus einem konstanten Term und der Summe aus zeit- und 
frequenzabhängigen Sinus- und Cosinusfunktionen, multipliziert mit deren 
Amplitude, definiert.
\begin{equation}
    \label{eqn:3}
    f(t) = \frac{a_0}{2} + \sum\limits_{n=1}^\infty a_n cos \left(n \omega t\right) + b_n sin \left(n \omega t\right)
\end{equation}
Hierbei entspricht $\omega$ der Frequenz und ist definiert als $\frac{2 \pi}{T}$.
Es ist wichtig, dass die Reihe gleichmäßig konvergiert, was der Fall ist, wenn 
$f(t)$ überall stetig ist. Sobald eine Stelle $t_0$ existiert, an der die Funktion 
nicht stetig ist, kann diese nicht mehr mithilfe der Fourierreihe angenähert 
werden. Es kommt zu einer Abweichung. Diese wird als Gibbsches Phänomen bezeichnet, 
anschaulich beschreibt dieses Überschwingungen, welche bei der Approximation 
von Funktionen mit sprunghaften Verhalten durch Fourierreihen auftreten. Diese 
Überschwingungen sind unabhängig von der Anzahl der Terme, welche die Summe ergibt.

Die Amplituden sowie der konstante Term werden Fourierkoeffizienten genannt und 
sind definiert als 
\begin{align}
    a_0 &= \frac{2}{T} \int_{-\frac{T}{2}}^{\frac{T}{2}} f(t) dt \label{eqn:4.1} \\
    a_k &= \frac{2}{T} \int_{-\frac{T}{2}}^{\frac{T}{2}} f(t) cos(k \omega t) dt \label{eqn:4.2} \\
    b_k &= \frac{2}{T} \int_{-\frac{T}{2}}^{\frac{T}{2}} f(t) sin(k \omega t) dt \label{eqn:4.3}
\end{align}
Grundsätzlich treten damit im Allgemeinen nur Vielfache der Grundfrequenz 
$\nu_1 = \frac{1}{T}$ auf, jene sind als harmonische Oberschwingungen zu bezeichnen.
Die Ermittlung der Koeffizienten wird als Fourieranalyse bezeichnet, diese kann 
in einigen Spezialfällen einfacher werden. Ist $f(t)$ eine gerade Funktion, so 
gilt $b_n = 0, \forall n \in \mathbb{N}$. Im Umkehrschluss folgt 
$a_n = 0, \forall n \in \mathbb{N}$, sofern $f(t)$ eine ungerade Funktion ist.
Werden die Amplituden einer periodischen Funktion gegen die Frequenz aufgetragen,
ergibt sich ein Linienspektrum, welches bei der Grundfrequenz am größten ist
und darauffolgend exponentiell abnimmt. Bei Funktionen, welche nicht periodisch 
sind, stellt sich lückenloses Spektrum ein.

\subsection{Fourier-Transformation}
Während mit der Berechnung der Fourierkoeffizienten die Bestimmung der einzelnen 
Komponenten einer periodischen Funktion $f(t)$ stattfindet, kann ebenso das 
ganze Spektrum bestimmt werden. Das fällt unter den Namen der Fourier-
Transformation und ist unabhängig von der Periodizität der Funktion. Die 
Transformation $g(\nu)$ wird gegeben durch
\begin{equation}
    \label{eqn:5}
    g(\nu) = \int_{-\infty}^\infty f(t) e^{i \nu t} dt.
\end{equation}
Sollte $f(t)$ doch periodisch sein, so kommt es zu einem ähnlichen Spektrum 
wie im Vorherigen geschildert. Analog verfügen nicht-periodische Funktionen 
über ein kontinuierliches Spektrum.

Die Fourier-Trafo ist umkehrbar, eine Rücktransformation nimmt die Gestalt
\begin{equation}
    \label{eqn:6}
    \tilde{g}(\nu) = \frac{1}{2 \pi} \int_{-\infty}^\infty g(\nu) e^{-i \nu t} d\nu
\end{equation}
an.

\subsection{Fehlerrechnung}
Die gemessenen Werte unterliegen Messunsicherheiten und werden demnach im
Folgenden nicht als fehlerfrei angesehen. Die Fehler entstehen bei der
Bildung der Mittelwerte durch den Fehler des Mittelwerts und bei der
Regressionsrechnung sowie der Fehlerforpflanzung durch Python.
Der Fehler des Mittelwerts ist gegeben durch 
\begin{equation}
    \begin{aligned}
        \increment \overline{x} &= \sqrt{\overline{x^2\kern-0.1em} - \overline{x}^2} \\
                            &= \frac{\sqrt{\frac{1}{N-1} \sum\limits_{i=1}^N (x_i - \overline{x})^2}}{\sqrt{N}}.
    \end{aligned}
\end{equation}

Um Fehler einzubeziehen, wird die Gauß'sche Fehlerfortpflanzung verwendet:
\begin{equation}
    \label{eqn:9}
    \increment f = \sqrt{\left(\frac{\partial f}{\partial x}\right)^2 \cdot \left(\increment x\right)^2 + \left(\frac{\partial f}{\partial y}\right)^2 \cdot \left(\increment y\right)^2 + .... + \left(\frac{\partial f}{\partial z}\right)^2 \cdot \left(\increment z\right)^2}
\end{equation}