\section{Diskussion}
\label{sec:Diskussion}

Zunächst ist festzuhalten, dass sich das Experiment sehr leicht durch äußere 
Einflüsse beeinflussen lässt. Es handelt sich um einen systematischen Fehler:
Laptops oder andere technische Geräte sowie sogar das körpereigene Magnetfeld 
beeinflussen die Messungen, wodurch sich einige schwankenden Werte besser 
erklären lassen. 

\subsection{Kennlinienschar}
Wie erwartet steigt der Sättigungsstrom proportional zum Heizstrom. Folglich 
ähnelt die Kennlinienschar der theoretischen Kennlinie. Trotzdessen sind die 
Messwerte nicht exakt, wie erhofft. Eine Abflachung ist lediglich bei einer 
Kennlinie wirklich gut zu erkennen. Jenes wird vor Allem im folgenden Absatz 
zu einem Verhängnis.

\subsection{Langmuir-Schottky'sches Raumladungsgesetz}
In diesem Bereich sollte geprüft werden, in welchem Bereich das Langmuir-
Schottky'sches Raumladungsgesetz Gültigkeit beweist. Jenes ist im Intervall
von 0 bis 100 $\unit{\volt}$ der Fall. Dabei ist jedoch anzumerken, dass für 
den Fit des Graphen andere Messwerte genutzt werden mussten. Die eigenen
experimentell ermittelten Werte führten zu keinem sinnvoll verwertbaren 
Ergebnis, weshalb die Messwerte des Tutoren zugezogen wurden.
Ziel war es, bei dem Fit der Form $a \cdot U^b$ den Parameter $b$ als $3/2$ 
zu bestimmen, was auch gelang; mithilfe des Fits ergaben die Messwerte einen 
Parameter von $b = 1.4 \pm 1.8$. Abgesehen vom verhältnismäßig sehr großen 
Fehler handelt es sich um eine Abweichung von 6.67\%.

\subsection{Anlaufstromgebiet}
Die berechnete Temperatur der Kathode beläuft sich auf $T = 2782.9 \pm 0.5 \unit{\kelvin}$.
Jenes steht nicht ganz im Einklang mit den Werten zur ermittelten Temperatur 
aus \autoref{sec:kattempmet1}. Allerdings lässt sich positiv festhalten, dass 
der Verlauf der Regression exponentiell abnimmt und die Abweichungen der Werte 
sehr minimal sind. Demnach ist die Richardson-Glechung bestätigt und der Versuch 
kann als Erfolg gerwertet werden.

\subsection{Kathodentemperatur}
Die errechneten Werte passen zur Theorie. Mit steigendem Heizstrom und steigender 
Heizspannung erhöht sich ebenso die Temperatur der Kathode. Auch dieser Wert 
ist fehlerbehaftet. Jedoch kann nichts mit Literaturwerten sondergleichen 
verglichen werden, somit ist keine Aussage über die Genauigkeiten der Messwerte 
zu treffen. Es lässt sich jedoch vermuten, dass es zu Verlusteffekten innerhalb 
der Apparatur kam. 

\subsection{Austrittsarbeit}
Die errechnete Austrittsarbeit von Wolfram beträgt $\overline{W_A} = 5.9236 \pm 0.1110 \, \unit{\electronvolt}$.
Ausgehend von einem Literaturwert von $\overline{W_{A,Lit}} = 4.55 \, \unit{\electronvolt}$ 
\cite{austrittsarbeitwolfram} entspricht das einer Abweichung von $23.189\%$.
Diese Abweichung ist mit den fehlerbehafteten Temperaturen zu erklären. Da diese 
dem systematischen Fehler unterliegen, zieht sich dieser in der Rechnung zur 
Bestimmung der Austrittsarbeit fort.