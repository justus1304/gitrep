\section{Theorie}
\label{sec:Theorie}

Wird ein Körper in einem elastischen Maße deformiert, so lässt sich der lineare
Zusammenhang der Größen $\sigma$ (die angreifende Spannung am Objekt) und der
Deformation ($\Delta L/L$) mithilfe des Hookeschen Gesetzes. Hinzu kommt der
Elastizitätsmodul $E$, eine Materialkonstante eines Werkstoffes. Alles zusammen 
wird beschrieben durch 
\begin{equation}
    \sigma = E \frac{\Delta L}{L}.
\end{equation}
Die Funktion $D(x)$ setzt sich aus zwei Teilen zusammen; Der Auslenkung ohne
Gewicht, eine Nullmessung ($D_0 (x)$) und der Messung mit Gewicht ($D_M (x)$).
Beides zusammen ergänzt sich zu:
\begin{equation}
    D(x) = D_M (x)-D_0 (x).
\end{equation}
\subsection{Biegung bei einseitiger Befestigung}
Bei den beiden Stäben, unabhängig davon, ob eckig oder rund (von der Grundfläche
her), lässt sich die Durchbiedung $D$ durch eine Funktion in Abhängigkeit von x
darstellen. Bei einem Stab der Länge L wird für die Funktion eine horizontale
Komponente (x), der Querschnitt Q und eine nach unten gerichtete Kraft benötigt.

\noindent Durch die diese Kraft $F$ am Stab entsteht ein Trägheitsmoment, welches am 
Punkt x ansetzt, dieses wird beschrieben durch:
\begin{equation}
    M_F = F(L-x).
\end{equation}
Dieses Drehmoment bewirkt die Streckug und Stauchung an der Ober-und Unterseite
des Stabes. Die durch diese Spannungen verursachten Drehmomente sind wiederum
von gleicher Größe, jedoch in entgegengesetzte Richtung um die neutrale Faser 
des Stabes, welcher keine Veränderungen vernimmt. Dieses Drehmoment im Inneren 
ist definiert durch 
\begin{equation}
    M_\sigma = \int_Q y \sigma (y) dq.
\end{equation}
Variable y steht dabei für für den Abstand zum Zentrum des
Stabes, dem neutralen Faser, welcher Spannungsfrei ist. $\sigma$ ist die
angreifende Spannung, während dq als Flächenelement von y gedeutet wird.
Um die Durchbiegungsfunktion $D(x)$ zu erlangen, muss eine Gleichsetzung der
beiden Drehmomente erfolgen: $M_F = M_\sigma$. Damit ergibt sich letztendlich:
\begin{equation}
    D(x)=\frac{F}{2EI} (Lx^2-\frac{x^3}{3})
\end{equation}
mit I, dem Flächenträgheitsmoment, gegeben als 
\begin{equation}
    I = \int_Q y^2 dq(y).
\end{equation}
\subsection{Biegung bei beidseitiger Befestigung}
Neben der Bestimmung durch einseitige Bespannung, kann die Auslenkung auch bei
beidseitiger Befestigung bestimmt werden. Das Moment im Inneren beträgt auf einer
Seite des Stabes:
\begin{equation}
    M_F = -\frac{F}{2} x \quad , 0 \leq x \leq L/2
\end{equation}
und auf der Anderen:
\begin{equation}
    M_F = -\frac{F}{2} (L-x) \quad , L/2 \leq x \leq L
\end{equation}
Die Bestimmung für $D(x)$ erfolgt analog zu der Vorherigen. Es ergibt sich
\begin{equation}
    \label{eqn:linksDx}
    D(x)=\frac{F}{48EI} (3L^2-4x^3)
\end{equation}
für die Seite im Bereich $0 \leq x \leq L/2$ und 
\begin{equation}
    \label{eqn:RechtsDx}
    D(x)=\frac{F}{48EI} (4x^3-12Lx^3+9L^2x-L^3)
\end{equation}
folgerlich im Bereich $L/2 \leq x \leq L$. Beides wieder mit dem Flächenträgheitsmoment I.


\cite{sample}
