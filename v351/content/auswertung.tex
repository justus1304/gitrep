\section{Auswertung}
\label{sec:Auswertung}

\subsection{Fourier-Analyse}
Bei der Furier-Analyse werden Am Oberwellengeneratur einmal die Rechtecksfunktion,
die Dreiecksfunktion und die Sägezahnfunktion generiert. Am Amperemeter werden 
die Amplituden bei den jewailigen frequenzen der oberwelle 
abgeesen, um so auf die abhängigkeit der Spaannung der n-ten Oberwelle von n schließen zu können.
\subsubsection{Rechteckspannung}
\begin{table}[H]
    \centering
    \caption{Amplituden der Oberschwingungen Rechtecksfunktion}
    \label{tab:j1}
    \begin{tblr}{
        colspec = {S S S},
        row{1} = {guard, mode=math},
      }
    \toprule
    f (\unit{\hertz}) &  \\
    \midrule
    10 & 17.8\\
    30  & 9.01\\
    50  & 4.61\\
    70  & 1.81\\
    90  &-0.19\\
    110 &-1.79\\
    130 &-3.39\\
    150 &-4.59\\
    170 &-6.19\\
    190 &-8.19\\
    \bottomrule
    \end{tblr}
\end{table}

\begin{figure}
    \centering
    \caption{}
    \includegraphics{rechteck.pdf}
\end{figure}

\subsubsection{Dreiecksfunktion}
\begin{figure}
    \centering
    \caption{}
    \includegraphics{dreieck.pdf}
\end{figure}

\subsubsection{Sägezahnfunktion}
\begin{figure}
    \centering
    \caption{}
    \includegraphics{saegezahn.pdf}
\end{figure}


\subsection{Fourier-Synthese}