\section{Diskussion}
\label{sec:Diskussion}

Bei dem Vergleich der experimentell bestimmten Verdampfungswärme und dem
Literaturwert lässt sich eine Abweichung von 21,1\% feststellen. Während 
sich dieser auf $L = \num{40.8} \si{\kilo\joule\per\mole}$ beläuft, wird
bei dem Experiment der Wert $L = \num{3.22 +- 0.02 e4} \unit{\joule\per\mol}$ 
erreicht \cite{verdampfungswaerme}.
\noindent Eine derartige Abweichung war zu erwarten, schließlich wurde 
angenommen, dass die Verdampfungswärme konstant sei. Darüber hinaus erfolgen 
sämtliche Veränderungen an der Installation per Hand, wodurch eine präzise 
Änderung der Kühlung oder Ähnlichem nicht garantiert werden kann. Grundsätzlich 
kann der Fall einer undichten Stelle nicht vollständig ausgeschlossen werden.
Bei der Dokumentation der Messdaten muss ebenfalls berücksichtigt werden, dass 
alle Werte per Augenmaß aufgenommen wurden, das wird vor Allem im Zweiten Teil 
des Experiments zum Verhängnis, da das Manometer aus genauer Frontansicht 
betrachtet werden muss. Nur so kann eine Temperatur bei einem (verhältnismäßig)
genauen Druck abgelesen werden. Weiterhin ist bei dem zweiten Teil des Experiments 
wichtig, dass nur eine Lösung physikalisch relevant ist: Für $L(T)$ kann 
durch die Wurzel ein negatives Ergebnis entstehen. Da die Verdampfungswärme 
eine stets positive Energie ist, welche für den Übergang von flüssig zu 
gasförmig  aufgewendet wird, hat nur das positive Wurzelzeichen einen Sinn. 
Wäre es negativ, so würde das bedeuten, dass die Energie bei der Evaporation 
freigesetzt wird, das widerspräche allerdings dem physikalischen Prozess.