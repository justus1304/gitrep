\section{Diskussion}
\label{sec:Diskussion}

Zunächst ist festzuhalten, dass jegliche bestimmte Größen nicht mit Literaturwerten 
zu vergleichen sind. Die tatsächlichen Werte der Wiederstände sind unbekannt.
Jedoch liegen die Werte der Mittelwertbildung größtenteils verhältnismäßig nah
beieinander, was auf eine gute Genauigkeit bei den Messungen hindeutet. 
Lediglich bei der Berechnung der Induktivität des "Wert16" wichen die berechneten
Werte des Widerstands $R_x$ mit $R_{x1} = 464, R_{x2} = 422$ und $R_{x3} = 394$ 
entsprechend stark voneinander ab.
Es ist anzunehmen, dass diese Abweichung durch eine Ungenauigkeit am Widerstand 
zurückzuführen ist, da auch bei mehrfacher Durchführung des Versuches mit dem 
erwähnten Widerstand große Abweichungen zustande kamen.
Als Fehler für die Bauelemente wurde bei Wiederständen, Kapazitäten, sowie 
Induktivitäten jeweils ein Fehler von $2 \%$ angenommen. Dieser Fehler wurde
auf Grundlage vorhergeganger Berechnungen abgeschätzt, sodass dieser annähernd
mit den Abweichungen zusammenpasst.
