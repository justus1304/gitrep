\section{Auswertung}
\label{sec:Auswertung}

\subsection{Verifizierung der Bragg-Bedingung.}
\begin{figure}
    \centering
    \caption{Messwerte zur Bragg-Überprüfung.}
    \label{fig:brag}
    \includegraphics{build/bragg.pdf}
\end{figure}
\noindent Anhand von \autoref{fig:brag} lässt sich das Maximum ziemlich genau bei 
$\theta = 28\unit{\degree}$ erkennen. Da das Medium, an welchen 
die Photonen reflektiert werden in einem Winkel von $14\unit{\degree}$ stand, 
ist das Maximum bei dem Winkel erwartungsgemäß. (Einfallswinkel = Ausfallswinkel).

\subsection{Das Emissionsspektrum einer Cu-Roentgenroehre}
In \autoref{fig:spektrum} sind die Messwerte geplottet. Eingezeichnet sind auch
der Bremsberg sowie die $K_\alpha $ und $K_\beta $-Linien. Der Grenzwinkel
bestimmt sich durch Ablesen zu 
\begin{equation}
    \theta_\text{G} = \qty{5.7(0.1)}{\degree}
\end{equation}
daraus ergibt sich nach \autoref{eqn:E} und \autoref{eqn:B}
eine minimale Wellenlänge, bzw eine maximale Energie.
 \begin{align}
    \lambda_\text{min} &= \qty{40(0.7)}{\pico\meter}\\
    E_\text{max} &= \qty{49.7(0.7)}{\kilo\electronvolt}
 \end{align}
\begin{figure} [H]
    \centering
    \caption{Emmisionsspektrum einer Cu-Röhre.}
    \label{fig:spektrum}
    \includegraphics{build/wertejust.pdf}
\end{figure}
\noindent Um nun das Auflösevermögen der Apparatur zu bestimmen, werden die K-Linien zusätzlich
vergrößert, wie in \autoref{fig:zoom} dargestellt. Des Weiteren wurden die
waagerechten Linien auf Höhe der beiden Halbwertsbreiten der K-Linien eingezeichnet.
Daraus ergeben sich jeweils zwei Winkel aus den Schnittpunkten mit der 
$K_\alpha$- und $K_\beta$-Linie.
Es ergibt sich:
\begin{align}
    \text{K$_\alpha$: } \theta_{\text{min},\alpha} &= \qty{22(0.1)}{\degree}   \\
                          \theta_{\text{max},\alpha} &= \qty{22.7(0.1)}{\degree} \\
    \text{K$_\beta$: }  \theta_{\text{min},\beta} &= \qty{19.8(0.1)}{\degree}  \\
                          \theta_{\text{max},\beta} &= \qty{20.4(0.1)}{\degree}
\end{align}
\begin{figure}[H]
    \centering
    \caption{Zoom auf $K_{\alpha} $und $ K_{\beta}$ Lienien des Emissionsspektrums.}
    \label{fig:zoom}
    \includegraphics{build/zoom.pdf}
\end{figure}
\noindent Durch \autoref{eqn:E} werden daraus erneut die Energien ermittelt. Das Auflösungsvermögen
lässt sich dann durch folgende Gleichung bestimmen.
\begin{equation}
    A = \frac{E}{\Delta E}
\end{equation}
Die Ergebnisse sind in \autoref{tab:auf} aufgetragen.
\begin{table}[H]
  \centering
  \caption{Berechnete Energien und Energiedifferenzen.}
  \label{tab:auf}
  \begin{tblr}{
          colspec = {S S S S S S},
          row{1} = {guard, mode = math},
      }
      \toprule
      \text{Linie} & E/\unit{\kilo\electronvolt}& \Delta E /\unit{\kilo\electronvolt}&A\\
      \midrule
      $\alpha$ & 8.10 +- 0.04 & 0.26+-0.07 & 31+-8\\
      $\beta$ & 8.95 +- 0.04 & 0.26+-0.07 & 34+-9\\
      \bottomrule
  \end{tblr}
\end{table}
\noindent Aus \autoref{eqn:a}, \autoref{eqn:b} und \autoref{eqn:c} lassen sich mit den
berechneten Werten für die Energien die Abschirmkonstanten berechnen. Bei der 
Berechnung von $\sigma_1$ wird für die Energie $E_\text{K,abs} = \qty{8.987}{\kilo\electronvolt}
$ als Theoriewert angenommen, da dieser nicht experimentell bestimmt wurde. Die berechneten Abschirmkonstanten befinden sich in \autoref{tab:ab}
\begin{table}[H]
  \centering
  \caption{Berechnete Abschirmkonstanten.}
  \label{tab:ab}
  \begin{tblr}{
          colspec = {S S S S S S},
          row{1} = {guard, mode = math},
      }
      \toprule
      \sigma & \text{experimenteller Wert} & \text{theoretischer Wert}\\
      \midrule
       \sigma 1 & \text{n.A.} & 3.29\\
       \sigma 2& 12.85+-0.36 & 12.14\\
       \sigma 3& 24.05+-2.67 & 24.04\\
      \bottomrule
  \end{tblr}
\end{table}


\subsection{Das Absorptionsspektrum}
In den folgenden Abbildungen sind die Messwerte für das Absorptionsspektrum graphisch dargestellt.
Daraus wird der Braggwinkel $\theta_\text{k,exp}$ bestimmt. Zur besseren Übersicht ist der Mittelwert der Messwerte eingezeichnet.
Aus dem Braggwinkel werden, wie schon im letzten Unterabschnitt, die Energien
bestimmt und letztendlich mit \autoref{sigma} die Abschirmkonstanten. 
Die Ergebnisse sind in \autoref{tab:abs} dargestellt.
\begin{table}[H]
  \centering
  \caption{Berechnete Braggwinkel,Absorptionsenergie und Abschirmkonstanten.}
  \label{tab:abs}
  \begin{tblr}{
          colspec = {S S S[table-format=2.2] S S[table-format=2.2] S},
          row{1} = {guard, mode = math},
      }
      \toprule
      \text{Material} & \theta_\text{k,exp} / \unit{\degree} & \theta_\text{k,lit} / \unit{\degree} &
      E_\text{k,exp} / \unit{\kilo\electronvolt} & E_\text{k,lit} / \unit{\kilo\electronvolt} 
      & \sigma_{K,exp} & \sigma_{K,lit}\\
      \midrule
      Sr & 10.95 +- 0.1 & 11.01 & 16.19 +-0.15 & 16.12 & 3.50 +- 0.09 & 3.61 \\
      Ga & 17.12 +- 0.1 & 17.30 & 10.46 +-0.05 & 11.11 & 3.26 +- 0.08 & 3.41 \\
      Br & 13.1  +- 0.1 & 13.20 & 13.58 +-0.11 & 13.48 & 3.41 +- 0.06 & 3.83 \\
      Zn & 18.45 +- 0.1 & 18.60 & 9.74  +-0.05 & 9.65  & 4.26 +- 0.06 & 3.37 \\
      Zr & 9.85  +- 0.1 & 9.84 & 17.99 +-0.17 & 18.01 & 3.62 +- 0.17 & 3.65 \\
      \bottomrule
  \end{tblr}
\end{table}
\begin{figure}[H]
    \centering
    \caption{Ermittlung des Braggwinkels von Strontium.}
    \label{fig:zink}
    \includegraphics{build/strontium.pdf}
\end{figure}

\begin{figure}[H]
    \centering
    \caption{Ermittlung des Braggwinkels von Gallium.}
    \label{fig:}
    \includegraphics{build/gallium.pdf}
\end{figure}

\begin{figure}[H]
    \centering
    \caption{Ermittlung des Braggwinkels von Zink.}
    \label{fig:}
    \includegraphics{build/zink.pdf}
\end{figure}

\begin{figure}[H]
    \centering
    \caption{Ermittlung des Braggwinkels von Zirkonium.}
    \label{fig:}
    \includegraphics{build/zr.pdf}
\end{figure}

\begin{figure}[H]
    \centering
    \caption{Ermittlung des Braggwinkels von Brom.}
    \label{fig:}
    \includegraphics{build/brom.pdf}
\end{figure}


\subsection{Bestimmung der Rydbergkonstante}
Wird der Theorie nach dem Moseleyschen Gesetz gefolgt, besteht ein linearer 
Zusammenhang zwischen $\sqrt{E_K}$ und der effektiven Kernladung $z_{eff}$.
Der Proportionalitätsfaktor entspricht der Wurzel der Rydbergenergie, welcher 
mithilfe eines linearen fit der Form $y = mx + b$ bestimmt werden kann.
Mithilfe aller gegebenen Größen ergeben sich die Parameter zu
\begin{align*}
    m =& \qty{3.43 (0.16)}{\sqrt{eV}} \\
    b =& \qty{8.79 (5.02)}{\sqrt{eV}}.
\end{align*}
Da die Rydbergkonstante Faktor $m^2$ entspricht, beläuft sich die Rydbergenergie 
auf 
\begin{equation*}
    R_{\infty,\text{exp}} = 11.76\,\text{eV}.
\end{equation*}
Der graphische Verlauf befindet sich in \autoref{fig:ryd}
\begin{figure}[H]
    \centering
    \label{fig:ryd}
    \includegraphics{build/ryd.pdf}
    \caption{Lineare Regression der Wurzel der Absorptionsenergien.}
\end{figure}