\section{Diskussion}
\label{sec:Diskussion}
\subsection{Kennlienie}
Bei zunehmendem Alter des Zählrohres kommt es zu einer höheren warscheinlichkeit 
für Nachentladungen im Zählrohr. Dies macht sch in einer Höheren Plateausteigung 
bemerkbar, welche so ein Wert für die Güte des Zählrohres angibt.
Diese steigung beträgt hier \qty{0.178}{\%}. Das ist eine sehr niedrige Steigung, was auch 
in \autoref{fig:10} zu erkennen ist. Dies ist verwunderlich und die Steigung 
dürfte so Klein vermutlich nicht sein.

\subsection{Bestimmung der Todzeit}
Die Todzeit wurde auf die zwei unterschiedlichen Arten berechnet.
Es ergaben sich die Zeiten 
\begin{equation}
    \tau_{Zwei-Quellen} = \qty{129.0}{\micro\second} 
\end{equation}
\begin{equation}
    \tau_{Oszilloskop} = \qty{150}{\micro\second} 
\end{equation}
Das ablesen vom oszloskop hat eine Abweichung von $16 \%$ von dem 
ermittelten Wert durch die zwei Quellen Mehode, die tatsächliche 
Todzeit wird also in dem Bereich liegen.

