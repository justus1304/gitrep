\section{Auswertung}
\label{sec:Auswertung}

\subsection{Strom-Spannungs Kennlienie}
Es werden die Strom-Spannungs Kennlinien für das violette Spektrum der 
Hg-Lampe zunächst bei voller und danach bei halber Intensität in einem 
Diagramm aufgetragen. Die Messwerte sind in \autoref{tab:10} und \autoref{tab:11} 
zu finden. Die dazugehörige graphische Darstellung befindet sich in \autoref{fig:10}.

\begin{figure}[H]
    \centering
    \caption{Strom-Spannungs Kennlinie: Violett.}
    \label{fig:10}
    \includegraphics{build/violettUI.pdf}
\end{figure}    

\begin{table}[H]
  \centering
  \caption{Messwerte violettes Farbspektrom volle Intensität.}
  \label{tab:10}
  \begin{tblr}{
          colspec = {S S S S S S},
          row{1} = {guard, mode = math},
      }
      \toprule
      U/\unit{\volt}&I/\unit{\milli\ampere}&U/\unit{\volt}&I/\unit{\milli\ampere}&U/\unit{\volt}&I/\unit{\milli\ampere}\\
      \midrule
      -1.168 & 0.000  & -1.065 & 0.054 & -0.650 & 1.000\\
      -1.160 & 0.004  & -1.060 & 0.058 & -0.600 & 1.200\\
      -1.155 & 0.006  & -1.055 & 0.062 & -0.550 & 1.600\\
      -1.150 & 0.010  & -1.050 & 0.064 & -0.500 & 2.000\\
      -1.145 & 0.012  & -1.045 & 0.068 & -0.400 & 2.600\\
      -1.140 & 0.014  & -1.040 & 0.072 & -0.300 & 3.400\\
      -1.135 & 0.014  & -1.035 & 0.076 & -0.200 & 4.400\\
      -1.130 & 0.016  & -1.030 & 0.078 & -0.100 & 5.400\\
      -1.125 & 0.018  & -1.025 & 0.084 & 0.000  & 6.200\\
      -1.120 & 0.022  & -1.020 & 0.088 & 0.100  & 7.200\\
      -1.115 & 0.024  & -1.015 & 0.094 & 0.200  & 8.200\\
      -1.110 & 0.028  & -1.010 & 0.098 & 0.300  & 9.200\\
      -1.105 & 0.032  & -1.005 & 0.100 & 0.400  & 10.00\\
      -1.100 & 0.034  & -1.000 & 0.100 & 0.500  & 12.00\\
      -1.095 & 0.038  & -0.950 & 0.160 & 1.000  & 14.00\\
      -1.090 & 0.042  & -0.900 & 0.240 & 1.500  & 20.00\\
      -1.085 & 0.042  & -0.850 & 0.340 & 5.000  & 50.00\\
      -1.080 & 0.044  & -0.800 & 0.460 & 10.00  & 70.00\\
      -1.075 & 0.048  & -0.750 & 0.640 & 15.00  & 80.00\\
      -1.070 & 0.052  & -0.700 & 0.820 & 19.14  & 90.00\\
      \bottomrule
  \end{tblr}
\end{table}
\noindent Erkennbar ist, dass der Strom proportional zur Intensität ist. Wird 
der Spalt verkleinert, dass nur noch die Hälfte der Lichtintensität an der 
Photozelle ankommt, so ergibt sich die gleiche Kurve des Photostroms, nur um ca. die 
Hälfte gestaucht. 

\begin{table}[H]
  \centering
  \caption{Messwerte violettes Farbspektrom halbe Intensität.}
  \label{tab:11}
  \begin{tblr}{
          colspec = {S S S S S S},
          row{1} = {guard, mode = math},
      }
      \toprule
      U/\unit{\volt}&I/\unit{\milli\ampere}\\
      \midrule
      -1.17  & 0 \\
      -1.00  & 0.058 \\
      -0.50  & 1 \\
      0      & 3 \\
      1      & 7.8 \\
      3      & 16 \\
      5      & 22 \\
      8      & 30 \\
      11     & 34 \\
      15     & 40 \\
      19.15  & 44 \\
      \bottomrule
  \end{tblr}
\end{table}

\subsection{Bestimmung der Grenzspannung $U_\text{g}$}
\label{sec:gruen}
Für die Bestimmung von $U_\text{g,regr}$ wird die Spannung gegen die 
Wurzel des Photostroms $\sqrt{I}$ aufgetragen. Es wird im linearen 
Bereich eine Ausgleichsrechnung durchgeführt, um auf die genaue Grenzspannung 
zu schließen. Die Messwerte zur Bestimmung der Grenzspannung des violetten Lichts 
sind jene aus \autoref{tab:10}. Des Weiteren sind die Messwerte und Regression
in \autoref{fig:12} graphisch dargestellt.

\begin{figure}[H]
    \centering
    \caption{$U-\sqrt{I}$ Werte mit Ausgleichsrechnung der Violetten Linie}
    \label{fig:12}
    \includegraphics{build/violettUsqrtI.pdf}
\end{figure} 
\noindent Die Parameter der Ausgleichsgeraden ergeben sich zu 
\begin{align}
    a = & \qty{2.06(0.01)e-3}{\ampere}\\
    b = & \qty{2.42(0.01)e-3}{\ampere}
\end{align}
Damit ergibt sich die Grenzspannung durch Regression zu 
\begin{equation}
    U_\text{g,regr,violett} = \qty{1.175(0.01)}{\volt}
\end{equation}

\subsection{Grenzspannung des grünen Farbspektrums}
Die Grenzspannung der grünen Linie Lässt sich analog zu \autoref{sec:gruen} 
bestimmen. Die Messwerte befinden sich in \autoref{tab:gruen} sowie die 
Graphische Darstellung inklisive Ausgleichsgeraden in \autoref{fig:gruen}.
\begin{table}[H]
  \centering
  \caption{Messwerte grünes Farbspektrum.}
  \label{tab:gruen}
  \begin{tblr}{
          colspec = {S S S S S S},
          row{1} = {guard, mode = math},
      }
      \toprule
      U/\unit{\volt}&I/\unit{\milli\ampere}\\
      \midrule
      -0.579  & 0.000\\
      -0.575 &  0.002\\
      -0.570 &  0.004\\
      -0.565 &  0.006\\
      -0.560 &  0.006\\
      -0.534 &  0.014\\
      -0.518 &  0.018\\
      -0.497 &  0.030\\
      -0.480 &  0.036\\
      -0.470 &  0.042\\
      \bottomrule
  \end{tblr}
\end{table}
\begin{figure}[H]
    \centering
    \caption{$U-\sqrt{I}$ Werte mit Ausgleichsrechnung des grünen Lichts.}
    \label{fig:gruen}
    \includegraphics{build/gruen.pdf}
\end{figure} 
\noindent Die Parameter der Ausgleichsgeraden ergeben sich zu 
\begin{align}
    a = & \qty{1.58(0.13)e-3}{\ampere}\\
    b = & \qty{0.95(0.07)e-3}{\ampere}
\end{align}
Damit ergibt sich die Grenzspannung durch Regression zu 
\begin{equation}
    U_\text{g,regr,gruen} = \qty{1.175(0.01)}{\volt}
\end{equation}


\subsection{Grenzspannung des türkisen Farbspektrums}
Die Messwerte zur türkisen Linie befinden sich in \autoref{tab:11} sowie die 
Graphische Darstellung inklisive Ausgleichsgeraden in \autoref{fig:tuerkis}.
\begin{table}[H]
  \centering
  \caption{Messwerte türkises Farbspektrum.}
  \label{tab:11}
  \begin{tblr}{
          colspec = {S S S S S S},
          row{1} = {guard, mode = math},
      }
      \toprule
      U/\unit{\volt}&I/\unit{\milli\ampere}\\
      \midrule
      -0.922  & 0.000\\
      -0.880  & 0.002\\
      -0.834  & 0.004\\
      -0.798  & 0.006\\
      -0.764  & 0.008\\
      -0.711  & 0.010\\
      -0.680  & 0.012\\
      -0.648  & 0.014\\
      -0.622  & 0.016\\
      -0.605  & 0.018\\
      -0.522  & 0.020\\
      \bottomrule
  \end{tblr}
\end{table}
\begin{figure}[H]
    \centering
    \caption{$U-\sqrt{I}$ Werte mit Ausgleichsrechnung des türkisen Lichts.}
    \label{fig:tuerkis}
    \includegraphics{build/tuerkies.pdf}
\end{figure} 
\noindent Die Parameter der Ausgleichsgeraden ergeben sich zu 
\begin{align}
    a = & \qty{0.32(0.03)e-3}{\ampere}\\
    b = & \qty{0.33(0.02)e-3}{\ampere}
\end{align}
Damit ergibt sich die Grenzspannung durch Regression zu 
\begin{equation}
    U_\text{g,regr,tuerkis} = \qty{-1.0(0.12)}{\volt}
\end{equation}

\subsection{Grenzspannung des gelben Farbspektrums}
Auch für die gelbe Linie wird noch einmal die Grenzspannung berechnet.
Die Messwerte zur gelben Linie befinden sich in \autoref{tab:12} und die 
graphische Darstellung inklisive Ausgleichsgeraden in \autoref{fig:gelb}.
\begin{table}[H]
  \centering
  \caption{Messwerte Gelbes Farbspektrom}
  \label{tab:12}
  \begin{tblr}{
          colspec = {S S S S S S},
          row{1} = {guard, mode = math},
      }
      \toprule
      U/\unit{\volt}&I/\unit{\milli\ampere}\\
      \midrule
      -0.525 &  0.000\\
      -0.501 &  0.002\\
      -0.472 &  0.004\\
      -0.426 &  0.006\\
      -0.390 &  0.008\\
      -0.389 &  0.010\\
      -0.380 &  0.012\\
      -0.370 &  0.014\\
      -0.357 &  0.016\\
      -0.342 &  0.018\\
      -0.322 &  0.020\\
      \bottomrule
  \end{tblr}
\end{table}
\begin{figure}[H]
    \centering
    \caption{$U-\sqrt{I}$ Werte mit Ausgleichsrechnung des gelben Lichts.}
    \label{fig:gelb}
    \includegraphics{build/gelb.pdf}
\end{figure} 
\noindent Die Parameter der Ausgleichsgeraden ergeben sich zu 
\begin{align}
    a = & \qty{0.62(0.04)e-3}{\ampere}\\
    b = & \qty{0.34(0.01)e-3}{\ampere}
\end{align}
Damit ergibt sich die Grenzspannung durch Regression zu 
\begin{equation}
    U_\text{g,regr,gelb} = \qty{-0.55(0.05)}{\volt}
\end{equation}

\subsection{Bestimmung der Planck-Konstante und Austritsarbeit}
\label{sec:planckWa}
\noindent In \autoref{fig:aa} sind die Grenzspannungen gegen die 
Frequenz des jeweiligen ausgestrahlten Lichts der HG-Lampe aufgetragen. Die
Messwerte zu den Frequenzen der Farben befinden sich in \autoref{sec:aa}. 
\begin{figure}[H]
    \centering
    \caption{Grenzspannungen als Funktion der Frequenz.}
    \label{fig:aa}
    \includegraphics{build/aa.pdf}
\end{figure} 
\noindent Nach \autoref{eqn:5} gibt die Steigung der Ausgleichsgeraden 
$a = \frac{h}{e}$ das planck'sche Wirkungsquantum geteilt durch die 
Elementarladung an und der Wert des y-Achsen Abschnitts $b =  W_\text{A}$
die Austrittsarbeit. Somit folgen für $h$ und $W_\text{A}$
\begin{align}
    h =& \frac{a}{e} = \qty{6.3(1.1)e-34}{\joule\second}\\
    W_\text{A} =& \qty{1.5(0.4)}{\eV}.
\end{align}