\section{Theorie}
\label{sec:Theorie}

Wird ein Körper aus seinem Temperaturgleichgewicht gebracht, so wird die Energie 
in Form von Wärme zum kühleren Teil transportiert. In diesem Versuch geht es 
um den Wärmetransport durch Wärmeleitung in einem Stab. Dieser habe die Länge 
$L$, Querschnitt $A$, Dichte $\rho$ und jeweils spezifische Wärme $c$. Kommt 
es nun zu einem Wärmetransport, fließt die Wärmemenge $dQ$ durch die Querschnittsfläche
$A$. Diese Wärmemenge lässt sich beschreiben durch:
\begin{equation}
    dQ = - \kappa A \frac{\partial T}{\partial x} dt.
\end{equation}
Das $\kappa$ ist die materialabhängige Wärmeleitfähigkeit. Mithilfe der 
Kontinuitätsgleichung folgt daraus eine Differentialgleichung, welche als 
Wärmeleitungsgleichung bezeichnet wird:
\begin{equation}
    \frac{\partial T}{\partial t} = \frac{\kappa}{\rho c} \frac{\partial ^2 T}{\partial x^2}
\end{equation}
Diese Gleichung gibt Aufschluss darüber, wie sich die Temperaturverteilung 
über Raum und Zeit hinweg entwickelt. Der unabhängige Faktor $\frac{\kappa}{\rho c}$
steht für die Temperaturleitfähigkeit, sie wiederum gibt Aufschluss darüber, 
wie schnell ein Temperaturungleichgewicht ausgeglichen wird. Die Lösung dieser 
DGL für einen langen Stab, welcher abwechselnd erhitzt und abgekühlt wird, ist 
eine Temperaturwelle, gegeben durch die Gleichung 
\begin{equation}
    T(x,t)=T_{max}e^{-\sqrt{\frac{\omega \rho c}{2\kappa}}x}cos(\omega t - \sqrt{\frac{\omega \rho c}{2\kappa}}x).
\end{equation}
$T_{max}$ ist die Maximalamplitude und $\omega$ die Kreisfrequenz. Nach der 
Dispersionsrelation
\begin{equation}
    \omega = v |k|
\end{equation}
ergibt sich für die Phasengeschwindigkeit
\begin{equation}
    v = \frac{\omega}{k} = \sqrt{\frac{2\kappa \omega}{\rho c}}.
\end{equation}
Die Wärmeleitfähigkeit ist gegeben durch 
\begin{equation}
    \kappa = \frac{\rho c(\Delta x)^2}{2 \Delta t \, ln(\frac{A_{nah}}{A_{fern}})}
\end{equation}
mit $\Delta = x_{fern} - x_{nah}$ und den Amplituden an diesen Stellen, sowie 
$\Delta t$, der Phasendifferenz.