\section{Auswertung}
\label{sec:Auswertung}

\subsection{Kennliene der Hochvakuumdiode und Sättigungsströme}
Die verschiedenen Werte für die Heizströme $I_H$ und Saugspannungen $U_S$ 
befinden sich in den Tabellen \autoref{tab:1} und \autoref{tab:2}. Tabelle 
1 stellt die Werte für die erste Messung bei einem Heizstrom von $I_H = 1,8 \unit{\ampere}$
dar. Die restlichen Messungen mit mehr Messdaten befinden sich in Tabelle 2.
\begin{table}[H]
    \centering
    \caption{Messdaten für die Kennlinien.}
    \label{tab:1}
    \begin{tblr}{
        colspec = {Q[l,si={table-format=3.0}] Q[l,si={table-format=1.3}]},
        row{1} = {font=\bfseries},
    }
    \toprule
    \( U_S \) / \si{\volt} 
        & \( I_S \) / \si{\milli\ampere} \\
    \midrule
    0   & 0.000 \\
    5   & 0.001 \\
    10  & 0.003 \\
    15  & 0.004 \\
    20  & 0.004 \\
    25  & 0.005 \\
    30  & 0.005 \\
    35  & 0.005 \\
    40  & 0.005 \\
    45  & 0.005 \\
    50  & 0.005 \\
    55  & 0.005 \\
    60  & 0.006 \\
    65  & 0.006 \\
    70  & 0.006 \\
    80  & 0.006 \\
    90  & 0.006 \\
    100 & 0.007 \\
    \bottomrule
    \end{tblr}
\end{table}

\begin{table}[H]
    \centering
    \caption{Messdaten für die Kennlinien.}
    \label{tab:2}
    \begin{tblr}{
        colspec = {c *{4}{c}},  % 1 Spalte für U_S + 4 Spalten für I_S
        row{1} = {font=\bfseries},
        hline{1,2,Z} = {1pt},  % Linien oben, unter Überschrift und unten
    }
    \toprule
        & \( I_H = \SI{2.0}{\ampere} \) 
        & \( I_H = \SI{2.1}{\ampere} \) 
        & \( I_H = \SI{2.3}{\ampere} \) 
        & \( I_H = \SI{2.5}{\ampere} \) \\
    \midrule
    \( U_S \) / \si{\volt} 
        & \( I_S \) / \si{\milli\ampere} 
        & \( I_S \) / \si{\milli\ampere} 
        & \( I_S \) / \si{\milli\ampere} 
        & \( I_S \) / \si{\milli\ampere} \\
    \midrule
    0   & 0.000 & 0.000 & 0.000 & 0.000 \\
    5   & 0.010 & 0.012 & 0.020 & 0.023 \\
    10  & 0.023 & 0.030 & 0.051 & 0.062 \\
    15  & 0.035 & 0.051 & 0.090 & 0.109 \\
    20  & 0.042 & 0.069 & 0.133 & 0.160 \\
    25  & 0.046 & 0.079 & 0.179 & 0.217 \\
    30  & 0.049 & 0.086 & 0.216 & 0.276 \\
    35  & 0.050 & 0.091 & 0.255 & 0.329 \\
    40  & 0.052 & 0.094 & 0.292 & 0.321 \\
    45  & 0.053 & 0.097 & 0.292 & 0.358 \\
    50  & 0.054 & 0.099 & 0.309 & 0.397 \\
    55  & 0.055 & 0.100 & 0.341 & 0.444 \\
    60  & 0.056 & 0.102 & 0.365 & 0.484 \\
    70  & 0.056 & 0.103 & 0.406 & 0.594 \\
    80  & 0.056 & 0.100 & 0.438 & 0.686 \\
    90  & 0.059 & 0.105 & 0.466 & 0.791 \\
    100 & 0.060 & 0.108 & 0.488 & 0.890 \\
    110 & 0.061 & 0.111 & 0.513 & 1.010 \\
    120 & 0.062 & 0.113 & 0.540 & 1.121 \\
    130 & 0.063 & 0.114 & 0.549 & 1.228 \\
    140 & 0.064 & 0.116 & 0.554 & 1.336 \\
    150 & 0.064 & 0.117 & 0.559 & 1.436 \\
    160 & 0.065 & 0.118 & 0.563 & 1.516 \\
    170 & 0.066 & 0.119 & 0.566 & 1.592 \\
    180 & 0.066 & 0.119 & 0.569 & 1.666 \\
    190 & 0.066 & 0.119 & 0.573 & 1.746 \\
    200 & 0.066 & 0.120 & 0.576 & 1.822 \\
    210 & 0.066 & 0.120 & 0.579 & 1.903 \\
    220 & 0.066 & 0.120 & 0.583 & 1.969 \\
    230 & 0.066 & 0.120 & 0.586 & 2.040 \\
    240 & 0.066 & 0.120 & 0.590 & 2.090 \\
    250 & 0.066 & 0.120 & 0.593 & 2.140 \\
    \bottomrule
    \end{tblr}
\end{table}
\noindent In \autoref{fig:REFDUHS} ist die Saugspannung gegen den detektierten Strom 
Strom aufgetragen. Dabei ist das sigmoidale Wachstum einigermaßen erkennbar, 
vor Allem jedoch bei den höheren Heizspannungen fällt es einfacher, den Verlauf 
zu identifizieren.
\noindent Mithilfe dieser Daten lassen sich die unterschiedlichen Sättigungsströme 
ablesen, diese lauten wiefolgt:
\begin{figure}[H]
    \centering
    \includegraphics[width=0.95\textwidth]{mw_plot.pdf}
    \caption{Kennlinien bei unterschiedlichen Heizströmen.}
    \label{fig:REFDUHS}  % <- Label NACH der Caption!
\end{figure}
\begin{table}[H]
    \centering
    \caption{Ermittelte Sättigungsströme.}
    \label{tab:t1}
    \begin{tblr}{
        colspec = {S S},
        row{1} = {guard, mode=math},
      }
    \toprule
    I / \si{\volt} & I_H / \si{\milli\ampere}\\
    \midrule
    1.8 & 0.007 \\
    2.0 & 0.066 \\
    2.1 & 0.120 \\
    2.3 & 0.593 \\
    2.5 & 2.140 \\
    \bottomrule 
    \end{tblr}
\end{table}

\subsection{Bestimmung der Kathodentemperatur}

\subsection{Langmuir-Schottky'sche Raumladungsgesetz}

\subsection{Bestimmung der Austrittsarbeit aus dem Kathodenmaterial}

\subsection{Berechnung der Austrittsarbeit}