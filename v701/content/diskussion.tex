\section{Diskussion}
\label{sec:Diskussion}

\subsection{Energieverlust}
Die gemessenen Werte des Energieverlusts beliefen sich auf 
\begin{align*}
    \frac{dE_{\alpha_1}} &= \qty{-81.26(2.24)}{\mega\electronvolt\per\meter}, \\
    \frac{dE_{\alpha_2}} &= \qty{-99.14(3.83)}{\mega\electronvolt\per\meter}
\end{align*}
Die Indizes 1 und 2 stehen dabei für die Messungen bei einer effektiven 
Weglänge von $x = 4 \unit{\centi\meter}$ und $x = 7 \unit{\centi\meter}$.
Die Abweichung der beiden Werte ist mit einem Wert von $18.06 \%$ sehr hoch.
Diese Abweichung ist allerdings zu erwarten gewesen in Anbetracht der Tatsache, 
dass es sich um die Bestimmung über eine Ausgleichsgerade handelt, dessen Werte 
fehlerbelastet sind. Das kommt daher, dass die Einstellungen an der Vakuumpumpe 
per Hand erfolgen und der Druck dementsprechend nach Augenmaß reguliert wird. 
