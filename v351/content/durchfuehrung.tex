\section{Durchführung}
\label{sec:Durchführung}

\subsection{Fourier-Synthese}
Für diesen Teil des Experiments werden das Oszilloskop, der Oberwellengenerator
(siehe \autoref{fig:1}) sowie das Multimeter genutzt. Zu Untersuchen sind die
Rechteck-, Sägezahn- und Dreieckspannung.
Der Oberwellengenerator wird als erstes mit dem 
Multimeter verbunden, woraufhin an diesem die höchstmögliche Spannungsamplitude 
eingestellt wird. Bei Kanal eins befindet sich nun die Spannungsamplitude zu
dem Fourierkoeffizienten $a_1$. Im Folgenden wird dann für die jeweiligen Kanäle 
bei der Rechteck-und Sägezahnspannung mit der Ordnung $n$ das $\frac{1}{n}$-fache
dieser Amplitude eingestellt. Bei der Dreieckspannung werden für die Kanäle 
allerdings das $\frac{1}{n^2}$-fache eingestellt. Detektiert werden die Amplituden 
mithilfe des Multimeters.
\begin{figure}
    \caption{Der Oberwellengenerator.}
    \centering
    \includegraphics[width=0.7\textwidth]{"Bilder/owg.jpg"}
    \label{fig:1}
\end{figure}
Darauffolgend wird der Oberwellengenerator vom Multimeter getrennt und 
stattdessen mit dem Oszilloskop verbunden. Die Phasenverhältnisse der 
Grundschwingung auf Kanal eins werden mit den restlichen Kurven eingestellt, 
indem auf den X-Eingang des Oszilloskops die Oberwelle des ersten Kanals gesetzt 
wird. Die $n$-te Oberwelle des $n$-ten Kanals befindet sich auf dem Y-Eingang.
Auf dem Oszilloskop sind nun die Lissajous-Figuren zu sehen (eine beliebige ist 
in \autoref{fig:2} zu sehen). Grundsätzlich entstehen sie durch die Überlagerung 
zweier orthogonaler harmonischer Schwingungen.
% \begin{figure}[H]
%     \caption{Lissajous-Figur \cite{lissajous}.}
%     \centering
%     \includegraphics[width=\textwidth]{"Bilder/lissajous.jpg"}
%     \label{fig:2}
% \end{figure}
\begin{figure}[H]
    \centering
    % Erstes Bild
    \begin{minipage}{0.48\textwidth}
        \centering
        \includegraphics[width=\linewidth]{"Bilder/lissajous.jpg"}
        \caption{Lissajous-Figur \cite{lissajous}.}
        \label{fig:2}
    \end{minipage}
    \hfill
    % Zweites Bild
    \begin{minipage}{0.48\textwidth}
        \centering
        \includegraphics[width=0.9\linewidth, angle=-90]{"Bilder/la.jpg"}
        \caption{Variierte Lissajous-Figur.}
        \label{fig:3}
    \end{minipage}
\end{figure}
\noindent Diese werden abgeglichen, indem das virtuelle Bilder dermaßen geändert wird, 
dass sich die Kurven überlagern wie in \autoref{fig:3} dargestellt.
Für die Sägezahnspannung muss jede Oberwelle ausgegeben gleichzeitig ausgegeben 
und auf dem Oszilloskop angezeigt werden.
Für die Rechteck-und Dreieckspannung reicht es, jede zweite Oberwelle derartig 
zu detektieren.

\subsection{Fourier-Analyse}