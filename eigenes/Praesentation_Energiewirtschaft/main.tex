% --- Dies ist ein minimales Beamer-Template ---
\documentclass{beamer}

% --- Paket-Definitionen ---
\usepackage[utf8]{inputenc}   % Erlaubt Umlaute (ä, ö, ü) direkt im Code
\usepackage[T1]{fontenc}      % Stellt sicher, dass Umlaute korrekt kopiert/dargestellt werden
\usepackage{graphicx}         % Zum Einfügen von Bildern mit \includegraphics

% --- Theme-Einstellungen (optional) ---
% Du kannst das Aussehen leicht ändern.
% Probiere andere Themes aus, z.B. "Warsaw", "Berlin", "Singapore"
\usetheme{Madrid}
\usecolortheme{beaver}

% --- Titel-Informationen ---
\title[Kurztitel]{Der vollständige Titel deiner Präsentation}
\author[Mustermann]{Max Mustermann}
\institute[Meine Uni]{Meine Universität \\ Fakultät für Irgendwas}
\date{\today} % Setzt das heutige Datum

% --- Dokument-Start ---
\begin{document}

% --- 1. Die Titelseite ---
% Erstellt eine Titelseite basierend auf den obigen Infos
\begin{frame}
  \titlepage
\end{frame}

% --- 2. Das Inhaltsverzeichnis ---
% Erstellt automatisch ein Verzeichnis basierend auf \section
\begin{frame}
  \frametitle{Inhaltsverzeichnis} % Titel für diese Folie
  \tableofcontents
\end{frame}

% --- 3. Inhalt (Abschnitte und Folien) ---

% Jeder \section-Befehl erstellt einen neuen Eintrag im Inhaltsverzeichnis
\section{Einleitung}

\begin{frame}
  \frametitle{Was ist das Problem?} % Titel für diese Folie

  Willkommen zu meiner Präsentation.
  \begin{itemize}
    \item Das ist der erste Punkt.
    \item Und ein zweiter Punkt.
  \end{itemize}
\end{frame}

\begin{frame}
  \frametitle{Ein Bild einfügen}
  
  % (Dafür muss ein Ordner "bilder" mit "logo.png" existieren)
  % \includegraphics[width=0.6\textwidth]{bilder/logo.png}
  
  Hier könnte auch Text stehen.
\end{frame}

\section{Hauptteil}

\begin{frame}
  \frametitle{Die wichtigsten Ergebnisse}
  
  Hier stehen die Kernergebnisse.
  \begin{enumerate}
    \item Erstens...
    \item Zweitens...
    \item Drittens...
  \end{enumerate}
\end{frame}

\section{Zusammenfassung}

\begin{frame}
  \frametitle{Fazit}
  
  Vielen Dank für Ihre Aufmerksamkeit!
  
  \bigskip % Macht einen kleinen vertikalen Abstand
  
  Haben Sie noch Fragen?
\end{frame}

% --- Dokument-Ende ---
\end{document}